\documentclass[10pt,letterpaper]{article}

\usepackage{geometry}
\usepackage{hyperref}
\usepackage{graphicx}

\geometry{
	body={7.0in, 10.0in},
	left=0.75in,
	top=0.75in
}

\hypersetup{
	colorlinks = true,
	urlcolor = black,
	pdfauthor = {Alexander Brown},
	pdfkeywords = {},
	pdftitle = {Alexander Brown: CS22510 - Assignment 1},
	pdfsubject = {CS22510 - Assignment 1},
	pdfpagemode = Default
}

\setlength\parindent{0em}
\setlength{\parskip}{1ex plus 0.5ex minus 0.2ex}
\setcounter{tocdepth}{2}


\title{CS22510 - Assignment 1}
\author{Alexander Brown}
\begin{document}
	\maketitle
	\newpage	
		\section{File Layouts}
			\subsection{logfile.log}
				The format for the logfile is:
				
				\verb+yyyy-mm-dd-hh-mm-ss_progr: uid action path/file+

				

			\subsection{.auth}
				The format of the .auth file is:

				\verb+user group password+
				
				I also included a way of specifying comments (using the `\verb+#+' charcter) to allow leeway.
			
		\section{Language Choices}
			I chose Java for the Staff Program as I believed it to be somewhat tougher with file handlling, but easier to work with Objects like Strings.
			
			I then chose C++ for the somewhat more complex Administrator Program, as I knew I would be able to handle the complexity better due to being more practised with Object Orientated programming. This left the Student Program for C. This played out fairly well as it seemed to have the least string manipulation.

		\section{Program Description}

			All programs have the same initial behaviour; the config file is found and the base directory is read. Then the user is prompted for a user ID, followed by a password (which is masked by disabling the stdin echo in C and C++, and using a system password call in Java). This is then made into a string, with the name of the program to match how it would appear in the .auth file. The .auth file is then opened and read line by line until either the two strings match, in which case the user is logged in and passed to the menu, or the end of the file is reached, in which case an error message is printed and the program exits.

			\subsection{Adminstrator Program Description}
				The administrator has four primary functions; uploading an assignment, creating a staff user, creating a student user and reading the logfile. Uploading an assignment is a matter of specifying a filepath to the file to upload as an assignment. A File Handller static function is then called with the location to upload (repository/assignment/<filename>), which in turn calls the copy static function.

				Creating a staff and student user is almost exactly the same and as such they both call a private function to create a general user, passing in the type of user they are as an enum. This then creates a new file handller object for the file, generates a random password and formats the information into a string. This string is then passed into the file handller's append function. This function creates a new lock for the file and appends the input string to the file. The file lock is automatically released due to the destructor. These functions then return the generated password, which is printed to the administrator. The create student function also creates a directory for the uid of the student in the students directory, followed by a results directory in this directory.

				These two functions also append what they did to the log file in a similar way to how the .auth file is appended to.

				Reading the log file involves reading each line of the logfile, spliting and formatting it, then printing it to the screen.

			\subsection{Student Program}
				The student program has three primary features; downloading the assignment, uploading a submission and viewing the marks for a submission.

				Downloading the assignment is simply a matter of asking the user where it should be downloaded to, then copying from the assignment directory to that specified location.

				Uploading a submission is also fairly simple, copying from the specified location of the submission to the student's directory.

				Viewing the marks is also fairly simple, just searching the results directory of the student's directory for the file result.txt. If it's there, split and format it somewhat nicely, work out the grade achieved using if statements and printing.

				Appending to the logfile is also fairly simple, first off the logfile is locked, then opened with the append mode on. Then writing the formatted string and closing and unlocking the file.

			\subsection{Staff Program}
                                Much of the file handlling in the student program is modelled on how the admin program does the same. Refer to the above section for more information.

				The Staff Program has three main features; listing all students that have made submissions, downloading the submission of a single student and giving marks for a specific student's submission.

				Listing all students that have made submissions is a case of looping through the students directory, entrying each student's directory in turn and seeing if there's any files there other than the results directory. If there is it's printed.

				Downloading is a case of locking the file, copying it to a specified location then unlocking it.

				Giving marks is simply taking input from the keyboard and putting it into a text file in the right format.

		\section{Evaluation}
			At current the program is still fairly rough around the edges, it assumes all users are fairly competent and portable to systems other than *NIX, due to both the Makefile and the file locking. However it does meet the requirements fairly well and I have learned a fair amount about file handlling in all languages (unfortunately Java's new io libraries aren't yet fully available as they would have made some tasks much easier). I also took the time to learn a lot more about makefiles as it did make the project a lot easier to debug and run (and also being able to create the directory structure using BASH commands did make life a lot easier).

			If I'd had more time I probably would have made changes to the file locking, creating a temporary <filename>.lock file instead of relying on external, non-portable libraries to ensure locking was done correctly. Written inside these files would be the pid of the program locking them. Though this would probably produce a little more overhead, it would have made debugging easier.

			I would also have changed to the way logging occured; possibly to the extent of using a threaded dqueue (new entries are added to the back of the dqueue, then all the while to top element is popped, attempted to be written to the logfile. If it fails it's put back onto the top of the queue, otherwise it's done with) to ensure loggin happens without affecting the user. Currently this probably wouldn't be a problem due to the small userbase, however it's likely the with increased traffic getting locked out of the logfile could be an issue and at current the program would either give up logging or wait until it can log, either way stopping the user from continuing.

			Finally I would have made the admin program set up the directory stucture on the first run instead of the makefile to allow for increased portability.

		\section{Run Example}
			\section{Example Usage}

\begin{verbatim}
$ make

$ ./admin_program
Enter User ID: softly
softly's password:
Enter option code (h for help): h
        c [file]: Upload Assignment
        s [uid]: Create a new Staff User
        d [uid]: Create a new Student User
        l: Read the log file
Enter option code (h for help): c
Enter the file (including path) of the assignment: assignment.txt

$ ./admin_program
Enter User ID: softly
softly's password:
Enter option code (h for help): s
Enter the UID for the new staff user: fwl
Created Staff User: fwl
password: xf\xgDq\L

$ ./admin_program
Enter User ID: softly
softly's password:
Enter option code (h for help): d
Enter the UID for the new student user: adb9
Created Student User: adb9
password: pIgC2Aw7>]b

$ ./student_prog
Enter User ID: adb9
adb9's password:
Enter option code (h for help): h
        g - Get Assignment.
        s - Submit Assignment.
        m - View Marks.
        q - Quit.
Enter option code (h for help): g
Location to download the assignment to: download-assign.txt

$./student_prog
Enter User ID: adb9
adb9's password:
Enter option code (h for help): s
Location of file to submit: submission.txt

$ java -jar staff_prog.jar
Enter User ID: fwl
Password for fwl:
Enter option code (h for help): h
        l - List all students with submitted work.
        g - Download a specified student's work.
        m - Submit a mark for a specified student's work.
        q - Quit the program.
Enter option code (h for help): l
The following students have made submissions:
        adb9

$ ./admin_prog
Enter User ID: softly
softly's password:
Enter option code (h for help): d
Enter the UID for the new student user: crl9
Created Student User: crl9
password: qepA=\2236

$ ./student_prog
Enter User ID: crl9
crl9's password:
Enter option code (h for help): s
Location of file to submit: submission.txt

$ java -jar staff_prog.jar
Enter User ID: fwl
Password for fwl:
Enter option code (h for help): l
The following students have made submissions:
        crl9
        adb9

$ java -jar staff_prog.jar
Enter User ID: fwl
Password for fwl:
Enter option code (h for help): g
Enter the uid of the student to get the submission of: adb9
Enter the location you wish to download this submission to: adb9-to-mark.txt

$ java -jar staff_prog.jar
Enter User ID: fwl
Password for fwl:
Enter option code (h for help): m
Enter the uid of the student to submit a mark for: adb9
/home/softly/svn/assign_cs22510/trunk/repository/students/adb9/results/result.txt
Enter the mark (%): 75
Enter any comments:
A good piece of work, however some of the features I had asked for were not as good as expected. Also you need to work on ensuring the C elements work together properly. All in all a good piece of work though.

$ ./student_prog
Enter User ID: adb9
adb9's password:
Enter option code (h for help): m
Results for user 'adb9':
        Mark: 75% (A-)
        Comments: A good piece of work, however some of the features I had asked for were not as good as expected. Also you need to work on ensuring the C elements work together properly. All in all a good piece of work

$ ./admin_prog
Enter User ID: softly
softly's password:
Enter option code (h for help): l
========================================
Date:           24/03/2011
Time:           16:01:13
User:           softly
Program:        init
Action:         Initilisation
File:           repository/
========================================
Date:           24/03/2011
Time:           16:01:13
User:           softly
Program:        init
Action:         Initilisation
File:           repository/logfile.log
========================================
Date:           24/03/2011
Time:           16:01:13
User:           softly
Program:        init
Action:         Initilisation
File:           repository/.auth
========================================
Date:           24/03/2011
Time:           16:01:13
User:           softly
Program:        init
Action:         Created user softly
File:           repository/.auth
========================================
Date:           24/03/2011
Time:           16:01:13
User:           softly
Program:        init
Action:         Initilisation
File:           repository/assignment/
========================================
Date:           24/03/2011
Time:           16:01:13
User:           softly
Program:        init
Action:         Initilisation
File:           repository/students/
========================================
Date:           24/03/2011
Time:           16:02:42
User:           softly
Program:        admin
Action:         Uploaded Assignment
File:           repository/assignment/assignment.txt
========================================
Date:           24/03/2011
Time:           16:03:55
User:           softly
Program:        admin
Action:         Created Staff User
File:           repository/.auth
========================================
Date:           24/03/2011
Time:           16:04:39
User:           softly
Program:        admin
Action:         Created Student User
File:           repository/.auth
========================================
Date:           24/03/2011
Time:           16:06:09
User:           adb9
Program:        stdnt
Action:         Downloaded assignment
File:           repository/assignment/
========================================
Date:           24/03/2011
Time:           16:22:04
User:           adb9
Program:        stdnt
Action:         Downloaded assignment
File:           repository/assignment/
========================================
Date:           24/03/2011
Time:           16:26:38
User:           adb9
Program:        stdnt
Action:         Uploaded assignment
File:           repository/students/adb9/
========================================
Date:           24/03/2011
Time:           16:27:27
User:           adb9
Program:        stdnt
Action:         Uploaded assignment
File:           repository/students/adb9/
========================================
Date:           24/03/2011
Time:           16:28:09
User:           adb9
Program:        stdnt
Action:         Uploaded assignment
File:           repository/students/adb9/
========================================
Date:           24/03/2011
Time:           16:28:41
User:           adb9
Program:        stdnt
Action:         Uploaded assignment
File:           repository/students/adb9/
========================================
Date:           24/03/2011
Time:           16:29:25
User:           fwl
Program:        staff
Action:         Listed Submissions
File:           repository/students/
========================================
Date:           24/03/2011
Time:           16:30:17
User:           softly
Program:        admin
Action:         Created Student User
File:           repository/.auth
========================================
Date:           24/03/2011
Time:           16:30:54
User:           crl9
Program:        stdnt
Action:         Uploaded assignment
File:           repository/students/crl9/
========================================
Date:           24/03/2011
Time:           16:32:58
User:           fwl
Program:        staff
Action:         Listed Submissions
File:           repository/students/
========================================
Date:           24/03/2011
Time:           16:33:48
User:           fwl
Program:        staff
Action:         Downloaded assignment
File:           repository/students/adb9/submission.txt
========================================
Date:           24/03/2011
Time:           16:35:44
User:           fwl
Program:        staff
Action:         Submitted mark for adb9
File:           repository/students/adb9/results/result.txt
========================================

\end{verbatim}


		\section{Source code - Adminisrator Program}
\normalsize
\rmfamily
\subsection{FileHandller.cpp}
\scriptsize
\sffamily
% Generator: GNU source-highlight, by Lorenzo Bettini, http://www.gnu.org/software/src-highlite
\noindent
\mbox{}\textit{/*\ FileHandller.cpp\ Copyright\ (c)\ Alexander\ Brown,\ March\ 2011\ */} \\
\mbox{} \\
\mbox{}\textbf{\#include}\ \texttt{$<$string$>$} \\
\mbox{}\textbf{\#include}\ \texttt{$<$fstream$>$} \\
\mbox{}\textbf{\#include}\ \texttt{$<$ios$>$} \\
\mbox{}\textbf{\#include}\ \texttt{$<$iostream$>$} \\
\mbox{} \\
\mbox{}\textbf{\#include}\ \texttt{"{}FileHandller.h"{}} \\
\mbox{}\textbf{\#include}\ \texttt{"{}FileLock.h"{}} \\
\mbox{} \\
\mbox{}\textbf{using}\ \textbf{namespace}\ std; \\
\mbox{} \\
\mbox{}FileHandller::\textbf{FileHandller}(std::string\ file)\ :\ \textbf{$\_$filename}(file)\ \{ \\
\mbox{}\} \\
\mbox{} \\
\mbox{}FileHandller::\textasciitilde{}\textbf{FileHandller}()\ \{ \\
\mbox{} \\
\mbox{}\} \\
\mbox{} \\
\mbox{}bool\ FileHandller::\textbf{append$\_$file}(std::string\ append$\_$text)\ \{ \\
\mbox{}\ \ \ \ \ \ \ \ \textbf{try}\ \{ \\
\mbox{}\ \ \ \ \ \ \ \ \ \ \ \ \ \ \ \ FileLock\ \textbf{lock}($\_$filename); \\
\mbox{} \\
\mbox{}\ \ \ \ \ \ \ \ \ \ \ \ \ \ \ \ std::ofstream\ dest; \\
\mbox{}\ \ \ \ \ \ \ \ \ \ \ \ \ \ \ \ dest.\textbf{open}($\_$filename.\textbf{data}(),\ ofstream::app); \\
\mbox{}\ \ \ \ \ \ \ \ \ \ \ \ \ \ \ \ dest\ $<$$<$\ append$\_$text\ $<$$<$\ endl; \\
\mbox{}\ \ \ \ \ \ \ \ \ \ \ \ \ \ \ \ dest.\textbf{close}(); \\
\mbox{}\ \ \ \ \ \ \ \ \ \ \ \ \ \ \ \ \textit{//\ lock\ automatically\ freed\ and\ unlocked\ here.} \\
\mbox{}\ \ \ \ \ \ \ \ \ \ \ \ \ \ \ \ \textbf{return}\ \textbf{true}; \\
\mbox{}\ \ \ \ \ \ \ \ \}\ \textbf{catch}\ (std::string\ e)\ \{ \\
\mbox{}\ \ \ \ \ \ \ \ \ \ \ \ \ \ \ \ \textit{//cout\ $<$$<$\ "{}Append\ Failed:\ "{}\ $<$$<$\ e\ $<$$<$\ endl;} \\
\mbox{}\ \ \ \ \ \ \ \ \ \ \ \ \ \ \ \ \textbf{return}\ \textbf{false}; \\
\mbox{}\ \ \ \ \ \ \ \ \} \\
\mbox{}\} \\
\mbox{} \\
\mbox{}bool\ FileHandller::\textbf{download$\_$file}(std::string\ dest$\_$filename)\ \textbf{throw}\ (std::string)\ \{ \\
\mbox{}\ \ \ \ \ \ \ \ \textbf{try}\ \{ \\
\mbox{}\ \ \ \ \ \ \ \ \ \ \ \ \ \ \ \ FileLock\ \textbf{lock}($\_$filename); \\
\mbox{}\ \ \ \ \ \ \ \ \ \ \ \ \ \ \ \ \textbf{return}\ FileHandller::\textbf{copy}($\_$filename,\ dest$\_$filename); \\
\mbox{}\ \ \ \ \ \ \ \ \ \ \ \ \ \ \ \ \textit{//\ lock\ automatically\ freed\ and\ unlocked\ here.} \\
\mbox{}\ \ \ \ \ \ \ \ \}\ \textbf{catch}\ (std::string\ e)\ \{ \\
\mbox{}\ \ \ \ \ \ \ \ \ \ \ \ \ \ \ \ cout\ $<$$<$\ \texttt{"{}Download\ Failed:\ "{}}\ $<$$<$\ e\ $<$$<$\ endl; \\
\mbox{}\ \ \ \ \ \ \ \ \} \\
\mbox{}\} \\
\mbox{} \\
\mbox{}bool\ FileHandller::\textbf{upload$\_$file}(std::string\ src$\_$filename\ , \\
\mbox{}\ \ \ \ \ \ \ \ \ \ \ \ \ \ \ \ std::string\ dest$\_$filename)\ \textbf{throw}\ (std::string)\{ \\
\mbox{}\ \ \ \ \ \ \ \ \textbf{try}\ \{ \\
\mbox{}\ \ \ \ \ \ \ \ \ \ \ \ \ \ \ \ FileLock\ \textbf{lock}(dest$\_$filename);\ \textit{//Lock\ the\ file\ if\ it\ exists.} \\
\mbox{}\ \ \ \ \ \ \ \ \}\ \textbf{catch}\ (std::string\ e)\ \{ \\
\mbox{}\ \ \ \ \ \ \ \ \ \ \ \ \ \ \ \ \textbf{if}(e\ ==\ \texttt{"{}File\ Descriptor\ Error"{}})\ \{ \\
\mbox{}\ \ \ \ \ \ \ \ \ \ \ \ \ \ \ \ \ \ \ \ \ \ \ \ \textit{//\ Do\ nothing\ -\ the\ file\ needs\ to\ be\ created\ instead,\ defying\ the\ point\ of\ locking\ it.} \\
\mbox{}\ \ \ \ \ \ \ \ \ \ \ \ \ \ \ \ \}\ \textbf{else}\ \{ \\
\mbox{}\ \ \ \ \ \ \ \ \ \ \ \ \ \ \ \ \ \ \ \ \ \ \ \ \textbf{throw}\ e;\ \textit{//Otherwise\ the\ file\ is\ locked.} \\
\mbox{}\ \ \ \ \ \ \ \ \ \ \ \ \ \ \ \ \} \\
\mbox{}\ \ \ \ \ \ \ \ \} \\
\mbox{}\ \ \ \ \ \ \ \ \textbf{return}\ \textbf{copy}(src$\_$filename,\ dest$\_$filename); \\
\mbox{}\ \ \ \ \ \ \ \ \textit{//\ Lock\ automatically\ freed\ and\ unlocked\ here.} \\
\mbox{}\} \\
\mbox{} \\
\mbox{}bool\ FileHandller::\textbf{copy}(std::string\ src$\_$filename,\ std::string\ dest$\_$filename)\ \{ \\
\mbox{}\ \ \ \ \ \ \ \ std::ifstream\ src; \\
\mbox{}\ \ \ \ \ \ \ \ std::ofstream\ dest; \\
\mbox{} \\
\mbox{}\ \ \ \ \ \ \ \ src.\textbf{open}(src$\_$filename.\textbf{data}(),\ std::ios::binary); \\
\mbox{}\ \ \ \ \ \ \ \ dest.\textbf{open}(dest$\_$filename.\textbf{data}(),\ std::ios::binary); \\
\mbox{} \\
\mbox{}\ \ \ \ \ \ \ \ \textbf{if}\ (!src.\textbf{is$\_$open}()\ $|$$|$\ !dest.\textbf{is$\_$open}())\ \{ \\
\mbox{}\ \ \ \ \ \ \ \ \ \ \ \ \textbf{return}\ \textbf{false}; \\
\mbox{}\ \ \ \ \ \ \ \ \} \\
\mbox{}\ \ \ \ \ \ \ \ dest\ $<$$<$\ src.\textbf{rdbuf}(); \\
\mbox{} \\
\mbox{}\ \ \ \ \ \ \ \ dest.\textbf{close}(); \\
\mbox{}\ \ \ \ \ \ \ \ src.\textbf{close}(); \\
\mbox{}\ \ \ \ \ \ \ \ \textbf{return}\ \textbf{true}; \\
\mbox{}\} \\

\clearpage
\normalsize
\rmfamily
\subsection{FileHandller.h}
\scriptsize
\sffamily
% Generator: GNU source-highlight, by Lorenzo Bettini, http://www.gnu.org/software/src-highlite
\noindent
\mbox{}\textit{/*\ FileHandeller.h\ \ Copyright\ (c)\ Alexander\ Brown,\ March\ 2011\ */} \\
\mbox{} \\
\mbox{}\textbf{\#ifndef}\ $\_$FILEHANDLLER$\_$H \\
\mbox{}\textbf{\#define}\ \ \ \ \ \ \ \ $\_$FILEHANDLLER$\_$H \\
\mbox{} \\
\mbox{}\textit{/**} \\
\mbox{}\textit{\ *\ }@brief\textit{\ A\ file\ handller\ for\ a\ single\ file.} \\
\mbox{}\textit{\ *} \\
\mbox{}\textit{\ *\ Due\ to\ the\ nature\ of\ the\ repository\ and\ the\ need\ for\ locking,\ it\ is\ best\ that} \\
\mbox{}\textit{\ *\ all\ file\ operations\ are\ handlled\ properly.\ This\ class\ enforces\ the\ locking} \\
\mbox{}\textit{\ *\ mechanism\ (calling\ the\ File\ Lock\ class)\ and\ performs\ basic\ operations\ with} \\
\mbox{}\textit{\ *\ files\ found\ in\ the\ repository.} \\
\mbox{}\textit{\ *} \\
\mbox{}\textit{\ *\ }@author\textit{\ Alexander\ Brown} \\
\mbox{}\textit{\ *\ }@version\textit{\ 1.0} \\
\mbox{}\textit{\ *\ }@see\textit{\ FileLock} \\
\mbox{}\textit{\ */} \\
\mbox{}\textbf{class}\ FileHandller\ \{ \\
\mbox{}\textbf{public}: \\
\mbox{}\ \ \ \ \ \ \ \ \textit{/**} \\
\mbox{}\textit{\ \ \ \ \ \ \ \ \ *\ Creates\ a\ file\ handller\ for\ a\ given\ file.} \\
\mbox{}\textit{\ \ \ \ \ \ \ \ \ *\ }@param\textit{\ filename\ \ \ \ \ \ The\ file\ to\ work\ with.} \\
\mbox{}\textit{\ \ \ \ \ \ \ \ \ */} \\
\mbox{}\ \ \ \ \ \ \ \ \textbf{FileHandller}(std::string\ filename); \\
\mbox{} \\
\mbox{}\ \ \ \ \ \ \ \ \textit{/**} \\
\mbox{}\textit{\ \ \ \ \ \ \ \ \ *\ Destructor.} \\
\mbox{}\textit{\ \ \ \ \ \ \ \ \ */} \\
\mbox{}\ \ \ \ \ \ \ \ \textbf{virtual}\ \textasciitilde{}\textbf{FileHandller}(); \\
\mbox{} \\
\mbox{}\ \ \ \ \ \ \ \ \textit{/**} \\
\mbox{}\textit{\ \ \ \ \ \ \ \ \ *\ Uploads\ the\ local\ sorce\ file\ to\ the\ destination\ file\ in\ the\ repository.} \\
\mbox{}\textit{\ \ \ \ \ \ \ \ \ *\ }@param\textit{\ src$\_$filename\ \ The\ location\ of\ the\ local\ file\ to\ upload.} \\
\mbox{}\textit{\ \ \ \ \ \ \ \ \ *\ }@param\textit{\ dest$\_$filename\ The\ destination\ of\ the\ source\ file.} \\
\mbox{}\textit{\ \ \ \ \ \ \ \ \ *\ }@see\textit{\ copy()} \\
\mbox{}\textit{\ \ \ \ \ \ \ \ \ *\ }@return\textit{\ \ \ \ \ \ \ \ \ \ \ \ \ \ \ }\textbf{$<$code$>$}\textit{true}\textbf{$<$/code$>$}\textit{\ -\ If\ the\ operation\ completed\ sucessfully} \\
\mbox{}\textit{\ \ \ \ \ \ \ \ \ */} \\
\mbox{}\ \ \ \ \ \ \ \ \textbf{static}\ bool\ \textbf{upload$\_$file}(std::string\ src$\_$filename, \\
\mbox{}\ \ \ \ \ \ \ \ \ \ \ \ \ \ \ \ std::string\ dest$\_$filename)\ \textbf{throw}\ (std::string); \\
\mbox{} \\
\mbox{}\ \ \ \ \ \ \ \ \textit{/**} \\
\mbox{}\textit{\ \ \ \ \ \ \ \ \ *\ Downloads\ the\ current\ file\ from\ the\ repository.} \\
\mbox{}\textit{\ \ \ \ \ \ \ \ \ *\ }@param\textit{\ dest$\_$filename\ The\ location\ of\ the\ file\ to\ download\ to.} \\
\mbox{}\textit{\ \ \ \ \ \ \ \ \ *\ }@see\textit{\ copy()} \\
\mbox{}\textit{\ \ \ \ \ \ \ \ \ *\ }@return\textit{\ \ \ \ \ \ \ \ \ \ \ \ \ \ \ }\textbf{$<$code$>$}\textit{true}\textbf{$<$/code$>$}\textit{\ -\ If\ the\ operation\ completed\ sucessfully} \\
\mbox{}\textit{\ \ \ \ \ \ \ \ \ */} \\
\mbox{}\ \ \ \ \ \ \ \ bool\ \textbf{download$\_$file}(std::string\ dest$\_$filename)\ \textbf{throw}\ (std::string); \\
\mbox{} \\
\mbox{}\ \ \ \ \ \ \ \ \textit{/**} \\
\mbox{}\textit{\ \ \ \ \ \ \ \ \ *\ Appends\ a\ given\ string\ to\ the\ current\ file.} \\
\mbox{}\textit{\ \ \ \ \ \ \ \ \ *\ }@param\textit{\ append$\_$text\ \ \ The\ string\ to\ append.} \\
\mbox{}\textit{\ \ \ \ \ \ \ \ \ *\ }@return\textit{\ \ \ \ \ \ \ \ \ \ \ \ \ \ \ \ }\textbf{$<$code$>$}\textit{true}\textbf{$<$/code$>$}\textit{\ -\ If\ the\ operation\ completed\ sucessfully} \\
\mbox{}\textit{\ \ \ \ \ \ \ \ \ */} \\
\mbox{}\ \ \ \ \ \ \ \ bool\ \textbf{append$\_$file}(std::string\ append$\_$text); \\
\mbox{}\textbf{private}: \\
\mbox{}\ \ \ \ \ \ \ \ \textit{/**\ The\ current\ file\ in\ the\ repository.\ */} \\
\mbox{}\ \ \ \ \ \ \ \ std::string\ $\_$filename; \\
\mbox{} \\
\mbox{}\ \ \ \ \ \ \ \ \textit{/**} \\
\mbox{}\textit{\ \ \ \ \ \ \ \ \ *\ Copies\ the\ src\ file\ to\ the\ destination\ file.} \\
\mbox{}\textit{\ \ \ \ \ \ \ \ \ *\ }@param\textit{\ src$\_$filename\ \ The\ location\ of\ the\ file\ to\ copy.} \\
\mbox{}\textit{\ \ \ \ \ \ \ \ \ *\ }@param\textit{\ dest$\_$filename\ The\ destination\ of\ the\ source\ file.} \\
\mbox{}\textit{\ \ \ \ \ \ \ \ \ *\ }@return\textit{\ \ \ \ \ \ \ \ \ \ \ \ \ \ \ }\textbf{$<$code$>$}\textit{true}\textbf{$<$/code$>$}\textit{\ -\ If\ the\ operation\ completed\ sucessfully} \\
\mbox{}\textit{\ \ \ \ \ \ \ \ \ */} \\
\mbox{}\ \ \ \ \ \ \ \ \textbf{static}\ bool\ \textbf{copy}(std::string\ src$\_$filename,\ std::string\ dest$\_$filename); \\
\mbox{}\}; \\
\mbox{} \\
\mbox{}\textbf{\#endif}\ \ \ \ \ \ \ \ \textit{/*\ $\_$FILEHANDLLER$\_$H\ */} \\
\mbox{} \\

\clearpage
\normalsize
\rmfamily
\subsection{FileLock.cpp}
\scriptsize
\sffamily
% Generator: GNU source-highlight, by Lorenzo Bettini, http://www.gnu.org/software/src-highlite
\noindent
\mbox{}\textit{/*\ FileLock.cpp\ \ Copyright\ (c)\ Alexander\ Brown,\ March\ 2011\ */} \\
\mbox{} \\
\mbox{}\textbf{\#include}\ \texttt{$<$cstdlib$>$} \\
\mbox{}\textbf{\#include}\ \texttt{$<$fcntl.h$>$} \\
\mbox{}\textbf{\#include}\ \texttt{$<$errno.h$>$} \\
\mbox{}\textbf{\#include}\ \texttt{$<$unistd.h$>$} \\
\mbox{}\textbf{\#include}\ \texttt{$<$stdio.h$>$} \\
\mbox{}\textbf{\#include}\ \texttt{$<$string$>$} \\
\mbox{} \\
\mbox{}\textbf{\#include}\ \texttt{"{}FileLock.h"{}} \\
\mbox{} \\
\mbox{}\textit{/**} \\
\mbox{}\textit{\ *\ Structure\ contaning\ details\ on\ the\ lock.} \\
\mbox{}\textit{\ *\ }@author\textit{\ Neal\ Snooke} \\
\mbox{}\textit{\ */} \\
\mbox{}\textbf{struct}\ flock*\ \textbf{file$\_$lock}(short\ type,\ short\ whence)\ \{ \\
\mbox{}\ \ \ \ \textbf{static}\ \textbf{struct}\ flock\ ret; \\
\mbox{}\ \ \ \ ret.l$\_$type\ =\ type; \\
\mbox{}\ \ \ \ ret.l$\_$start\ =\ 0; \\
\mbox{}\ \ \ \ ret.l$\_$whence\ =\ whence; \\
\mbox{}\ \ \ \ ret.l$\_$len\ =\ 0; \\
\mbox{}\ \ \ \ ret.l$\_$pid\ =\ \textbf{getpid}(); \\
\mbox{}\ \ \ \ \textbf{return}\ \&ret; \\
\mbox{}\} \\
\mbox{} \\
\mbox{}FileLock::\textbf{FileLock}(std::string\ filename)\ \textbf{throw}\ (std::string)\ \{ \\
\mbox{}\ \ \ \ $\_$file$\_$des\ =\ \textbf{open}(filename.\textbf{data}(),\ O$\_$RDWR); \\
\mbox{} \\
\mbox{}\ \ \ \ \textbf{if}\ ($\_$file$\_$des\ ==\ -1)\ \{ \\
\mbox{}\ \ \ \ \ \ \ \ \textbf{throw}\ std::\textbf{string}(\texttt{"{}File\ Descriptor\ Error"{}}); \\
\mbox{}\ \ \ \ \} \\
\mbox{}\ \ \ \ \textbf{lock}(); \\
\mbox{}\} \\
\mbox{} \\
\mbox{}FileLock::\textasciitilde{}\textbf{FileLock}()\ \{ \\
\mbox{}\ \ \ \ \textbf{unlock}(); \\
\mbox{}\ \ \ \ \textbf{close}($\_$file$\_$des); \\
\mbox{}\} \\
\mbox{} \\
\mbox{}void\ FileLock::\textbf{lock}()\ \textbf{throw}\ (std::string)\ \{ \\
\mbox{}\ \ \ \ $\_$file$\_$lock\ =\ \textbf{file$\_$lock}(F$\_$WRLCK,\ SEEK$\_$SET); \\
\mbox{} \\
\mbox{}\ \ \ \ \textbf{if}\ (\textbf{fcntl}($\_$file$\_$des,\ F$\_$SETLK,\ $\_$file$\_$lock)\ ==\ -1)\ \{ \\
\mbox{}\ \ \ \ \ \ \ \ \textbf{if}\ (errno\ ==\ EACCES\ $|$$|$\ errno\ ==\ EAGAIN)\ \{ \\
\mbox{}\ \ \ \ \ \ \ \ \ \ \ \ \textbf{throw}\ std::\textbf{string}(\texttt{"{}Already\ locked\ by\ another\ process"{}}); \\
\mbox{}\ \ \ \ \ \ \ \ \}\ \textbf{else}\ \{ \\
\mbox{}\ \ \ \ \ \ \ \ \ \ \ \ \textbf{throw}\ std::\textbf{string}(\texttt{"{}Unkown\ Error"{}}); \\
\mbox{}\ \ \ \ \ \ \ \ \} \\
\mbox{}\ \ \ \ \} \\
\mbox{}\} \\
\mbox{} \\
\mbox{}void\ FileLock::\textbf{unlock}()\ \{ \\
\mbox{}\ \ \ \ \textbf{fcntl}($\_$file$\_$des,\ F$\_$SETLKW,\ \textbf{file$\_$lock}(F$\_$UNLCK,\ SEEK$\_$SET)); \\
\mbox{}\} \\
\mbox{} \\
\mbox{}int\ FileLock::\textbf{get$\_$locked$\_$file}()\ \{ \\
\mbox{}\ \ \ \ \textbf{return}\ $\_$file$\_$des; \\
\mbox{}\} \\

\clearpage
\normalsize
\rmfamily
\subsection{FileLock.h}
\scriptsize
\sffamily
% Generator: GNU source-highlight, by Lorenzo Bettini, http://www.gnu.org/software/src-highlite
\noindent
\mbox{}\textit{/*\ FileLock.h\ \ Copyright\ (c)\ Alexander\ Brown,\ March\ 2011\ */} \\
\mbox{} \\
\mbox{}\textbf{\#ifndef}\ $\_$FILELOCK$\_$H \\
\mbox{}\textbf{\#define}\ \ \ \ \ \ \ \ $\_$FILELOCK$\_$H \\
\mbox{} \\
\mbox{}\textbf{\#include}\ \texttt{$<$string$>$} \\
\mbox{} \\
\mbox{}\textit{/**} \\
\mbox{}\textit{\ *\ }@brief\textit{\ Locks\ a\ File\ so\ that\ only\ this\ process\ can\ use\ it.} \\
\mbox{}\textit{\ *} \\
\mbox{}\textit{\ *\ }@author\textit{\ Alexander\ Brown} \\
\mbox{}\textit{\ *\ }@author\textit{\ Neil\ Snooke\ (Locking\ Code).} \\
\mbox{}\textit{\ */} \\
\mbox{}\textbf{class}\ FileLock\ \{ \\
\mbox{}\textbf{public}: \\
\mbox{}\ \ \ \ \textit{/**} \\
\mbox{}\textit{\ \ \ \ \ *\ Creates\ a\ Lock\ on\ the\ given\ file.} \\
\mbox{}\textit{\ \ \ \ \ *\ }@param\textit{\ filename\ \ The\ file\ to\ lock.} \\
\mbox{}\textit{\ \ \ \ \ *\ }@throws\textit{\ std::string\ If\ the\ lock\ fails.} \\
\mbox{}\textit{\ \ \ \ \ */} \\
\mbox{}\ \ \ \ \textbf{FileLock}(std::string\ filename)\ \textbf{throw}\ (std::string); \\
\mbox{} \\
\mbox{}\ \ \ \ \textit{/**} \\
\mbox{}\textit{\ \ \ \ \ *\ Destructs,\ freeing\ the\ lock\ in\ the\ process.} \\
\mbox{}\textit{\ \ \ \ \ */} \\
\mbox{}\ \ \ \ \textbf{virtual}\ \textasciitilde{}\textbf{FileLock}(); \\
\mbox{} \\
\mbox{}\ \ \ \ \textit{/**} \\
\mbox{}\textit{\ \ \ \ \ *\ Returns\ the\ file\ descriptor\ of\ the\ locked\ file.} \\
\mbox{}\textit{\ \ \ \ \ *\ }@return\textit{\ The\ file\ descriptor\ of\ the\ locked\ file.} \\
\mbox{}\textit{\ \ \ \ \ *\ }@deprecated\textit{\ Use\ of\ std::fstream\ makes\ this\ unecessary.} \\
\mbox{}\textit{\ \ \ \ \ */} \\
\mbox{}\ \ \ \ int\ \textbf{get$\_$locked$\_$file}(); \\
\mbox{} \\
\mbox{}\textbf{private}: \\
\mbox{}\ \ \ \ \textit{/**\ A\ structure\ containing\ the\ details\ of\ the\ file\ lock.\ */} \\
\mbox{}\ \ \ \ \textbf{struct}\ flock*\ $\_$file$\_$lock; \\
\mbox{} \\
\mbox{}\ \ \ \ \textit{/**\ The\ file\ descriptor\ of\ the\ locked\ file.\ */} \\
\mbox{}\ \ \ \ int\ $\_$file$\_$des; \\
\mbox{} \\
\mbox{}\ \ \ \ \textit{/**} \\
\mbox{}\textit{\ \ \ \ \ *\ Performs\ the\ lock\ operation.} \\
\mbox{}\textit{\ \ \ \ \ *\ }@throws\textit{\ std::string\ If\ the\ operation\ was\ not\ sucessful.} \\
\mbox{}\textit{\ \ \ \ \ */} \\
\mbox{}\ \ \ \ void\ \textbf{lock}()\ \textbf{throw}\ (std::string); \\
\mbox{} \\
\mbox{}\ \ \ \ \textit{/**\ Performs\ the\ unlock\ operation.\ */} \\
\mbox{}\ \ \ \ void\ \textbf{unlock}(); \\
\mbox{}\}; \\
\mbox{} \\
\mbox{}\textbf{\#endif}\ \ \ \ \ \ \ \ \textit{/*\ $\_$FILELOCK$\_$H\ */} \\
\mbox{} \\

\clearpage
\normalsize
\rmfamily
\subsection{Logger.cpp}
\scriptsize
\sffamily
% Generator: GNU source-highlight, by Lorenzo Bettini, http://www.gnu.org/software/src-highlite
\noindent
\mbox{}\textit{/*\ Logger.cpp\ \ Copyright\ (c)\ Alexander\ Brown,\ March\ 2011\ */} \\
\mbox{} \\
\mbox{}\textbf{\#include}\ \texttt{$<$string$>$} \\
\mbox{}\textbf{\#include}\ \texttt{$<$bits/basic$\_$string.h$>$} \\
\mbox{}\textbf{\#include}\ \texttt{$<$time.h$>$} \\
\mbox{}\textbf{\#include}\ \texttt{$<$stdio.h$>$} \\
\mbox{}\textbf{\#include}\ \texttt{$<$iostream$>$} \\
\mbox{}\textbf{\#include}\ \texttt{$<$fstream$>$} \\
\mbox{} \\
\mbox{}\textbf{\#include}\ \texttt{"{}Logger.h"{}} \\
\mbox{}\textbf{\#include}\ \texttt{"{}FileHandller.h"{}} \\
\mbox{}\textbf{\#include}\ \texttt{"{}Setup.h"{}} \\
\mbox{}\textbf{\#include}\ \texttt{"{}User.h"{}} \\
\mbox{} \\
\mbox{}\textbf{using}\ \textbf{namespace}\ std; \\
\mbox{} \\
\mbox{}Logger::\textbf{Logger}()\ \{ \\
\mbox{}\ \ \ \ $\_$file$\_$handller\ =\ \textbf{new}\ \textbf{FileHandller}(Setup::LOG$\_$FILE); \\
\mbox{}\} \\
\mbox{} \\
\mbox{}Logger::\textasciitilde{}\textbf{Logger}()\ \{ \\
\mbox{}\ \ \ \ \textbf{delete}\ $\_$file$\_$handller; \\
\mbox{}\ \ \ \ $\_$user\ =\ NULL; \\
\mbox{}\} \\
\mbox{} \\
\mbox{}void\ Logger::\textbf{set$\_$user}(User*\ user)\ \{ \\
\mbox{}\ \ \ \ $\_$user\ =\ user; \\
\mbox{}\} \\
\mbox{} \\
\mbox{}void\ Logger::\textbf{log}(std::string\ activity,\ std::string\ filepath)\ \{ \\
\mbox{}\ \ \ \ \ \ \ \ bool\ complete\ =\ $\_$file$\_$handller-$>$\textbf{append$\_$file}(\textbf{generate$\_$log$\_$entry}(activity,\ filepath)); \\
\mbox{}\ \ \ \ \ \ \ \ \textbf{if}(!complete)\ \{ \\
\mbox{}\ \ \ \ \ \ \ \ \ \ \ \ \ \ \ \ \textbf{log}(activity,\ filepath); \\
\mbox{}\ \ \ \ \ \ \ \ \} \\
\mbox{}\} \\
\mbox{} \\
\mbox{}std::string\ Logger::\textbf{generate$\_$log$\_$entry}(std::string\ activity, \\
\mbox{}\ \ \ \ \ \ \ \ std::string\ filepath)\ \{ \\
\mbox{}\ \ \ \ std::string\ entry\ =\ \textbf{encode$\_$date$\_$time}(); \\
\mbox{}\ \ \ \ entry\ +=\ \texttt{"{}\ admin:\ "{}}; \\
\mbox{}\ \ \ \ entry\ +=\ $\_$user-$>$\textbf{get$\_$uid}()\ +\ \texttt{"{}}\texttt{\textbackslash{}t}\texttt{"{}}; \\
\mbox{}\ \ \ \ entry\ +=\ activity\ +\ \texttt{"{}}\texttt{\textbackslash{}t}\texttt{"{}}; \\
\mbox{}\ \ \ \ int\ base$\_$dir$\_$pos\ =\ filepath.\textbf{rfind}(Setup::BASE$\_$DIR); \\
\mbox{}\ \ \ \ filepath\ =\ filepath.\textbf{replace}(base$\_$dir$\_$pos,\ base$\_$dir$\_$pos\ +\ Setup::BASE$\_$DIR.\textbf{length}(),\ \texttt{"{}"{}}); \\
\mbox{}\ \ \ \ filepath\ =\ Setup::REPO$\_$DIR\ +\ filepath; \\
\mbox{}\ \ \ \ entry\ +=\ filepath; \\
\mbox{} \\
\mbox{} \\
\mbox{}\ \ \ \ \textbf{return}\ entry; \\
\mbox{}\} \\
\mbox{} \\
\mbox{}std::string\ Logger::\textbf{encode$\_$date$\_$time}()\ \{ \\
\mbox{}\ \ \ \ time$\_$t\ rawtime; \\
\mbox{}\ \ \ \ \textbf{struct}\ tm\ *\ timeinfo; \\
\mbox{}\ \ \ \ char\ buffer\ [20]; \\
\mbox{} \\
\mbox{}\ \ \ \ \textbf{time}(\&rawtime); \\
\mbox{}\ \ \ \ timeinfo\ =\ \textbf{localtime}(\&rawtime); \\
\mbox{}\ \ \ \ \textbf{strftime}(buffer,\ 20,\ \texttt{"{}\%Y-\%m-\%d-\%H-\%M-\%S"{}},\ timeinfo); \\
\mbox{} \\
\mbox{}\ \ \ \ \textbf{return}\ std::\textbf{string}(buffer); \\
\mbox{}\} \\
\mbox{} \\
\mbox{}void\ Logger::\textbf{read$\_$log}()\ \{ \\
\mbox{}\ \ \ \ ifstream\ \textbf{authstream}(Setup::LOG$\_$FILE.\textbf{data}()); \\
\mbox{} \\
\mbox{}\ \ \ \ \textbf{if}\ (!authstream.\textbf{is$\_$open}())\ \{ \\
\mbox{}\ \ \ \ \ \ \ \ \textbf{throw}\ \textbf{string}(\texttt{"{}Error\ opening\ '"{}}\ +\ Setup::LOG$\_$FILE\ +\ \texttt{"{}'"{}}); \\
\mbox{}\ \ \ \ \} \\
\mbox{}\ \ \ \ string\ line; \\
\mbox{}\ \ \ \ \textbf{getline}(authstream,\ line); \\
\mbox{} \\
\mbox{} \\
\mbox{}\ \ \ \ \textbf{while}\ (authstream.\textbf{good}())\ \{ \\
\mbox{}\ \ \ \ \ \ \ \ \textbf{if}\ (line.\textbf{data}()[0]\ !=\ \texttt{'\#'})\ \{ \\
\mbox{}\ \ \ \ \ \ \ \ \ \ \ \ cout\ $<$$<$\ \texttt{"{}========================================"{}}\ $<$$<$\ endl; \\
\mbox{}\ \ \ \ \ \ \ \ \ \ \ \ std::string\ date$\_$time\ =\ line.\textbf{substr}(0,\ 20); \\
\mbox{}\ \ \ \ \ \ \ \ \ \ \ \ line\ =\ line.\textbf{substr}(20,\ line.\textbf{length}()); \\
\mbox{}\ \ \ \ \ \ \ \ \ \ \ \ date$\_$time\ =\ \textbf{decode$\_$date$\_$time}(date$\_$time); \\
\mbox{} \\
\mbox{}\ \ \ \ \ \ \ \ \ \ \ \ std::string\ user$\_$type\ =\ line.\textbf{substr}(0,5); \\
\mbox{}\ \ \ \ \ \ \ \ \ \ \ \ line\ =\ line.\textbf{substr}(7,\ line.\textbf{length}()); \\
\mbox{} \\
\mbox{}\ \ \ \ \ \ \ \ \ \ \ \ std::string\ user\ =\ line.\textbf{substr}(0,\ line.\textbf{find}(\texttt{"{}}\texttt{\textbackslash{}t}\texttt{"{}})); \\
\mbox{}\ \ \ \ \ \ \ \ \ \ \ \ line\ =\ line.\textbf{substr}(line.\textbf{find}(\texttt{"{}}\texttt{\textbackslash{}t}\texttt{"{}})+1,\ line.\textbf{length}()); \\
\mbox{} \\
\mbox{}\ \ \ \ \ \ \ \ \ \ \ \ std::string\ action\ =\ line.\textbf{substr}(0,\ line.\textbf{find}(\texttt{"{}}\texttt{\textbackslash{}t}\texttt{"{}})); \\
\mbox{}\ \ \ \ \ \ \ \ \ \ \ \ line\ =\ line.\textbf{substr}(line.\textbf{find}(\texttt{"{}}\texttt{\textbackslash{}t}\texttt{"{}})+1,\ line.\textbf{length}()); \\
\mbox{}\ \ \ \ \ \ \ \ \ \ \ \  \\
\mbox{}\ \ \ \ \ \ \ \ \ \ \ \ cout\ $<$$<$\ date$\_$time; \\
\mbox{}\ \ \ \ \ \ \ \ \ \ \ \ cout\ $<$$<$\ \texttt{"{}User:}\texttt{\textbackslash{}t\textbackslash{}t}\texttt{"{}}\ $<$$<$\ user\ $<$$<$\ endl; \\
\mbox{}\ \ \ \ \ \ \ \ \ \ \ \ cout\ $<$$<$\ \texttt{"{}Program:}\texttt{\textbackslash{}t}\texttt{"{}}\ $<$$<$\ user$\_$type\ $<$$<$\ endl; \\
\mbox{}\ \ \ \ \ \ \ \ \ \ \ \ cout\ $<$$<$\ \texttt{"{}Action:}\texttt{\textbackslash{}t\textbackslash{}t}\texttt{"{}}\ $<$$<$\ action\ $<$$<$\ endl; \\
\mbox{}\ \ \ \ \ \ \ \ \ \ \ \ cout\ $<$$<$\ \texttt{"{}File:}\texttt{\textbackslash{}t\textbackslash{}t}\texttt{"{}}\ $<$$<$\ line\ $<$$<$\ endl; \\
\mbox{}\ \ \ \ \ \ \ \ \} \\
\mbox{}\ \ \ \ \ \ \ \ \textbf{getline}(authstream,\ line); \\
\mbox{}\ \ \ \ \} \\
\mbox{}\ \ \ \ cout\ $<$$<$\ \texttt{"{}========================================"{}}\ $<$$<$\ endl; \\
\mbox{} \\
\mbox{}\ \ \ \ authstream.\textbf{close}(); \\
\mbox{}\} \\
\mbox{} \\
\mbox{}std::string\ Logger::\textbf{decode$\_$date$\_$time}(std::string\ date$\_$time)\ \{ \\
\mbox{}\ \ \ \ \textbf{struct}\ tm\ tm; \\
\mbox{}\ \ \ \ \textbf{strptime}(date$\_$time.\textbf{data}(),\ \texttt{"{}\%Y-\%m-\%d-\%H-\%M-\%S\ "{}},\ \&tm); \\
\mbox{}\ \ \ \  \\
\mbox{}\ \ \ \ char\ buffer[80]; \\
\mbox{}\ \ \ \ \textbf{strftime}(buffer,\ 80,\ \texttt{"{}Date:}\texttt{\textbackslash{}t\textbackslash{}t}\texttt{\%d/\%m/\%Y}\texttt{\textbackslash{}n}\texttt{Time:}\texttt{\textbackslash{}t\textbackslash{}t}\texttt{\%H:\%M:\%S}\texttt{\textbackslash{}n}\texttt{"{}},\ \&tm); \\
\mbox{} \\
\mbox{}\ \ \ \ \textbf{return}\ std::\textbf{string}(buffer); \\
\mbox{}\} \\

\clearpage
\normalsize
\rmfamily
\subsection{Logger.h}
\scriptsize
\sffamily
% Generator: GNU source-highlight, by Lorenzo Bettini, http://www.gnu.org/software/src-highlite
\noindent
\mbox{}\textit{/*\ Logger.h\ \ Copyright\ (c)\ Alexander\ Brown,\ March\ 2011\ */} \\
\mbox{} \\
\mbox{}\textbf{\#ifndef}\ $\_$LOGGER$\_$H \\
\mbox{}\textbf{\#define}\ $\_$LOGGER$\_$H \\
\mbox{} \\
\mbox{}\textbf{\#include}\ \texttt{$<$string$>$} \\
\mbox{} \\
\mbox{}\textbf{class}\ User; \\
\mbox{}\textbf{class}\ FileHandller; \\
\mbox{} \\
\mbox{}\textit{/**} \\
\mbox{}\textit{\ *\ }@brief\textit{\ A\ class\ which\ handles\ all\ log\ file\ actions.} \\
\mbox{}\textit{\ *} \\
\mbox{}\textit{\ *\ This\ class\ deals\ with\ all\ operations\ that\ will\ be\ performed\ on} \\
\mbox{}\textit{\ *\ the\ logfile,\ ensuring\ the\ file\ is\ locked\ whilst\ these\ actions} \\
\mbox{}\textit{\ *\ are\ performed,\ etc.} \\
\mbox{}\textit{\ *} \\
\mbox{}\textit{\ *\ It\ also\ ensures\ that\ the\ same\ format\ is\ used,\ so\ that\ the} \\
\mbox{}\textit{\ *\ logfile\ can\ be\ read\ and\ formatted\ back.} \\
\mbox{}\textit{\ *} \\
\mbox{}\textit{\ *\ }@author\textit{\ Alexander\ Brown} \\
\mbox{}\textit{\ *\ }@version\textit{\ 1.0} \\
\mbox{}\textit{\ */} \\
\mbox{}\textbf{class}\ Logger\ \{ \\
\mbox{}\textbf{public}: \\
\mbox{}\ \ \ \ \textbf{Logger}(); \\
\mbox{}\ \ \ \ \textasciitilde{}\textbf{Logger}(); \\
\mbox{} \\
\mbox{}\ \ \ \ \textit{/**} \\
\mbox{}\textit{\ \ \ \ \ *\ Sets\ user\ to\ be\ the\ current\ User.} \\
\mbox{}\textit{\ \ \ \ \ *} \\
\mbox{}\textit{\ \ \ \ \ *\ }@param\textit{\ user\ The\ current\ User.} \\
\mbox{}\textit{\ \ \ \ \ */} \\
\mbox{}\ \ \ \ void\ \textbf{set$\_$user}(User\ *user); \\
\mbox{} \\
\mbox{}\ \ \ \ \textit{/**} \\
\mbox{}\textit{\ \ \ \ \ *\ Makes\ an\ entry\ to\ the\ logfile.} \\
\mbox{}\textit{\ \ \ \ \ *} \\
\mbox{}\textit{\ \ \ \ \ *\ }@param\textit{\ activity\ \ The\ activity\ that\ was\ performed.} \\
\mbox{}\textit{\ \ \ \ \ *\ }@param\textit{\ filepath\ \ The\ path\ to\ the\ file\ the\ activity\ was\ performed\ on.} \\
\mbox{}\textit{\ \ \ \ \ *} \\
\mbox{}\textit{\ \ \ \ \ *\ }@see\textit{\ Logger::generateLogEntry(std::string,\ std::string)} \\
\mbox{}\textit{\ \ \ \ \ */} \\
\mbox{}\ \ \ \ void\ \textbf{log}(std::string\ activity,\ std::string\ filepath); \\
\mbox{} \\
\mbox{}\ \ \ \ \textit{/**} \\
\mbox{}\textit{\ \ \ \ \ *\ Prints\ the\ logfile\ in\ a\ human\ readable\ format.} \\
\mbox{}\textit{\ \ \ \ \ *} \\
\mbox{}\textit{\ \ \ \ \ *\ }@see\textit{\ Logger::decodeDateTime(std::string)} \\
\mbox{}\textit{\ \ \ \ \ */} \\
\mbox{}\ \ \ \ void\ \textbf{read$\_$log}(void); \\
\mbox{}\textbf{private}: \\
\mbox{}\ \ \ \ \textit{/**\ Aggregation\ to\ the\ User\ */} \\
\mbox{}\ \ \ \ User\ *\ $\_$user; \\
\mbox{} \\
\mbox{}\ \ \ \ FileHandller\ *\ $\_$file$\_$handller; \\
\mbox{} \\
\mbox{}\ \ \ \ \textit{/**} \\
\mbox{}\textit{\ \ \ \ \ *\ Generates\ the\ text\ for\ the\ log\ entry.} \\
\mbox{}\textit{\ \ \ \ \ *} \\
\mbox{}\textit{\ \ \ \ \ *\ }@param\textit{\ activity\ \ The\ activity\ that\ was\ performed.} \\
\mbox{}\textit{\ \ \ \ \ *\ }@param\textit{\ filepath\ \ The\ path\ to\ the\ file\ the\ activity\ was\ performed\ on.} \\
\mbox{}\textit{\ \ \ \ \ *} \\
\mbox{}\textit{\ \ \ \ \ *\ }@return\textit{\ The\ textual\ form\ of\ a\ log\ entry.} \\
\mbox{}\textit{\ \ \ \ \ *} \\
\mbox{}\textit{\ \ \ \ \ *\ }@see\textit{\ Logger::encodeDateTime()} \\
\mbox{}\textit{\ \ \ \ \ */} \\
\mbox{}\ \ \ \ std::string\ \textbf{generate$\_$log$\_$entry}(std::string\ activity,\ std::string\ filepath); \\
\mbox{} \\
\mbox{}\ \ \ \ \textit{/**} \\
\mbox{}\textit{\ \ \ \ \ *\ Encodes\ the\ current\ Date\ and\ Time\ into\ the\ string\ format\ used\ by\ the\ log} \\
\mbox{}\textit{\ \ \ \ \ *\ file.} \\
\mbox{}\textit{\ \ \ \ \ *} \\
\mbox{}\textit{\ \ \ \ \ *\ }@return\textit{\ The\ current\ Date\ and\ Time\ into\ the\ string\ format\ used\ by\ the\ log} \\
\mbox{}\textit{\ \ \ \ \ *\ file.} \\
\mbox{}\textit{\ \ \ \ \ */} \\
\mbox{}\ \ \ \ std::string\ \textbf{encode$\_$date$\_$time}(void); \\
\mbox{} \\
\mbox{}\ \ \ \ \textit{/**} \\
\mbox{}\textit{\ \ \ \ \ *\ Decodes\ the\ Date\ and\ Time\ from\ a\ log\ entry\ into\ a\ human-readable\ format.} \\
\mbox{}\textit{\ \ \ \ \ *} \\
\mbox{}\textit{\ \ \ \ \ *\ }@param\textit{\ date$\_$time\ \ The\ date\ and\ time\ to\ decode\ into\ a\ human\ readable} \\
\mbox{}\textit{\ \ \ \ \ *\ format} \\
\mbox{}\textit{\ \ \ \ \ *\ } \\
\mbox{}\textit{\ \ \ \ \ *\ }@return\textit{\ The\ Date\ and\ Time\ from\ a\ log\ entry\ into\ a\ human-readable\ format.} \\
\mbox{}\textit{\ \ \ \ \ */} \\
\mbox{}\ \ \ \ std::string\ \textbf{decode$\_$date$\_$time}(std::string\ date$\_$time); \\
\mbox{}\}; \\
\mbox{} \\
\mbox{}\textbf{\#endif}\ \ \textit{/*\ $\_$LOGGER$\_$H\ */} \\

\clearpage
\normalsize
\rmfamily
\subsection{main.cpp}
\scriptsize
\sffamily
% Generator: GNU source-highlight, by Lorenzo Bettini, http://www.gnu.org/software/src-highlite
\noindent
\mbox{}\textit{/*\ main.cpp\ \ Copyright\ (c)\ Alexander\ Brown,\ March\ 2011\ */} \\
\mbox{} \\
\mbox{}\textbf{\#include}\ \texttt{$<$stdlib.h$>$} \\
\mbox{}\textbf{\#include}\ \texttt{$<$stdio.h$>$} \\
\mbox{}\textbf{\#include}\ \texttt{$<$iostream$>$} \\
\mbox{} \\
\mbox{}\textbf{\#include}\ \texttt{"{}Setup.h"{}} \\
\mbox{}\textbf{\#include}\ \texttt{"{}User.h"{}} \\
\mbox{}\textbf{\#include}\ \texttt{"{}Logger.h"{}} \\
\mbox{} \\
\mbox{}\textbf{\#include}\ \texttt{"{}UserCreator.h"{}} \\
\mbox{}\textbf{\#include}\ \texttt{"{}UserType.h"{}} \\
\mbox{} \\
\mbox{}\textbf{using}\ \textbf{namespace}\ std; \\
\mbox{} \\
\mbox{}\textbf{\#define}\ FLAG$\_$CREATE$\_$ASSIGN\ \ \texttt{'c'} \\
\mbox{}\textbf{\#define}\ FLAG$\_$CREATE$\_$STAFF\ \ \ \texttt{'s'} \\
\mbox{}\textbf{\#define}\ FLAG$\_$CREATE$\_$STDNT\ \ \ \texttt{'d'} \\
\mbox{}\textbf{\#define}\ FLAG$\_$PRINT$\_$LOG\ \ \ \ \ \ \texttt{'l'} \\
\mbox{}\textbf{\#define}\ FLAG$\_$HELP\ \ \ \ \ \ \ \ \ \ \ \texttt{'h'} \\
\mbox{}\textbf{\#define}\ FLAG$\_$NO$\_$OPTION\ \ \ \ \ \ \texttt{'n'} \\
\mbox{} \\
\mbox{}void\ \textbf{print$\_$help}()\ \{ \\
\mbox{}\ \ \ \ cout\ $<$$<$\ \texttt{"{}}\texttt{\textbackslash{}t}\texttt{"{}}\ $<$$<$\ FLAG$\_$CREATE$\_$ASSIGN\ $<$$<$\ \texttt{"{}\ [file]:\ Upload\ Assignment}\texttt{\textbackslash{}n}\texttt{"{}}; \\
\mbox{}\ \ \ \ cout\ $<$$<$\ \texttt{"{}}\texttt{\textbackslash{}t}\texttt{"{}}\ $<$$<$\ FLAG$\_$CREATE$\_$STAFF\ $<$$<$\ \texttt{"{}\ [uid]:\ Create\ a\ new\ Staff\ User}\texttt{\textbackslash{}n}\texttt{"{}}; \\
\mbox{}\ \ \ \ cout\ $<$$<$\ \texttt{"{}}\texttt{\textbackslash{}t}\texttt{"{}}\ $<$$<$\ FLAG$\_$CREATE$\_$STDNT\ $<$$<$\ \texttt{"{}\ [uid]:\ Create\ a\ new\ Student\ User}\texttt{\textbackslash{}n}\texttt{"{}}; \\
\mbox{}\ \ \ \ cout\ $<$$<$\ \texttt{"{}}\texttt{\textbackslash{}t}\texttt{"{}}\ $<$$<$\ FLAG$\_$PRINT$\_$LOG\ $<$$<$\ \texttt{"{}:\ Read\ the\ log\ file}\texttt{\textbackslash{}n}\texttt{"{}}; \\
\mbox{}\} \\
\mbox{} \\
\mbox{}int\ \textbf{main}(int\ argc,\ char**\ argv)\ \{ \\
\mbox{}\ \ \ \ \textbf{try}\ \{ \\
\mbox{}\ \ \ \ \ \ \ \ char\ flag\ =\ FLAG$\_$NO$\_$OPTION; \\
\mbox{} \\
\mbox{}\ \ \ \ \ \ \ \ Logger\ *logger\ =\ \textbf{new}\ \textbf{Logger}(); \\
\mbox{}\ \ \ \ \ \ \ \ User\ *\ user\ =\ User::\textbf{login}(logger); \\
\mbox{}\ \ \ \ \ \ \ \ logger-$>$\textbf{set$\_$user}(user); \\
\mbox{}\ \ \ \ \ \ \ \ Setup\ *\ setup\ =\ \textbf{new}\ \textbf{Setup}(logger); \\
\mbox{} \\
\mbox{}\ \ \ \ \ \ \ \ \textbf{if}\ (argc\ $<$\ 2)\ \{ \\
\mbox{}\ \ \ \ \ \ \ \ \ \ \ \ \textit{//Allow\ the\ user\ to\ input\ their\ own\ choice\ rather\ than\ forcing\ them\ to\ use\ command\ line\ arguments.} \\
\mbox{}\ \ \ \ \ \ \ \ \ \ \ \ \textbf{do}\ \{ \\
\mbox{}\ \ \ \ \ \ \ \ \ \ \ \ \ \ \ \ cout\ $<$$<$\ \texttt{"{}Enter\ option\ code\ (h\ for\ help):\ "{}}; \\
\mbox{}\ \ \ \ \ \ \ \ \ \ \ \ \ \ \ \ cin\ $>$$>$\ flag; \\
\mbox{}\ \ \ \ \ \ \ \ \ \ \ \ \ \ \ \ \textbf{switch}\ (flag)\ \{ \\
\mbox{}\ \ \ \ \ \ \ \ \ \ \ \ \ \ \ \ \ \ \ \ \textbf{case}\ FLAG$\_$CREATE$\_$ASSIGN: \\
\mbox{}\ \ \ \ \ \ \ \ \ \ \ \ \ \ \ \ \ \ \ \ \ \ \ \ \textbf{break}; \\
\mbox{}\ \ \ \ \ \ \ \ \ \ \ \ \ \ \ \ \ \ \ \ \textbf{case}\ FLAG$\_$CREATE$\_$STAFF: \\
\mbox{}\ \ \ \ \ \ \ \ \ \ \ \ \ \ \ \ \ \ \ \ \ \ \ \ \textbf{break}; \\
\mbox{}\ \ \ \ \ \ \ \ \ \ \ \ \ \ \ \ \ \ \ \ \textbf{case}\ FLAG$\_$CREATE$\_$STDNT: \\
\mbox{}\ \ \ \ \ \ \ \ \ \ \ \ \ \ \ \ \ \ \ \ \ \ \ \ \textbf{break}; \\
\mbox{}\ \ \ \ \ \ \ \ \ \ \ \ \ \ \ \ \ \ \ \ \textbf{case}\ FLAG$\_$PRINT$\_$LOG: \\
\mbox{}\ \ \ \ \ \ \ \ \ \ \ \ \ \ \ \ \ \ \ \ \ \ \ \ \textbf{break}; \\
\mbox{}\ \ \ \ \ \ \ \ \ \ \ \ \ \ \ \ \ \ \ \ \textbf{case}\ FLAG$\_$HELP: \\
\mbox{}\ \ \ \ \ \ \ \ \ \ \ \ \ \ \ \ \ \ \ \ \ \ \ \ \textbf{print$\_$help}(); \\
\mbox{}\ \ \ \ \ \ \ \ \ \ \ \ \ \ \ \ \ \ \ \ \ \ \ \ \textit{//Intended\ fallthrough\ so\ that\ the\ program\ doesn't\ continue\ and\ exit.} \\
\mbox{}\ \ \ \ \ \ \ \ \ \ \ \ \ \ \ \ \ \ \ \ \textbf{default}: \\
\mbox{}\ \ \ \ \ \ \ \ \ \ \ \ \ \ \ \ \ \ \ \ \ \ \ \ flag\ =\ FLAG$\_$NO$\_$OPTION; \\
\mbox{}\ \ \ \ \ \ \ \ \ \ \ \ \ \ \ \ \} \\
\mbox{}\ \ \ \ \ \ \ \ \ \ \ \ \}\ \textbf{while}\ (flag\ ==\ FLAG$\_$NO$\_$OPTION); \\
\mbox{}\ \ \ \ \ \ \ \ \}\ \textbf{else}\ \{ \\
\mbox{}\ \ \ \ \ \ \ \ \ \ \ \ flag\ =\ argv[1][1]; \\
\mbox{}\ \ \ \ \ \ \ \ \} \\
\mbox{} \\
\mbox{}\ \ \ \ \ \ \ \ std::string\ option; \\
\mbox{}\ \ \ \ \ \ \ \ \textbf{switch}\ (flag)\ \{ \\
\mbox{}\ \ \ \ \ \ \ \ \ \ \ \ \textbf{case}\ FLAG$\_$CREATE$\_$ASSIGN: \\
\mbox{}\ \ \ \ \ \ \ \ \ \ \ \ \ \ \ \ \textbf{if}\ (argc\ $<$\ 3)\ \{ \\
\mbox{}\ \ \ \ \ \ \ \ \ \ \ \ \ \ \ \ \ \ \ \ cout\ $<$$<$\ \texttt{"{}Enter\ the\ file\ (including\ path)\ of\ the\ assignment:\ "{}}; \\
\mbox{}\ \ \ \ \ \ \ \ \ \ \ \ \ \ \ \ \ \ \ \ cin\ $>$$>$\ option; \\
\mbox{}\ \ \ \ \ \ \ \ \ \ \ \ \ \ \ \ \}\ \textbf{else}\ \{ \\
\mbox{}\ \ \ \ \ \ \ \ \ \ \ \ \ \ \ \ \ \ \ \ option\ =\ argv[2]; \\
\mbox{}\ \ \ \ \ \ \ \ \ \ \ \ \ \ \ \ \} \\
\mbox{}\ \ \ \ \ \ \ \ \ \ \ \ \ \ \ \ setup-$>$\textbf{upload$\_$assignment}(option); \\
\mbox{}\ \ \ \ \ \ \ \ \ \ \ \ \ \ \ \ \textbf{break}; \\
\mbox{}\ \ \ \ \ \ \ \ \ \ \ \ \textbf{case}\ FLAG$\_$CREATE$\_$STAFF: \\
\mbox{}\ \ \ \ \ \ \ \ \ \ \ \ \ \ \ \ \textbf{if}\ (argc\ $<$\ 3)\ \{ \\
\mbox{}\ \ \ \ \ \ \ \ \ \ \ \ \ \ \ \ \ \ \ \ cout\ $<$$<$\ \texttt{"{}Enter\ the\ UID\ for\ the\ new\ staff\ user:\ "{}}; \\
\mbox{}\ \ \ \ \ \ \ \ \ \ \ \ \ \ \ \ \ \ \ \ cin\ $>$$>$\ option; \\
\mbox{}\ \ \ \ \ \ \ \ \ \ \ \ \ \ \ \ \}\ \textbf{else}\ \{ \\
\mbox{}\ \ \ \ \ \ \ \ \ \ \ \ \ \ \ \ \ \ \ \ option\ =\ argv[2]; \\
\mbox{}\ \ \ \ \ \ \ \ \ \ \ \ \ \ \ \ \} \\
\mbox{}\ \ \ \ \ \ \ \ \ \ \ \ \ \ \ \ user-$>$\textbf{add$\_$staff}(option); \\
\mbox{}\ \ \ \ \ \ \ \ \ \ \ \ \ \ \ \ \textbf{break}; \\
\mbox{}\ \ \ \ \ \ \ \ \ \ \ \ \textbf{case}\ FLAG$\_$CREATE$\_$STDNT: \\
\mbox{}\ \ \ \ \ \ \ \ \ \ \ \ \ \ \ \ \textbf{if}\ (argc\ $<$\ 3)\ \{ \\
\mbox{}\ \ \ \ \ \ \ \ \ \ \ \ \ \ \ \ \ \ \ \ cout\ $<$$<$\ \texttt{"{}Enter\ the\ UID\ for\ the\ new\ student\ user:\ "{}}; \\
\mbox{}\ \ \ \ \ \ \ \ \ \ \ \ \ \ \ \ \ \ \ \ cin\ $>$$>$\ option; \\
\mbox{}\ \ \ \ \ \ \ \ \ \ \ \ \ \ \ \ \}\ \textbf{else}\ \{ \\
\mbox{}\ \ \ \ \ \ \ \ \ \ \ \ \ \ \ \ \ \ \ \ option\ =\ argv[2]; \\
\mbox{}\ \ \ \ \ \ \ \ \ \ \ \ \ \ \ \ \} \\
\mbox{}\ \ \ \ \ \ \ \ \ \ \ \ \ \ \ \ user-$>$\textbf{add$\_$student}(option); \\
\mbox{}\ \ \ \ \ \ \ \ \ \ \ \ \ \ \ \ \textit{//TODO\ add\ something\ to\ create\ dir\ for\ the\ student.} \\
\mbox{} \\
\mbox{}\ \ \ \ \ \ \ \ \ \ \ \ \ \ \ \ \textbf{break}; \\
\mbox{}\ \ \ \ \ \ \ \ \ \ \ \ \textbf{case}\ FLAG$\_$PRINT$\_$LOG: \\
\mbox{}\ \ \ \ \ \ \ \ \ \ \ \ \ \ \ \ logger-$>$\textbf{read$\_$log}(); \\
\mbox{}\ \ \ \ \ \ \ \ \ \ \ \ \ \ \ \ \textbf{break}; \\
\mbox{}\ \ \ \ \ \ \ \ \ \ \ \ \textbf{case}\ FLAG$\_$HELP: \\
\mbox{}\ \ \ \ \ \ \ \ \ \ \ \ \ \ \ \ \textbf{print$\_$help}(); \\
\mbox{}\ \ \ \ \ \ \ \ \ \ \ \ \ \ \ \ \textbf{break}; \\
\mbox{}\ \ \ \ \ \ \ \ \ \ \ \ \textbf{default}: \\
\mbox{}\ \ \ \ \ \ \ \ \ \ \ \ \ \ \ \ \textbf{throw}\ std::\textbf{string}(\texttt{"{}Unrecognised\ Option"{}}); \\
\mbox{}\ \ \ \ \ \ \ \ \} \\
\mbox{} \\
\mbox{}\ \ \ \ \ \ \ \ \textbf{delete}\ logger; \\
\mbox{}\ \ \ \ \ \ \ \ \textbf{delete}\ user; \\
\mbox{}\ \ \ \ \ \ \ \ \textbf{delete}\ setup; \\
\mbox{} \\
\mbox{}\ \ \ \ \ \ \ \ \textbf{return}\ (EXIT$\_$SUCCESS); \\
\mbox{}\ \ \ \ \}\ \textbf{catch}\ (std::string\ e)\ \{ \\
\mbox{}\ \ \ \ \ \ \ \ cout\ $<$$<$\ argv[0]\ $<$$<$\ \texttt{"{}:\ "{}}\ $<$$<$\ e\ $<$$<$\ endl; \\
\mbox{}\ \ \ \ \ \ \ \ cout\ $<$$<$\ \texttt{"{}Useage:\ "{}}\ $<$$<$\ argv[0]\ $<$$<$\ \texttt{"{}\ [OPTION]\ [INPUT]"{}}\ $<$$<$\ endl; \\
\mbox{}\ \ \ \ \ \ \ \ cout\ $<$$<$\ \texttt{"{}Try\ `"{}}\ $<$$<$\ argv[0]\ $<$$<$\ \texttt{"{}\ -h'\ for\ more\ information"{}}\ $<$$<$\ endl; \\
\mbox{}\ \ \ \ \ \ \ \ \textbf{return}\ (EXIT$\_$FAILURE); \\
\mbox{}\ \ \ \ \}\ \textbf{catch}\ (char\ \textbf{const}*\ c)\ \{ \\
\mbox{}\ \ \ \ \ \ \ \ cout\ $<$$<$\ c\ $<$$<$\ endl; \\
\mbox{}\ \ \ \ \} \\
\mbox{}\} \\

\clearpage
\normalsize
\rmfamily
\subsection{Setup.cpp}
\scriptsize
\sffamily
% Generator: GNU source-highlight, by Lorenzo Bettini, http://www.gnu.org/software/src-highlite
\noindent
\mbox{}\textit{/*\ Setup.cpp\ \ Copyright\ (c)\ Alexander\ Brown,\ March\ 2011\ */} \\
\mbox{} \\
\mbox{}\textbf{\#include}\ \texttt{$<$string$>$} \\
\mbox{}\textbf{\#include}\ \texttt{$<$iostream$>$} \\
\mbox{}\textbf{\#include}\ \texttt{$<$fstream$>$} \\
\mbox{} \\
\mbox{}\textbf{\#include}\ \texttt{"{}Setup.h"{}} \\
\mbox{}\textbf{\#include}\ \texttt{"{}FileHandller.h"{}} \\
\mbox{}\textbf{\#include}\ \texttt{"{}Logger.h"{}} \\
\mbox{} \\
\mbox{}\textbf{using}\ \textbf{namespace}\ std; \\
\mbox{} \\
\mbox{}std::string\ \textbf{get$\_$base$\_$dir}()\ \{ \\
\mbox{}\ \ \ \ string\ line\ =\ \texttt{"{}"{}}; \\
\mbox{}\ \ \ \ ifstream\ \textbf{config$\_$file}(\texttt{"{}config"{}}); \\
\mbox{}\ \ \ \ \textbf{if}(config$\_$file.\textbf{is$\_$open}())\ \{ \\
\mbox{}\ \ \ \ \ \ \ \ \textbf{getline}(config$\_$file,\ line); \\
\mbox{}\ \ \ \ \} \\
\mbox{}\ \ \ \ line\ =\ line\ +\ \texttt{"{}/"{}}; \\
\mbox{}\ \ \ \ \textbf{return}\ line; \\
\mbox{}\} \\
\mbox{} \\
\mbox{}std::string\ \textbf{get$\_$repo$\_$dir}()\ \{ \\
\mbox{}\ \ \ \ string\ line\ =\ \texttt{"{}"{}}; \\
\mbox{}\ \ \ \ ifstream\ \textbf{config$\_$file}(\texttt{"{}config"{}}); \\
\mbox{}\ \ \ \ \textbf{if}(config$\_$file.\textbf{is$\_$open}())\ \{ \\
\mbox{}\ \ \ \ \ \ \ \ \textbf{getline}(config$\_$file,\ line); \\
\mbox{}\ \ \ \ \ \ \ \ \textbf{getline}(config$\_$file,\ line); \\
\mbox{}\ \ \ \ \} \\
\mbox{}\ \ \ \ line\ =\ line\ +\ \texttt{"{}/"{}}; \\
\mbox{}\ \ \ \ \textbf{return}\ line; \\
\mbox{}\} \\
\mbox{} \\
\mbox{} \\
\mbox{}\textit{//\ Change\ this\ to\ change\ the\ base\ directory.} \\
\mbox{}\textbf{const}\ std::string\ Setup::BASE$\_$DIR\ =\ \textbf{get$\_$base$\_$dir}(); \\
\mbox{}\textbf{const}\ std::string\ Setup::REPO$\_$DIR\ =\ \textbf{get$\_$repo$\_$dir}(); \\
\mbox{} \\
\mbox{}\textit{//\ Constant\ definitions.} \\
\mbox{}\textbf{const}\ std::string\ Setup::AUTH$\_$FILE\ =\ BASE$\_$DIR+\texttt{"{}.auth"{}}; \\
\mbox{}\textbf{const}\ std::string\ Setup::LOG$\_$FILE\ =\ BASE$\_$DIR+\texttt{"{}logfile.log"{}}; \\
\mbox{}\textbf{const}\ std::string\ Setup::ASSIGNMENT$\_$DIR\ =\ BASE$\_$DIR+\texttt{"{}assignment/"{}}; \\
\mbox{}\textbf{const}\ std::string\ Setup::STUDENT$\_$DIR\ =\ BASE$\_$DIR+\texttt{"{}students/"{}}; \\
\mbox{} \\
\mbox{}Setup::\textbf{Setup}(Logger*\ logger):\ \textbf{$\_$logger}(logger)\ \{ \\
\mbox{} \\
\mbox{}\} \\
\mbox{} \\
\mbox{}Setup::\textasciitilde{}\textbf{Setup}()\ \{ \\
\mbox{}\ \ \ \ $\_$logger\ =\ NULL; \\
\mbox{}\} \\
\mbox{} \\
\mbox{}void\ Setup::\textbf{upload$\_$assignment}(std::string\ src)\ \textbf{throw}\ (std::string)\ \{ \\
\mbox{}\ \ \ \ int\ pos\ =\ src.\textbf{find$\_$last$\_$of}(\texttt{'/'}); \\
\mbox{}\ \ \ \ std::string\ filename\ =\ src.\textbf{substr}(pos+1,\ src.\textbf{length}()); \\
\mbox{} \\
\mbox{}\ \ \ \ FileHandller::\textbf{upload$\_$file}(src,\ ASSIGNMENT$\_$DIR+filename); \\
\mbox{}\ \ \ \  \\
\mbox{}\ \ \ \ $\_$logger-$>$\textbf{log}(\texttt{"{}Uploaded\ Assignment"{}},\ ASSIGNMENT$\_$DIR+filename); \\
\mbox{}\} \\

\clearpage
\normalsize
\rmfamily
\subsection{Setup.h}
\scriptsize
\sffamily
% Generator: GNU source-highlight, by Lorenzo Bettini, http://www.gnu.org/software/src-highlite
\noindent
\mbox{}\textit{/*\ Setup.h\ \ Copyright\ (c)\ Alexander\ Brown,\ March\ 2011\ */} \\
\mbox{} \\
\mbox{}\textbf{\#ifndef}\ $\_$SETUP$\_$H \\
\mbox{}\textbf{\#define}\ \ \ \ \ \ \ \ $\_$SETUP$\_$H \\
\mbox{} \\
\mbox{}\textbf{\#include}\ \texttt{$<$string$>$} \\
\mbox{} \\
\mbox{}\textbf{class}\ User; \\
\mbox{}\textbf{class}\ Logger; \\
\mbox{} \\
\mbox{} \\
\mbox{}\textit{/**} \\
\mbox{}\textit{\ *\ }@brief\textit{\ A\ class\ to\ hadnle\ the\ setup\ of\ the\ repository.} \\
\mbox{}\textit{\ *} \\
\mbox{}\textit{\ *\ This\ class\ was\ orignally\ meant\ to\ handle\ the\ setup\ of\ the} \\
\mbox{}\textit{\ *\ retpository.\ However,\ due\ to\ a\ lack\ of\ time,\ I\ put\ the} \\
\mbox{}\textit{\ *\ majority\ of\ the\ code\ for\ this\ in\ the\ makefile\ as\ BASH} \\
\mbox{}\textit{\ *\ commands.} \\
\mbox{}\textit{\ *} \\
\mbox{}\textit{\ *\ Instead,\ this\ contains\ constants\ to\ locations\ in\ the} \\
\mbox{}\textit{\ *\ repository\ and\ allows\ for\ uploading\ of\ an\ assignment} \\
\mbox{}\textit{\ *\ file.} \\
\mbox{}\textit{\ *} \\
\mbox{}\textit{\ *\ }@author\textit{\ Alexander\ Brown} \\
\mbox{}\textit{\ *\ }@version\textit{\ 1.0} \\
\mbox{}\textit{\ */} \\
\mbox{}\textbf{class}\ Setup\ \{ \\
\mbox{}\textbf{public}: \\
\mbox{}\ \ \ \ \textit{/**\ The\ Base\ Directory\ of\ the\ repository.\ */} \\
\mbox{}\ \ \ \ \textbf{static}\ \textbf{const}\ std::string\ BASE$\_$DIR; \\
\mbox{} \\
\mbox{}\ \ \ \ \textit{/**\ The\ Repository\ directory\ */} \\
\mbox{}\ \ \ \ \textbf{static}\ \textbf{const}\ std::string\ REPO$\_$DIR; \\
\mbox{} \\
\mbox{}\ \ \ \ \textit{/**\ The\ location\ of\ the\ authentication\ file\ */} \\
\mbox{}\ \ \ \ \textbf{static}\ \textbf{const}\ std::string\ AUTH$\_$FILE; \\
\mbox{} \\
\mbox{}\ \ \ \ \textit{/**\ The\ location\ of\ the\ log\ file\ */} \\
\mbox{}\ \ \ \ \textbf{static}\ \textbf{const}\ std::string\ LOG$\_$FILE; \\
\mbox{} \\
\mbox{}\ \ \ \ \textit{/**\ The\ location\ of\ the\ assignment\ directory\ */} \\
\mbox{}\ \ \ \ \textbf{static}\ \textbf{const}\ std::string\ ASSIGNMENT$\_$DIR; \\
\mbox{} \\
\mbox{}\ \ \ \ \textit{/**\ The\ location\ of\ the\ assignment\ directory\ */} \\
\mbox{}\ \ \ \ \textbf{static}\ \textbf{const}\ std::string\ STUDENT$\_$DIR; \\
\mbox{} \\
\mbox{}\ \ \ \ \textit{/**} \\
\mbox{}\textit{\ \ \ \ \ *\ Aggregate\ the\ User\ and\ the\ Logger.} \\
\mbox{}\textit{\ \ \ \ \ *\ }@param\textit{\ user\ \ \ \ \ \ The\ User\ to\ aggregate\ to.} \\
\mbox{}\textit{\ \ \ \ \ *\ }@param\textit{\ logger\ \ \ \ The\ Logger\ to\ aggregate\ to.} \\
\mbox{}\textit{\ \ \ \ \ *\ }@deprecated\textit{\ \ Unneeded.} \\
\mbox{}\textit{\ \ \ \ \ */} \\
\mbox{}\ \ \ \ \textbf{Setup}(User\ *user,\ Logger\ *logger); \\
\mbox{} \\
\mbox{}\ \ \ \ \textit{/**} \\
\mbox{}\textit{\ \ \ \ \ *\ Aggregate\ the\ User\ and\ the\ Logger.} \\
\mbox{}\textit{\ \ \ \ \ *\ }@param\textit{\ user\ \ \ \ \ \ The\ User\ to\ aggregate\ to.} \\
\mbox{}\textit{\ \ \ \ \ *\ }@param\textit{\ logger\ \ \ \ The\ Logger\ to\ aggregate\ to.} \\
\mbox{}\textit{\ \ \ \ \ */} \\
\mbox{}\ \ \ \ \textbf{Setup}(Logger\ *logger); \\
\mbox{} \\
\mbox{}\ \ \ \ \textit{/**\ Removes\ the\ aggregation\ */} \\
\mbox{}\ \ \ \ \textbf{virtual}\ \textasciitilde{}\textbf{Setup}(void); \\
\mbox{} \\
\mbox{}\ \ \ \ \textit{/**\ Sets\ up\ the\ entire\ repository.\ */} \\
\mbox{}\ \ \ \ void\ \textbf{run$\_$setup}(void); \\
\mbox{} \\
\mbox{}\ \ \ \ \textit{/**} \\
\mbox{}\textit{\ \ \ \ \ *\ Allows\ an\ assignment\ to\ be\ uploaded.} \\
\mbox{}\textit{\ \ \ \ \ *\ }@param\textit{\ src\ \ \ The\ location\ of\ the\ local\ assignment\ file\ to\ upload.} \\
\mbox{}\textit{\ \ \ \ \ */} \\
\mbox{}\ \ \ \ void\ \textbf{upload$\_$assignment}(std::string\ src)\ \textbf{throw}\ (std::string); \\
\mbox{} \\
\mbox{}\textbf{private}: \\
\mbox{}\ \ \ \ \textit{/**} \\
\mbox{}\textit{\ \ \ \ \ *\ Aggregation\ to\ the\ user} \\
\mbox{}\textit{\ \ \ \ \ *\ }@deprecated\textit{\ \ Unneeded.} \\
\mbox{}\textit{\ \ \ \ \ */} \\
\mbox{}\ \ \ \ User*\ $\_$user; \\
\mbox{} \\
\mbox{}\ \ \ \ \textit{/**\ Aggregation\ to\ the\ logger\ */} \\
\mbox{}\ \ \ \ Logger*\ $\_$logger; \\
\mbox{} \\
\mbox{}\ \ \ \ void\ \textbf{create$\_$dir$\_$structure}(void); \\
\mbox{}\}; \\
\mbox{} \\
\mbox{}\textbf{\#endif}\ \ \ \ \ \ \ \ \textit{/*\ $\_$SETUP$\_$H\ */} \\
\mbox{} \\

\clearpage
\normalsize
\rmfamily
\subsection{User.cpp}
\scriptsize
\sffamily
% Generator: GNU source-highlight, by Lorenzo Bettini, http://www.gnu.org/software/src-highlite
\noindent
\mbox{}\textit{/**\ User.cpp\ \ Copyright\ (c)\ Alexander\ Brown,\ March\ 2011\ */} \\
\mbox{} \\
\mbox{}\textbf{\#include}\ \texttt{$<$iostream$>$} \\
\mbox{}\textbf{\#include}\ \texttt{$<$fstream$>$} \\
\mbox{}\textbf{\#include}\ \texttt{$<$string$>$} \\
\mbox{}\textbf{\#include}\ \texttt{$<$cstdio$>$} \\
\mbox{} \\
\mbox{}\textbf{\#include}\ \texttt{"{}Logger.h"{}} \\
\mbox{} \\
\mbox{}\textbf{\#include}\ \texttt{$<$termios.h$>$} \\
\mbox{}\textbf{\#include}\ \texttt{$<$unistd.h$>$} \\
\mbox{} \\
\mbox{}\textbf{\#include}\ \texttt{"{}User.h"{}} \\
\mbox{}\textbf{\#include}\ \texttt{"{}UserCreator.h"{}} \\
\mbox{}\textbf{\#include}\ \texttt{"{}Setup.h"{}} \\
\mbox{}\textbf{\#include}\ \texttt{"{}UserType.h"{}} \\
\mbox{} \\
\mbox{}\textbf{using}\ \textbf{namespace}\ std; \\
\mbox{} \\
\mbox{}\textit{/**} \\
\mbox{}\textit{\ *\ Sets\ whether\ the\ stdin\ stream\ should\ echo\ to} \\
\mbox{}\textit{\ *\ the\ console\ or\ not.} \\
\mbox{}\textit{\ *} \\
\mbox{}\textit{\ *\ }@param\textit{\ enable\ If\ stdin\ echo\ is\ enabled\ or\ not.} \\
\mbox{}\textit{\ *\ \ \ \ \ \ \ \ \ \ \ \ \ \ \ \ defaults\ to\ true.} \\
\mbox{}\textit{\ *\ Note:\ Linux\ systems\ only.} \\
\mbox{}\textit{\ *\ }@author\textit{\ Unknown\ (taken\ from\ Stack\ Overflow).} \\
\mbox{}\textit{\ */} \\
\mbox{}void\ \textbf{SetStdinEcho}(bool\ enable\ =\ \textbf{true})\ \{ \\
\mbox{}\ \ \ \ \textbf{struct}\ termios\ tty; \\
\mbox{}\ \ \ \ \textbf{tcgetattr}(STDIN$\_$FILENO,\ \&tty); \\
\mbox{}\ \ \ \ \textbf{if}\ (!enable) \\
\mbox{}\ \ \ \ \ \ \ \ tty.c$\_$lflag\ \&=\ \textasciitilde{}ECHO; \\
\mbox{}\ \ \ \ \textbf{else} \\
\mbox{}\ \ \ \ \ \ \ \ tty.c$\_$lflag\ $|$=\ ECHO; \\
\mbox{} \\
\mbox{}\ \ \ \ \textbf{tcsetattr}(STDIN$\_$FILENO,\ TCSANOW,\ \&tty); \\
\mbox{}\} \\
\mbox{} \\
\mbox{}User\ *User::$\_$INSTANCE\ =\ NULL; \\
\mbox{} \\
\mbox{}User::\textbf{User}(Logger\ *logger,\ std::string\ uid):\ \textbf{$\_$logger}(logger),\ \textbf{$\_$uid}(uid)\ \{ \\
\mbox{}\} \\
\mbox{} \\
\mbox{}User::\textasciitilde{}\textbf{User}()\ \{ \\
\mbox{}\ \ \ \ $\_$logger\ =\ NULL; \\
\mbox{}\ \ \ \ $\_$INSTANCE\ =\ NULL; \\
\mbox{}\} \\
\mbox{} \\
\mbox{}User*\ User::\textbf{login}(Logger\ *logger)\ \textbf{throw}\ (std::string)\ \{ \\
\mbox{}\ \ \ \ \textit{//\ TODO\ decide\ if\ it\ would\ be\ best\ to\ break\ this\ function\ down\ further.} \\
\mbox{}\ \ \ \ \textbf{if}\ ($\_$INSTANCE\ ==\ NULL)\ \{ \\
\mbox{}\ \ \ \ \ \ \ \ std::string\ user; \\
\mbox{}\ \ \ \ \ \ \ \ std::string\ pass; \\
\mbox{} \\
\mbox{}\ \ \ \ \ \ \ \ cout\ $<$$<$\ \texttt{"{}Enter\ User\ ID:\ "{}}; \\
\mbox{}\ \ \ \ \ \ \ \ cin\ $>$$>$\ user; \\
\mbox{} \\
\mbox{}\ \ \ \ \ \ \ \ cout\ $<$$<$\ \texttt{"{}"{}}\ $<$$<$\ user\ $<$$<$\ \texttt{"{}'s\ password:\ "{}}; \\
\mbox{} \\
\mbox{}\ \ \ \ \ \ \ \ \textbf{SetStdinEcho}(\textbf{false}); \\
\mbox{}\ \ \ \ \ \ \ \ cin\ $>$$>$\ pass; \\
\mbox{}\ \ \ \ \ \ \ \ \textbf{SetStdinEcho}(\textbf{true}); \\
\mbox{}\ \ \ \ \ \ \ \ cout\ $<$$<$\ endl; \\
\mbox{}\ \ \ \ \ \ \ \  \\
\mbox{}\ \ \ \ \ \ \ \ ifstream\ \textbf{authstream}(Setup::AUTH$\_$FILE.\textbf{data}()); \\
\mbox{} \\
\mbox{}\ \ \ \ \ \ \ \ \textbf{if}\ (!authstream.\textbf{is$\_$open}())\ \{ \\
\mbox{}\ \ \ \ \ \ \ \ \ \ \ \ std::string\ t\ =\ \texttt{"{}Error\ opening\ '"{}}\ +\ Setup::AUTH$\_$FILE\ +\ \texttt{"{}'"{}}; \\
\mbox{}\ \ \ \ \ \ \ \ \ \ \ \ \textbf{throw}\ t; \\
\mbox{}\ \ \ \ \ \ \ \ \} \\
\mbox{} \\
\mbox{}\ \ \ \ \ \ \ \ bool\ authenticated\ =\ \textbf{false}; \\
\mbox{}\ \ \ \ \ \ \ \ string\ line; \\
\mbox{} \\
\mbox{} \\
\mbox{}\ \ \ \ \ \ \ \ \textbf{while}\ (authstream.\textbf{good}())\ \{ \\
\mbox{}\ \ \ \ \ \ \ \ \ \ \ \ \textbf{getline}(authstream,\ line); \\
\mbox{}\ \ \ \ \ \ \ \ \ \ \ \ \textbf{if}\ (line.\textbf{data}()[0]\ !=\ \texttt{'\#'})\ \{ \\
\mbox{}\ \ \ \ \ \ \ \ \ \ \ \ \ \ \ \ \textbf{if}\ (line.\textbf{find}((user\ +\ \texttt{"{}\ admin\ "{}}\ +\ pass).\textbf{data}())\ !=\ string::npos)\ \{\ \textit{//\ TODO\ See\ if\ this\ can\ be\ made\ more\ secure.} \\
\mbox{}\ \ \ \ \ \ \ \ \ \ \ \ \ \ \ \ \ \ \ \ authenticated\ =\ \textbf{true}; \\
\mbox{}\ \ \ \ \ \ \ \ \ \ \ \ \ \ \ \ \} \\
\mbox{}\ \ \ \ \ \ \ \ \ \ \ \ \} \\
\mbox{}\ \ \ \ \ \ \ \ \} \\
\mbox{} \\
\mbox{}\ \ \ \ \ \ \ \ \textbf{if}\ (!authenticated)\ \{ \\
\mbox{}\ \ \ \ \ \ \ \ \ \ \ \ std::string\ t\ =\ \texttt{"{}Invalid\ Username/Password"{}}; \\
\mbox{}\ \ \ \ \ \ \ \ \ \ \ \ \textbf{throw}\ t; \\
\mbox{}\ \ \ \ \ \ \ \ \} \\
\mbox{} \\
\mbox{}\ \ \ \ \ \ \ \ $\_$INSTANCE\ =\ \textbf{new}\ \textbf{User}(logger,\ user); \\
\mbox{} \\
\mbox{}\ \ \ \ \ \ \ \ authstream.\textbf{close}(); \\
\mbox{}\ \ \ \ \} \\
\mbox{}\ \ \ \ \textbf{return}\ $\_$INSTANCE; \\
\mbox{}\} \\
\mbox{} \\
\mbox{}std::string\ User::\textbf{get$\_$uid}()\ \{ \\
\mbox{}\ \ \ \ \textbf{return}\ $\_$uid; \\
\mbox{}\} \\
\mbox{} \\
\mbox{}std::string\ User::\textbf{add$\_$user}(std::string\ uid,\ UserType\ type)\ \{ \\
\mbox{}\ \ \ \ UserCreator\ \textbf{creator}(uid,\ type); \\
\mbox{}\ \ \ \ \textbf{return}\ creator.\textbf{create$\_$user}(); \\
\mbox{}\} \\
\mbox{} \\
\mbox{}std::string\ User::\textbf{add$\_$staff}(std::string\ uid)\ \{ \\
\mbox{}\ \ \ \ \textbf{enum}\ UserType\ type\ =\ STAFF; \\
\mbox{}\ \ \ \ std::string\ pass\ =\ \textbf{add$\_$user}(uid,\ type); \\
\mbox{}\ \ \ \ $\_$logger-$>$\textbf{log}(\texttt{"{}Created\ Staff\ User"{}},\ Setup::AUTH$\_$FILE); \\
\mbox{}\ \ \ \ cout\ $<$$<$\ \texttt{"{}Created\ Staff\ User:\ "{}}\ $<$$<$\ uid\ $<$$<$\ endl\ $<$$<$\ \texttt{"{}password:\ "{}}\ $<$$<$\ pass\ $<$$<$\ endl; \\
\mbox{}\ \ \ \ \textbf{return}\ pass; \\
\mbox{}\} \\
\mbox{} \\
\mbox{}std::string\ User::\textbf{add$\_$student}(std::string\ uid)\ \{ \\
\mbox{}\ \ \ \ \textbf{enum}\ UserType\ type\ =\ STUDENT; \\
\mbox{}\ \ \ \ std::string\ pass\ =\ \textbf{add$\_$user}(uid,\ type); \\
\mbox{}\ \ \ \ $\_$logger-$>$\textbf{log}(\texttt{"{}Created\ Student\ User"{}},\ Setup::AUTH$\_$FILE); \\
\mbox{}\ \ \ \ cout\ $<$$<$\ \texttt{"{}Created\ Student\ User:\ "{}}\ $<$$<$\ uid\ $<$$<$\ endl\ $<$$<$\ \texttt{"{}password:\ "{}}\ $<$$<$\ pass\ $<$$<$\ endl; \\
\mbox{}\ \ \ \ \textbf{return}\ pass; \\
\mbox{}\} \\

\clearpage
\normalsize
\rmfamily
\subsection{UserCreator.cpp}
\scriptsize
\sffamily
% Generator: GNU source-highlight, by Lorenzo Bettini, http://www.gnu.org/software/src-highlite
\noindent
\mbox{}\textit{/*\ UserCreator.cpp\ \ Copyright\ (c)\ Alexander\ Brown,\ March\ 2011\ */} \\
\mbox{} \\
\mbox{}\textbf{\#include}\ \texttt{$<$stdlib.h$>$} \\
\mbox{}\textbf{\#include}\ \texttt{$<$time.h$>$} \\
\mbox{}\textbf{\#include}\ \texttt{$<$sys/stat.h$>$} \\
\mbox{}\textbf{\#include}\ \texttt{$<$sys/types.h$>$} \\
\mbox{} \\
\mbox{}\textbf{\#include}\ \texttt{"{}UserCreator.h"{}} \\
\mbox{}\textbf{\#include}\ \texttt{"{}Setup.h"{}} \\
\mbox{}\textbf{\#include}\ \texttt{"{}FileHandller.h"{}} \\
\mbox{} \\
\mbox{}UserCreator::\textbf{UserCreator}(std::string\ uid,\ \textbf{enum}\ UserType\ type)\ :\ \textbf{$\_$uid}(uid),\ \textbf{$\_$type}(type)\ \{ \\
\mbox{}\ \ \ \ $\_$file$\_$handller\ =\ \textbf{new}\ \textbf{FileHandller}(Setup::AUTH$\_$FILE); \\
\mbox{}\} \\
\mbox{} \\
\mbox{}UserCreator::\textasciitilde{}\textbf{UserCreator}()\ \{ \\
\mbox{}\ \ \ \ \textbf{delete}\ $\_$file$\_$handller; \\
\mbox{}\} \\
\mbox{} \\
\mbox{}std::string\ UserCreator::\textbf{create$\_$user}()\ \{ \\
\mbox{}\ \ \ \ std::string\ passwd\ =\ \textbf{generate$\_$password}(); \\
\mbox{}\ \ \ \ std::string\ entry\ =\ $\_$uid\ +\ \texttt{"{}\ "{}}; \\
\mbox{}\ \ \ \ \textbf{switch}\ ($\_$type)\ \{ \\
\mbox{}\ \ \ \ \ \ \ \ \textbf{case}\ STAFF: \\
\mbox{}\ \ \ \ \ \ \ \ \ \ \ \ entry\ +=\ \texttt{"{}staff\ "{}}; \\
\mbox{}\ \ \ \ \ \ \ \ \ \ \ \ \textbf{break}; \\
\mbox{}\ \ \ \ \ \ \ \ \textbf{case}\ STUDENT: \\
\mbox{}\ \ \ \ \ \ \ \ \ \ \ \ entry\ +=\ \texttt{"{}stdnt\ "{}}; \\
\mbox{}\ \ \ \ \ \ \ \ \ \ \ \ \textbf{mkdir}((Setup::STUDENT$\_$DIR+$\_$uid+\texttt{"{}/"{}}).\textbf{data}(),\ 0744); \\
\mbox{}\ \ \ \ \ \ \ \ \ \ \ \ \textbf{mkdir}((Setup::STUDENT$\_$DIR+$\_$uid+\texttt{"{}/results/"{}}).\textbf{data}(),\ 0744); \\
\mbox{}\ \ \ \ \ \ \ \ \ \ \ \ \textbf{break}; \\
\mbox{}\ \ \ \ \ \ \ \ \textbf{default}: \\
\mbox{}\ \ \ \ \ \ \ \ \ \ \ \ entry\ =\ \texttt{"{}\#"{}}\ +\ entry;\ \textit{//Something\ went\ wrong\ so\ comment\ it\ out.} \\
\mbox{}\ \ \ \ \} \\
\mbox{}\ \ \ \ entry\ +=\ passwd; \\
\mbox{}\ \ \ \ $\_$file$\_$handller-$>$\textbf{append$\_$file}(entry); \\
\mbox{}\ \ \ \ \textbf{return}\ passwd; \\
\mbox{} \\
\mbox{}\} \\
\mbox{} \\
\mbox{}std::string\ UserCreator::\textbf{generate$\_$password}()\ \{ \\
\mbox{}\ \ \ \ \textbf{srand}(\textbf{time}(NULL)); \\
\mbox{}\ \ \ \ int\ passwd$\_$length\ =\ 8\ +\ (\textbf{rand}()\ \%\ 5); \\
\mbox{}\ \ \ \ std::string\ passwd\ =\ \texttt{"{}"{}}; \\
\mbox{} \\
\mbox{}\ \ \ \ \textbf{for}\ (int\ i\ =\ 0;\ i\ $<$\ passwd$\_$length;\ i++)\ \{ \\
\mbox{}\ \ \ \ \ \ \ \ \textit{/*\ Modulo\ of\ the\ last\ character\ of\ the\ characters\ we\ want\ (z,\ as} \\
\mbox{}\textit{\ \ \ \ \ \ \ \ \ *\ lowercase\ is\ after\ uppercase\ and\ numbers)\ minus\ the\ first\ character,} \\
\mbox{}\textit{\ \ \ \ \ \ \ \ \ *\ plus\ one\ to\ make\ it\ encompass\ them\ all.\ Plus\ the\ starting\ character).} \\
\mbox{}\textit{\ \ \ \ \ \ \ \ \ */} \\
\mbox{}\ \ \ \ \ \ \ \ char\ c\ =\ \textbf{rand}()\ \%\ (\texttt{'z'}\ -\ \texttt{'0'}\ +\ 1)\ +\ \texttt{'0'}; \\
\mbox{}\ \ \ \ \ \ \ \ passwd\ +=\ c; \\
\mbox{}\ \ \ \ \} \\
\mbox{} \\
\mbox{}\ \ \ \ \textbf{return}\ passwd; \\
\mbox{}\} \\

\clearpage
\normalsize
\rmfamily
\subsection{UserCreator.h}
\scriptsize
\sffamily
% Generator: GNU source-highlight, by Lorenzo Bettini, http://www.gnu.org/software/src-highlite
\noindent
\mbox{}\textit{/*\ UserCreator.h\ Copyright\ (c)\ Alexander\ Brown,\ March\ 2011\ */} \\
\mbox{} \\
\mbox{}\textbf{\#ifndef}\ $\_$USERCREATOR$\_$H \\
\mbox{}\textbf{\#define}\ $\_$USERCREATOR$\_$H \\
\mbox{} \\
\mbox{}\textbf{\#include}\ \texttt{$<$string$>$} \\
\mbox{} \\
\mbox{}\textbf{\#include}\ \texttt{"{}UserType.h"{}} \\
\mbox{} \\
\mbox{}\textbf{class}\ User; \\
\mbox{}\textbf{class}\ Logger; \\
\mbox{}\textbf{class}\ FileHandller; \\
\mbox{} \\
\mbox{}\textit{/**} \\
\mbox{}\textit{\ *\ }@brief\textit{\ Handles\ the\ creation\ of\ users.} \\
\mbox{}\textit{\ *\ } \\
\mbox{}\textit{\ *\ This\ classes\ defines\ the\ behaviour\ to\ use\ when\ creating} \\
\mbox{}\textit{\ *\ new\ users,\ ensuring\ this\ behaviour\ is\ standardised.} \\
\mbox{}\textit{\ *} \\
\mbox{}\textit{\ *\ }@author\textit{\ Alexander\ Brown} \\
\mbox{}\textit{\ *\ }@version\textit{\ 1.0} \\
\mbox{}\textit{\ */} \\
\mbox{}\textbf{class}\ UserCreator\ \{ \\
\mbox{}\textbf{public}: \\
\mbox{}\ \ \ \ \ \ \ \ \textit{/**\ Constructs\ a\ UserCreator\ with\ the\ given\ uid\ and\ type.\ */} \\
\mbox{}\ \ \ \ \ \ \ \ \textbf{UserCreator}(std::string\ uid,\ \textbf{enum}\ UserType\ type); \\
\mbox{} \\
\mbox{}\ \ \ \ \ \ \ \ \textit{/**\ Destructs\ the\ UserCreator.\ */} \\
\mbox{}\ \ \ \ \ \ \ \ \textasciitilde{}\textbf{UserCreator}(); \\
\mbox{} \\
\mbox{}\ \ \ \ \ \ \ \ \textit{/**\ Writes\ the\ user\ to\ the\ .auth\ file\ */} \\
\mbox{}\ \ \ \ \ \ \ \ std::string\ \textbf{create$\_$user}(); \\
\mbox{}\textbf{private}: \\
\mbox{}\ \ \ \ \ \ \ \ \textit{/**\ The\ file\ handller\ */} \\
\mbox{}\ \ \ \ \ \ \ \ FileHandller\ *\ $\_$file$\_$handller; \\
\mbox{} \\
\mbox{}\ \ \ \ \ \ \ \ \textit{/**\ The\ User\ ID\ of\ the\ user\ to\ create\ */} \\
\mbox{}\ \ \ \ \ \ \ \ std::string\ $\_$uid; \\
\mbox{} \\
\mbox{}\ \ \ \ \ \ \ \ \textit{/**\ The\ generated\ password\ for\ the\ User\ */} \\
\mbox{}\ \ \ \ \ \ \ \ std::string\ $\_$password; \\
\mbox{} \\
\mbox{}\ \ \ \ \ \ \ \ \textit{/**\ The\ type\ of\ the\ User\ */} \\
\mbox{}\ \ \ \ \ \ \ \ \textbf{enum}\ UserType\ $\_$type; \\
\mbox{} \\
\mbox{}\ \ \ \ \ \ \ \ \textit{/**\ Generates\ a\ random\ password\ */} \\
\mbox{}\ \ \ \ \ \ \ \ std::string\ \textbf{generate$\_$password}(); \\
\mbox{}\}; \\
\mbox{} \\
\mbox{}\textbf{\#endif}\ \ \textit{/*\ $\_$USERCREATOR$\_$H\ */} \\

\clearpage
\normalsize
\rmfamily
\subsection{User.h}
\scriptsize
\sffamily
% Generator: GNU source-highlight, by Lorenzo Bettini, http://www.gnu.org/software/src-highlite
\noindent
\mbox{}\textit{/*\ User.h\ \ Copyright\ (c)\ Alexander\ Brown,\ March\ 2011\ */} \\
\mbox{} \\
\mbox{}\textbf{\#ifndef}\ $\_$USER$\_$H \\
\mbox{}\textbf{\#define}\ $\_$USER$\_$H \\
\mbox{} \\
\mbox{}\textbf{\#include}\ \texttt{$<$string$>$} \\
\mbox{}\textbf{\#include}\ \texttt{"{}UserType.h"{}} \\
\mbox{} \\
\mbox{}\textit{//\ Define\ the\ Logger\ class\ to\ stop\ strange\ dependancy\ errors.} \\
\mbox{}\textbf{class}\ Logger; \\
\mbox{} \\
\mbox{}\textit{/**} \\
\mbox{}\textit{\ *\ }@brief\textit{\ Defines\ the\ User\ for\ the\ Administator\ Programs\ and\ associated\ methods.} \\
\mbox{}\textit{\ *} \\
\mbox{}\textit{\ *\ This\ class\ is\ mostly\ used\ to\ control\ the\ logging\ in\ and\ out\ of\ the\ User,\ as} \\
\mbox{}\textit{\ *\ well\ as\ a\ few\ base\ actions\ (adding\ Users)\ so\ most\ items\ can\ be\ set\ up\ without} \\
\mbox{}\textit{\ *\ too\ much\ hastle.} \\
\mbox{}\textit{\ *} \\
\mbox{}\textit{\ *\ This\ class\ is\ a\ singleton,\ hence\ the\ private\ constructor,\ and\ can\ only\ be} \\
\mbox{}\textit{\ *\ instantiated\ through\ the\ login\ method.} \\
\mbox{}\textit{\ */} \\
\mbox{}\textbf{class}\ User\ \{ \\
\mbox{}\textbf{public}: \\
\mbox{}\ \ \ \ \textit{/**} \\
\mbox{}\textit{\ \ \ \ \ *\ Logs\ in\ a\ user.} \\
\mbox{}\textit{\ \ \ \ \ *} \\
\mbox{}\textit{\ \ \ \ \ *\ }@param\textit{\ logger\ \ \ \ The\ aggregation\ to\ the\ logger.} \\
\mbox{}\textit{\ \ \ \ \ *} \\
\mbox{}\textit{\ \ \ \ \ *\ }@return\textit{\ The\ single\ instance\ of\ the\ User.\ Note\ that\ if\ the\ User\ is\ already} \\
\mbox{}\textit{\ \ \ \ \ *\ logged\ in\ then\ this\ will\ still\ return\ the\ instance,\ just\ without\ any} \\
\mbox{}\textit{\ \ \ \ \ *\ ability\ to\ re-login.} \\
\mbox{}\textit{\ \ \ \ \ */} \\
\mbox{}\ \ \ \ \textbf{static}\ User*\ \textbf{login}(Logger\ *logger)\ \textbf{throw}\ (std::string); \\
\mbox{} \\
\mbox{}\ \ \ \ \textit{/**} \\
\mbox{}\textit{\ \ \ \ \ *\ Destructor.\ As\ ISTANCE\ is\ the\ only\ pointer\ that\ needs\ to\ be\ freed\ in\ this} \\
\mbox{}\textit{\ \ \ \ \ *\ class\ and\ is\ the\ one\ this\ destructor\ will\ be\ called\ upon,\ this\ doesn't} \\
\mbox{}\textit{\ \ \ \ \ *\ acutally\ do\ much,\ other\ than\ setting\ a\ few\ nulls.} \\
\mbox{}\textit{\ \ \ \ \ */} \\
\mbox{}\ \ \ \ \textasciitilde{}\textbf{User}(void); \\
\mbox{} \\
\mbox{}\ \ \ \ \textit{/**} \\
\mbox{}\textit{\ \ \ \ \ *\ Adds\ a\ member\ of\ staff\ to\ the\ program.} \\
\mbox{}\textit{\ \ \ \ \ *} \\
\mbox{}\textit{\ \ \ \ \ *\ }@param\textit{\ uid\ \ \ The\ User\ ID\ of\ the\ Staff\ to\ add.} \\
\mbox{}\textit{\ \ \ \ \ *} \\
\mbox{}\textit{\ \ \ \ \ *\ }@param\textit{\ The\ auto-generated\ password.} \\
\mbox{}\textit{\ \ \ \ \ *} \\
\mbox{}\textit{\ \ \ \ \ *\ }@see\textit{\ User::addUser(std::string,\ enum\ UserType)} \\
\mbox{}\textit{\ \ \ \ \ */} \\
\mbox{}\ \ \ \ std::string\ \textbf{add$\_$staff}(std::string\ uid); \\
\mbox{} \\
\mbox{} \\
\mbox{}\ \ \ \ \textit{/**} \\
\mbox{}\textit{\ \ \ \ \ *\ Adds\ a\ student\ to\ the\ program.} \\
\mbox{}\textit{\ \ \ \ \ *} \\
\mbox{}\textit{\ \ \ \ \ *\ }@param\textit{\ uid\ \ \ The\ User\ ID\ of\ the\ Student\ to\ add.} \\
\mbox{}\textit{\ \ \ \ \ *} \\
\mbox{}\textit{\ \ \ \ \ *\ }@param\textit{\ The\ auto-generated\ password.} \\
\mbox{}\textit{\ \ \ \ \ *} \\
\mbox{}\textit{\ \ \ \ \ *\ }@see\textit{\ User::addUser(std::string,\ enum\ UserType)} \\
\mbox{}\textit{\ \ \ \ \ */} \\
\mbox{}\ \ \ \ std::string\ \textbf{add$\_$student}(std::string\ uid); \\
\mbox{} \\
\mbox{}\ \ \ \ \textit{/**} \\
\mbox{}\textit{\ \ \ \ \ *\ Returns\ the\ User\ ID\ of\ the\ current\ User.} \\
\mbox{}\textit{\ \ \ \ \ *} \\
\mbox{}\textit{\ \ \ \ \ *\ }@return\textit{\ The\ User\ ID\ of\ the\ current\ User.} \\
\mbox{}\textit{\ \ \ \ \ */} \\
\mbox{}\ \ \ \ std::string\ \textbf{get$\_$uid}(void); \\
\mbox{} \\
\mbox{}\textbf{private}: \\
\mbox{}\ \ \ \ \textit{/**\ The\ single\ instance\ of\ this\ class\ */} \\
\mbox{}\ \ \ \ \textbf{static}\ User\ *\ $\_$INSTANCE; \\
\mbox{} \\
\mbox{}\ \ \ \ \textit{/**\ Aggregation\ to\ the\ logger.\ */} \\
\mbox{}\ \ \ \ Logger\ *\ $\_$logger; \\
\mbox{} \\
\mbox{}\ \ \ \ \textit{/**\ The\ User\ ID\ of\ the\ User,\ used\ for\ logging\ purposes.\ */} \\
\mbox{}\ \ \ \ std::string\ $\_$uid; \\
\mbox{} \\
\mbox{}\ \ \ \ \textit{/**} \\
\mbox{}\textit{\ \ \ \ \ *\ Private\ Constructor\ to\ ensure\ this\ remains\ a\ singleton.} \\
\mbox{}\textit{\ \ \ \ \ *} \\
\mbox{}\textit{\ \ \ \ \ *\ }@param\textit{\ logger\ \ \ \ The\ aggregation\ to\ the\ Logger.} \\
\mbox{}\textit{\ \ \ \ \ *\ }@param\textit{\ uid\ \ \ \ \ \ \ The\ User\ ID\ of\ this\ User.} \\
\mbox{}\textit{\ \ \ \ \ */} \\
\mbox{}\ \ \ \ \textbf{User}(Logger\ *logger,\ std::string\ uid); \\
\mbox{} \\
\mbox{}\ \ \ \ \textit{/**} \\
\mbox{}\textit{\ \ \ \ \ *\ Adds\ a\ user\ to\ the\ program,\ returning\ the\ auto-generated\ password.} \\
\mbox{}\textit{\ \ \ \ \ *} \\
\mbox{}\textit{\ \ \ \ \ *\ }@param\textit{\ uid\ \ \ The\ User\ ID\ of\ the\ user\ to\ create.} \\
\mbox{}\textit{\ \ \ \ \ *\ }@param\textit{\ type\ \ The\ type\ of\ the\ User.} \\
\mbox{}\textit{\ \ \ \ \ *} \\
\mbox{}\textit{\ \ \ \ \ *\ }@param\textit{\ The\ auto-generated\ password.} \\
\mbox{}\textit{\ \ \ \ \ */} \\
\mbox{}\ \ \ \ std::string\ \textbf{add$\_$user}(std::string\ uid,\ \textbf{enum}\ UserType\ type); \\
\mbox{}\}; \\
\mbox{} \\
\mbox{}\textbf{\#endif}\ \ \textit{/*\ $\_$USER$\_$H\ */} \\

\clearpage
\normalsize
\rmfamily
\subsection{UserType.h}
\scriptsize
\sffamily
% Generator: GNU source-highlight, by Lorenzo Bettini, http://www.gnu.org/software/src-highlite
\noindent
\mbox{}\textit{/*\ UserType.h\ \ Copyright\ (c)\ Alexander\ Brown,\ March\ 2011\ */} \\
\mbox{} \\
\mbox{}\textbf{\#ifndef}\ $\_$USERTYPE$\_$H \\
\mbox{}\textbf{\#define}\ $\_$USERTYPE$\_$H \\
\mbox{} \\
\mbox{}\textbf{enum}\ UserType\ \{ \\
\mbox{}\ \ \ \ STAFF\ =\ 0, \\
\mbox{}\ \ \ \ STUDENT\ =\ 1 \\
\mbox{}\}; \\
\mbox{} \\
\mbox{}\textbf{\#endif}\ \ \textit{/*\ $\_$USERTYPE$\_$H\ */} \\

\clearpage

		\section{Source code - Staff Program}
\normalsize
\rmfamily
\subsection{InvalidUserException.java}
\scriptsize
\sffamily
% Generator: GNU source-highlight, by Lorenzo Bettini, http://www.gnu.org/software/src-highlite
\noindent
\mbox{}\textbf{package}\ uk.ac.aber.users.adb9.cs22510.staff$\_$program; \\
\mbox{} \\
\mbox{}\textit{/**} \\
\mbox{}\textit{\ *\ An\ Exception\ for\ throwing\ when\ a\ Username\ or\ Password\ is\ incorrect.} \\
\mbox{}\textit{\ *\ }\textbf{$<$p$>$} \\
\mbox{}\textit{\ *\ Thrown\ from\ \{}@link\textit{\ User\#login(Logger)\}\ so\ that\ the\ program\ doesn't\ have} \\
\mbox{}\textit{\ *\ to\ compare\ ugly\ things\ like\ \{}@code\textit{\ null\}s\ to\ check\ if\ a\ User\ logged\ in.} \\
\mbox{}\textit{\ *\ } \\
\mbox{}\textit{\ *\ }@author\textit{\ Alexander\ Brown} \\
\mbox{}\textit{\ *\ }@version\textit{\ 1.0} \\
\mbox{}\textit{\ */} \\
\mbox{}\textbf{public}\ \textbf{class}\ InvalidUserException\ \textbf{extends}\ Exception\ \{ \\
\mbox{}\ \ \ \ \ \ \ \ \textit{/**\ The\ Serial\ Version\ UID\ for\ serialisation\ potential.\ */} \\
\mbox{}\ \ \ \ \ \ \ \ \textbf{private}\ \textbf{static}\ \textbf{final}\ long\ serialVersionUID\ =\ -239788711036612392L; \\
\mbox{}\ \ \ \ \ \ \ \  \\
\mbox{}\ \ \ \ \ \ \ \ \textit{/**\ The\ default\ error\ message.\ */} \\
\mbox{}\ \ \ \ \ \ \ \ \textbf{private}\ \textbf{static}\ \textbf{final}\ String\ DEFAULT$\_$MESSAGE\ =\ \texttt{"{}Invalid\ User/Password"{}}; \\
\mbox{}\ \ \ \ \ \ \ \  \\
\mbox{}\ \ \ \ \ \ \ \ \textit{/**\ Displays\ the\ default\ error\ message\ */} \\
\mbox{}\ \ \ \ \ \ \ \ \textbf{public}\ \textbf{InvalidUserException}()\ \{ \\
\mbox{}\ \ \ \ \ \ \ \ \ \ \ \ \ \ \ \ \textbf{super}(DEFAULT$\_$MESSAGE); \\
\mbox{}\ \ \ \ \ \ \ \ \} \\
\mbox{}\} \\

\clearpage
\normalsize
\rmfamily
\subsection{Logger.java}
\scriptsize
\sffamily
% Generator: GNU source-highlight, by Lorenzo Bettini, http://www.gnu.org/software/src-highlite
\noindent
\mbox{}\textbf{package}\ uk.ac.aber.users.adb9.cs22510.staff$\_$program; \\
\mbox{} \\
\mbox{}\textbf{import}\ java.text.SimpleDateFormat; \\
\mbox{}\textbf{import}\ java.util.Date; \\
\mbox{} \\
\mbox{}\textbf{import}\ uk.ac.aber.users.adb9.cs22510.staff$\_$program.io.FileHandller; \\
\mbox{} \\
\mbox{}\textit{/**} \\
\mbox{}\textit{\ *\ Defines\ the\ logging\ class.} \\
\mbox{}\textit{\ *\ }\textbf{$<$p$>$} \\
\mbox{}\textit{\ *\ This\ class\ handles\ all\ entries\ that\ have\ to\ be\ made\ to\ the} \\
\mbox{}\textit{\ *\ \{}@linkplain\textit{\ StaffProgram\#LOG$\_$FILE\ logfile\},\ controlling\ how\ the\ entries} \\
\mbox{}\textit{\ *\ are\ made\ and\ the\ format\ in\ which\ they\ are\ made.} \\
\mbox{}\textit{\ *} \\
\mbox{}\textit{\ *\ }@author\textit{\ \ \ \ \ \ \ \ Alexander\ Brown} \\
\mbox{}\textit{\ *\ }@version\textit{\ \ \ \ \ \ \ \ 1.0} \\
\mbox{}\textit{\ *} \\
\mbox{}\textit{\ *\ }@see\textit{\ FileHandller} \\
\mbox{}\textit{\ */} \\
\mbox{}\textbf{public}\ \textbf{class}\ Logger\ \{ \\
\mbox{}\ \ \ \ \ \ \ \ \textit{/**\ Link\ back\ to\ the\ User\ */} \\
\mbox{}\ \ \ \ \ \ \ \ \textbf{private}\ User\ user; \\
\mbox{}\ \ \ \ \ \ \ \  \\
\mbox{}\ \ \ \ \ \ \ \ \textit{/**} \\
\mbox{}\textit{\ \ \ \ \ \ \ \ \ *\ Sets\ the\ link\ to\ the\ User.} \\
\mbox{}\textit{\ \ \ \ \ \ \ \ \ *\ }@param\textit{\ user\ \ \ \ \ \ \ \ The\ User\ to\ set\ the\ link\ back\ to.} \\
\mbox{}\textit{\ \ \ \ \ \ \ \ \ */} \\
\mbox{}\ \ \ \ \ \ \ \ \textbf{public}\ void\ \textbf{setUser}(User\ user)\ \{ \\
\mbox{}\ \ \ \ \ \ \ \ \ \ \ \ \ \ \ \ \textbf{this}.user\ =\ user; \\
\mbox{}\ \ \ \ \ \ \ \ \} \\
\mbox{}\ \ \ \ \ \ \ \  \\
\mbox{}\ \ \ \ \ \ \ \ \textit{/**} \\
\mbox{}\textit{\ \ \ \ \ \ \ \ \ *\ Makes\ an\ entry\ to\ the\ logfile.} \\
\mbox{}\textit{\ \ \ \ \ \ \ \ \ *\ } \\
\mbox{}\textit{\ \ \ \ \ \ \ \ \ *\ }@param\textit{\ activity\ \ \ \ \ \ \ \ The\ activity\ to\ be\ logged.} \\
\mbox{}\textit{\ \ \ \ \ \ \ \ \ *\ }@param\textit{\ filepath\ \ \ \ \ \ \ \ The\ path\ to\ the\ file\ that\ the\ activity\ was} \\
\mbox{}\textit{\ \ \ \ \ \ \ \ \ *\ performed\ on.} \\
\mbox{}\textit{\ \ \ \ \ \ \ \ \ *} \\
\mbox{}\textit{\ \ \ \ \ \ \ \ \ *\ }@see\textit{\ generateLogEntry()} \\
\mbox{}\textit{\ \ \ \ \ \ \ \ \ */} \\
\mbox{}\ \ \ \ \ \ \ \ \textbf{public}\ void\ \textbf{log}(String\ activity,\ String\ filepath)\ \{ \\
\mbox{}\ \ \ \ \ \ \ \ \ \ \ \ \ \ \ \ FileHandller\ handller\ =\ \textbf{new}\ \textbf{FileHandller}(StaffProgram.LOG$\_$FILE); \\
\mbox{}\ \ \ \ \ \ \ \ \ \ \ \ \ \ \ \ handller.\textbf{appendFile}(\textbf{generateLogEntry}(activity,\ filepath)); \\
\mbox{}\ \ \ \ \ \ \ \ \ \ \ \ \ \ \ \ handller.\textbf{close}(); \\
\mbox{}\ \ \ \ \ \ \ \ \} \\
\mbox{} \\
\mbox{}\ \ \ \ \ \ \ \ \textit{/**} \\
\mbox{}\textit{\ \ \ \ \ \ \ \ \ *\ Generates\ the\ text\ for\ a\ log\ entry} \\
\mbox{}\textit{\ \ \ \ \ \ \ \ \ *} \\
\mbox{}\textit{\ \ \ \ \ \ \ \ \ *\ }@param\textit{\ activity\ \ \ \ \ \ \ \ The\ activity\ to\ be\ logged.} \\
\mbox{}\textit{\ \ \ \ \ \ \ \ \ *\ }@param\textit{\ filepath\ \ \ \ \ \ \ \ The\ path\ to\ the\ file\ that\ the\ activity\ was} \\
\mbox{}\textit{\ \ \ \ \ \ \ \ \ *\ performed\ on.} \\
\mbox{}\textit{\ \ \ \ \ \ \ \ \ *} \\
\mbox{}\textit{\ \ \ \ \ \ \ \ \ *\ }@see\textit{\ encodeDateTime()} \\
\mbox{}\textit{\ \ \ \ \ \ \ \ \ */} \\
\mbox{}\ \ \ \ \ \ \ \ \textbf{private}\ String\ \textbf{generateLogEntry}(String\ activity,\ String\ filepath)\ \{ \\
\mbox{}\ \ \ \ \ \ \ \ \ \ \ \ \ \ \ \ String\ file\ =\ filepath.\textbf{replace}(StaffProgram.BASE$\_$DIR,\ \texttt{"{}"{}}); \\
\mbox{} \\
\mbox{}\ \ \ \ \ \ \ \ \ \ \ \ \ \ \ \ String[]\ path\ =\ StaffProgram.BASE$\_$DIR.\textbf{split}(\texttt{"{}/"{}}); \\
\mbox{}\ \ \ \ \ \ \ \ \ \ \ \ \ \ \ \ String\ base\ =\ path[path.length-1]; \\
\mbox{} \\
\mbox{}\ \ \ \ \ \ \ \ \ \ \ \ \ \ \ \ String\ entry\ =\ String.\textbf{format}(\texttt{"{}\%s\ staff:\ \%s}\texttt{\textbackslash{}t}\texttt{\%s}\texttt{\textbackslash{}t}\texttt{\%s/\%s}\texttt{\textbackslash{}n}\texttt{"{}},\  \\
\mbox{}\ \ \ \ \ \ \ \ \ \ \ \ \ \ \ \ \ \ \ \ \ \ \ \ \textbf{encodeDateTime}(),\  \\
\mbox{}\ \ \ \ \ \ \ \ \ \ \ \ \ \ \ \ \ \ \ \ \ \ \ \ user.\textbf{getUID}(),\  \\
\mbox{}\ \ \ \ \ \ \ \ \ \ \ \ \ \ \ \ \ \ \ \ \ \ \ \ activity,\  \\
\mbox{}\ \ \ \ \ \ \ \ \ \ \ \ \ \ \ \ \ \ \ \ \ \ \ \ base, \\
\mbox{}\ \ \ \ \ \ \ \ \ \ \ \ \ \ \ \ \ \ \ \ \ \ \ \ file); \\
\mbox{}\ \ \ \ \ \ \ \ \ \ \ \ \ \ \ \ \textbf{return}\ entry; \\
\mbox{}\ \ \ \ \ \ \ \ \} \\
\mbox{}\ \ \ \ \ \ \ \  \\
\mbox{}\ \ \ \ \ \ \ \ \textit{/**} \\
\mbox{}\textit{\ \ \ \ \ \ \ \ \ *\ Takes\ the\ date\ and\ time\ and\ encodes\ them\ into\ the\ string\ format} \\
\mbox{}\textit{\ \ \ \ \ \ \ \ \ *\ used\ by\ the\ log.} \\
\mbox{}\textit{\ \ \ \ \ \ \ \ \ *} \\
\mbox{}\textit{\ \ \ \ \ \ \ \ \ *\ }@return\textit{\ An\ encoded\ date\ time.} \\
\mbox{}\textit{\ \ \ \ \ \ \ \ \ */} \\
\mbox{}\ \ \ \ \ \ \ \ \textbf{private}\ String\ \textbf{encodeDateTime}()\ \{ \\
\mbox{}\ \ \ \ \ \ \ \ \ \ \ \ \ \ \ \ Date\ d\ =\ \textbf{new}\ \textbf{Date}(); \\
\mbox{}\ \ \ \ \ \ \ \ \ \ \ \ \ \ \ \ SimpleDateFormat\ df\ =\ \textbf{new}\ \textbf{SimpleDateFormat}(\texttt{"{}yyyy-MM-dd-kk-mm-ss"{}}); \\
\mbox{}\ \ \ \ \ \ \ \ \ \ \ \ \ \ \ \ \textbf{return}\ df.\textbf{format}(d); \\
\mbox{}\ \ \ \ \ \ \ \ \} \\
\mbox{}\} \\

\clearpage
\normalsize
\rmfamily
\subsection{StaffProgram.java}
\scriptsize
\sffamily
% Generator: GNU source-highlight, by Lorenzo Bettini, http://www.gnu.org/software/src-highlite
\noindent
\mbox{}\textbf{package}\ uk.ac.aber.users.adb9.cs22510.staff$\_$program; \\
\mbox{} \\
\mbox{}\textbf{import}\ java.util.Scanner; \\
\mbox{}\textbf{import}\ java.io.File; \\
\mbox{}\textbf{import}\ java.io.FileInputStream; \\
\mbox{}\textbf{import}\ java.io.IOException; \\
\mbox{}\textbf{import}\ java.io.InputStream; \\
\mbox{}\textbf{import}\ java.io.InputStreamReader; \\
\mbox{} \\
\mbox{}\textbf{public}\ \textbf{class}\ StaffProgram\ \{ \\
\mbox{}\ \ \ \ \ \ \ \ \textbf{private}\ \textbf{static}\ \textbf{final}\ File\ CONFIG$\_$FILE\ =\ \textbf{new}\ \textbf{File}(\texttt{"{}config"{}}); \\
\mbox{} \\
\mbox{}\ \ \ \ \ \ \ \ \textbf{public}\ \textbf{static}\ \textbf{final}\ String\ BASE$\_$DIR\ =\ \textbf{getBaseDir}(); \\
\mbox{}\ \ \ \ \ \ \ \ \textbf{public}\ \textbf{static}\ \textbf{final}\ String\ LOG$\_$FILE\ =\ BASE$\_$DIR\ +\ \texttt{"{}logfile.log"{}}; \\
\mbox{}\ \ \ \ \ \ \ \ \textbf{public}\ \textbf{static}\ \textbf{final}\ String\ AUTH$\_$FILE\ =\ BASE$\_$DIR\ +\ \texttt{"{}.auth"{}}; \\
\mbox{}\ \ \ \ \ \ \ \ \textbf{public}\ \textbf{static}\ \textbf{final}\ String\ STUDENT$\_$DIR\ =\ BASE$\_$DIR\ +\ \texttt{"{}students/"{}}; \\
\mbox{} \\
\mbox{}\ \ \ \ \ \ \ \ \textit{/**} \\
\mbox{}\textit{\ \ \ \ \ \ \ \ \ *\ Defins\ the\ list\ option} \\
\mbox{}\textit{\ \ \ \ \ \ \ \ \ *\ }@see\textit{\ User\#list()} \\
\mbox{}\textit{\ \ \ \ \ \ \ \ \ */} \\
\mbox{}\ \ \ \ \ \ \ \ \textbf{private}\ \textbf{static}\ \textbf{final}\ char\ LIST\ =\ \texttt{'l'}; \\
\mbox{}\ \ \ \ \ \ \ \  \\
\mbox{}\ \ \ \ \ \ \ \ \textit{/**} \\
\mbox{}\textit{\ \ \ \ \ \ \ \ \ *\ Defines\ the\ get\ option} \\
\mbox{}\textit{\ \ \ \ \ \ \ \ \ *\ }@see\textit{\ User\#get(String\ uid,\ String\ destPath);} \\
\mbox{}\textit{\ \ \ \ \ \ \ \ \ */} \\
\mbox{}\ \ \ \ \ \ \ \ \textbf{private}\ \textbf{static}\ \textbf{final}\ char\ GET\ \ =\ \texttt{'g'}; \\
\mbox{}\ \ \ \ \ \ \ \  \\
\mbox{}\ \ \ \ \ \ \ \ \textit{/**} \\
\mbox{}\textit{\ \ \ \ \ \ \ \ \ *\ Defines\ the\ mark\ option} \\
\mbox{}\textit{\ \ \ \ \ \ \ \ \ *\ }@see\textit{\ User\#mark(String\ uid);} \\
\mbox{}\textit{\ \ \ \ \ \ \ \ \ */} \\
\mbox{}\ \ \ \ \ \ \ \ \textbf{private}\ \textbf{static}\ \textbf{final}\ char\ MARK\ =\ \texttt{'m'}; \\
\mbox{}\ \ \ \ \ \ \ \  \\
\mbox{}\ \ \ \ \ \ \ \ \textit{/**\ Defines\ the\ quit\ option\ */} \\
\mbox{}\ \ \ \ \ \ \ \ \textbf{private}\ \textbf{static}\ \textbf{final}\ char\ QUIT\ =\ \texttt{'q'}; \\
\mbox{}\ \ \ \ \ \ \ \  \\
\mbox{}\ \ \ \ \ \ \ \ \textit{/**\ } \\
\mbox{}\textit{\ \ \ \ \ \ \ \ \ *\ Defines\ the\ help\ option} \\
\mbox{}\textit{\ \ \ \ \ \ \ \ \ *\ }@see\textit{\ help()} \\
\mbox{}\textit{\ \ \ \ \ \ \ \ \ */} \\
\mbox{}\ \ \ \ \ \ \ \ \textbf{private}\ \textbf{static}\ \textbf{final}\ char\ HELP\ =\ \texttt{'h'}; \\
\mbox{}\ \ \ \ \ \ \ \  \\
\mbox{}\ \ \ \ \ \ \ \ \textit{/**\ Defines\ the\ Unrecognised\ option,\ used\ to\ continue\ looping\ the\ menu.\ */} \\
\mbox{}\ \ \ \ \ \ \ \ \textbf{private}\ \textbf{static}\ \textbf{final}\ char\ NONE\ =\ \texttt{'\ '}; \\
\mbox{} \\
\mbox{}\ \ \ \ \ \ \ \ \textbf{public}\ \textbf{static}\ void\ \textbf{main}(String\ args[])\ \{ \\
\mbox{}\ \ \ \ \ \ \ \ \ \ \ \ \ \ \ \ Logger\ logger\ =\ \textbf{new}\ \textbf{Logger}(); \\
\mbox{}\ \ \ \ \ \ \ \ \ \ \ \ \ \ \ \ User\ user\ =\ \textbf{null}; \\
\mbox{}\ \ \ \ \ \ \ \ \ \ \ \ \ \ \ \ \textbf{try}\ \{ \\
\mbox{}\ \ \ \ \ \ \ \ \ \ \ \ \ \ \ \ \ \ \ \ \ \ \ \ user\ =\ User.\textbf{login}(logger); \\
\mbox{}\ \ \ \ \ \ \ \ \ \ \ \ \ \ \ \ \}\ \textbf{catch}\ (InvalidUserException\ e)\ \{ \\
\mbox{}\ \ \ \ \ \ \ \ \ \ \ \ \ \ \ \ \ \ \ \ \ \ \ \ System.err.\textbf{println}(e.\textbf{getMessage}()); \\
\mbox{}\ \ \ \ \ \ \ \ \ \ \ \ \ \ \ \ \ \ \ \ \ \ \ \ System.\textbf{exit}(1); \\
\mbox{}\ \ \ \ \ \ \ \ \ \ \ \ \ \ \ \ \} \\
\mbox{}\ \ \ \ \ \ \ \ \ \ \ \ \ \ \ \  \\
\mbox{}\ \ \ \ \ \ \ \ \ \ \ \ \ \ \ \ logger.\textbf{setUser}(user); \\
\mbox{} \\
\mbox{}\ \ \ \ \ \ \ \ \ \ \ \ \ \ \ \ char\ option\ =\ NONE; \\
\mbox{}\ \ \ \ \ \ \ \ \ \ \ \ \ \ \ \ Scanner\ in\ =\ \textbf{new}\ \textbf{Scanner}(System.in); \\
\mbox{} \\
\mbox{}\ \ \ \ \ \ \ \ \ \ \ \ \ \ \ \ \textbf{while}\ (option\ ==\ NONE)\ \{ \\
\mbox{}\ \ \ \ \ \ \ \ \ \ \ \ \ \ \ \ \ \ \ \ \ \ \ \ System.out.\textbf{print}(\texttt{"{}Enter\ option\ code\ (h\ for\ help):\ "{}}); \\
\mbox{}\ \ \ \ \ \ \ \ \ \ \ \ \ \ \ \ \ \ \ \ \ \ \ \ option\ =\ in.\textbf{next}().\textbf{charAt}(0); \\
\mbox{}\ \ \ \ \ \ \ \ \ \ \ \ \ \ \ \ \ \ \ \ \ \ \ \ String\ uid\ =\ \texttt{"{}"{}}; \\
\mbox{}\ \ \ \ \ \ \ \ \ \ \ \ \ \ \ \ \ \ \ \ \ \ \ \ \textbf{switch}\ (option)\ \{ \\
\mbox{}\ \ \ \ \ \ \ \ \ \ \ \ \ \ \ \ \ \ \ \ \ \ \ \ \ \ \ \ \ \ \ \ \textbf{case}\ LIST: \\
\mbox{}\ \ \ \ \ \ \ \ \ \ \ \ \ \ \ \ \ \ \ \ \ \ \ \ \ \ \ \ \ \ \ \ \ \ \ \ \ \ \ \ user.\textbf{list}(); \\
\mbox{}\ \ \ \ \ \ \ \ \ \ \ \ \ \ \ \ \ \ \ \ \ \ \ \ \ \ \ \ \ \ \ \ \ \ \ \ \ \ \ \ \textbf{break}; \\
\mbox{}\ \ \ \ \ \ \ \ \ \ \ \ \ \ \ \ \ \ \ \ \ \ \ \ \ \ \ \ \ \ \ \ \textbf{case}\ \ GET: \\
\mbox{}\ \ \ \ \ \ \ \ \ \ \ \ \ \ \ \ \ \ \ \ \ \ \ \ \ \ \ \ \ \ \ \ \ \ \ \ \ \ \ \ System.out.\textbf{print}(\texttt{"{}Enter\ the\ uid\ of\ the\ student\ to\ get\ the\ submission\ of:\ "{}}); \\
\mbox{}\ \ \ \ \ \ \ \ \ \ \ \ \ \ \ \ \ \ \ \ \ \ \ \ \ \ \ \ \ \ \ \ \ \ \ \ \ \ \ \ uid\ =\ in.\textbf{next}(); \\
\mbox{}\ \ \ \ \ \ \ \ \ \ \ \ \ \ \ \ \ \ \ \ \ \ \ \ \ \ \ \ \ \ \ \ \ \ \ \ \ \ \ \ System.out.\textbf{print}(\texttt{"{}Enter\ the\ location\ you\ wish\ to\ download\ this\ submission\ to:\ "{}}); \\
\mbox{}\ \ \ \ \ \ \ \ \ \ \ \ \ \ \ \ \ \ \ \ \ \ \ \ \ \ \ \ \ \ \ \ \ \ \ \ \ \ \ \ String\ file\ =\ in.\textbf{next}(); \\
\mbox{}\ \ \ \ \ \ \ \ \ \ \ \ \ \ \ \ \ \ \ \ \ \ \ \ \ \ \ \ \ \ \ \ \ \ \ \ \ \ \ \ user.\textbf{download}(uid,\ file); \\
\mbox{}\ \ \ \ \ \ \ \ \ \ \ \ \ \ \ \ \ \ \ \ \ \ \ \ \ \ \ \ \ \ \ \ \ \ \ \ \ \ \ \ \textbf{break}; \\
\mbox{}\ \ \ \ \ \ \ \ \ \ \ \ \ \ \ \ \ \ \ \ \ \ \ \ \ \ \ \ \ \ \ \ \textbf{case}\ MARK: \\
\mbox{}\ \ \ \ \ \ \ \ \ \ \ \ \ \ \ \ \ \ \ \ \ \ \ \ \ \ \ \ \ \ \ \ \ \ \ \ \ \ \ \ System.out.\textbf{print}(\texttt{"{}Enter\ the\ uid\ of\ the\ student\ to\ submit\ a\ mark\ for:\ "{}}); \\
\mbox{}\ \ \ \ \ \ \ \ \ \ \ \ \ \ \ \ \ \ \ \ \ \ \ \ \ \ \ \ \ \ \ \ \ \ \ \ \ \ \ \ uid\ =\ in.\textbf{next}(); \\
\mbox{}\ \ \ \ \ \ \ \ \ \ \ \ \ \ \ \ \ \ \ \ \ \ \ \ \ \ \ \ \ \ \ \ \ \ \ \ \ \ \ \ user.\textbf{mark}(uid); \\
\mbox{}\ \ \ \ \ \ \ \ \ \ \ \ \ \ \ \ \ \ \ \ \ \ \ \ \ \ \ \ \ \ \ \ \ \ \ \ \ \ \ \ \textbf{break}; \\
\mbox{}\ \ \ \ \ \ \ \ \ \ \ \ \ \ \ \ \ \ \ \ \ \ \ \ \ \ \ \ \ \ \ \ \textbf{case}\ QUIT: \\
\mbox{}\ \ \ \ \ \ \ \ \ \ \ \ \ \ \ \ \ \ \ \ \ \ \ \ \ \ \ \ \ \ \ \ \ \ \ \ \ \ \ \ \textbf{break}; \\
\mbox{}\ \ \ \ \ \ \ \ \ \ \ \ \ \ \ \ \ \ \ \ \ \ \ \ \ \ \ \ \ \ \ \ \textbf{case}\ HELP: \\
\mbox{}\ \ \ \ \ \ \ \ \ \ \ \ \ \ \ \ \ \ \ \ \ \ \ \ \ \ \ \ \ \ \ \ \ \ \ \ \ \ \ \ StaffProgram.\textbf{help}(); \\
\mbox{}\ \ \ \ \ \ \ \ \ \ \ \ \ \ \ \ \ \ \ \ \ \ \ \ \ \ \ \ \ \ \ \ \ \ \ \ \ \ \ \ option\ =\ NONE; \\
\mbox{}\ \ \ \ \ \ \ \ \ \ \ \ \ \ \ \ \ \ \ \ \ \ \ \ \ \ \ \ \ \ \ \ \ \ \ \ \ \ \ \ \textbf{break}; \\
\mbox{}\ \ \ \ \ \ \ \ \ \ \ \ \ \ \ \ \ \ \ \ \ \ \ \ \ \ \ \ \ \ \ \ \textbf{default}: \\
\mbox{}\ \ \ \ \ \ \ \ \ \ \ \ \ \ \ \ \ \ \ \ \ \ \ \ \ \ \ \ \ \ \ \ \ \ \ \ \ \ \ \ System.out.\textbf{println}(\texttt{"{}Unrecognised\ option\ '"{}}+option+\texttt{"{}'"{}}); \\
\mbox{}\ \ \ \ \ \ \ \ \ \ \ \ \ \ \ \ \ \ \ \ \ \ \ \ \ \ \ \ \ \ \ \ \ \ \ \ \ \ \ \ option\ =\ NONE; \\
\mbox{}\ \ \ \ \ \ \ \ \ \ \ \ \ \ \ \ \ \ \ \ \ \ \ \ \} \\
\mbox{}\ \ \ \ \ \ \ \ \ \ \ \ \ \ \ \ \} \\
\mbox{}\ \ \ \ \ \ \ \ \} \\
\mbox{}\ \ \ \ \ \ \ \  \\
\mbox{}\ \ \ \ \ \ \ \ \textit{/**\ Prints\ out\ the\ help\ options\ */} \\
\mbox{}\ \ \ \ \ \ \ \ \textbf{private}\ \textbf{static}\ void\ \textbf{help}()\ \{ \\
\mbox{}\ \ \ \ \ \ \ \ \ \ \ \ \ \ \ \ System.out.\textbf{printf}(\texttt{"{}}\texttt{\textbackslash{}t}\texttt{\%s\ -\ List\ all\ students\ with\ submitted\ work.}\texttt{\textbackslash{}n}\texttt{"{}} \\
\mbox{}\ \ \ \ \ \ \ \ \ \ \ \ \ \ \ \ +\texttt{"{}}\texttt{\textbackslash{}t}\texttt{\%s\ -\ Download\ a\ specified\ student's\ work.}\texttt{\textbackslash{}n}\texttt{"{}} \\
\mbox{}\ \ \ \ \ \ \ \ \ \ \ \ \ \ \ \ +\texttt{"{}}\texttt{\textbackslash{}t}\texttt{\%s\ -\ Submit\ a\ mark\ for\ a\ specified\ student's\ work.}\texttt{\textbackslash{}n}\texttt{"{}} \\
\mbox{}\ \ \ \ \ \ \ \ \ \ \ \ \ \ \ \ +\texttt{"{}}\texttt{\textbackslash{}t}\texttt{\%s\ -\ Quit\ the\ program.}\texttt{\textbackslash{}n}\texttt{"{}},\  \\
\mbox{}\ \ \ \ \ \ \ \ \ \ \ \ \ \ \ \ LIST,\  \\
\mbox{}\ \ \ \ \ \ \ \ \ \ \ \ \ \ \ \ GET,\  \\
\mbox{}\ \ \ \ \ \ \ \ \ \ \ \ \ \ \ \ MARK,\  \\
\mbox{}\ \ \ \ \ \ \ \ \ \ \ \ \ \ \ \ QUIT); \\
\mbox{}\ \ \ \ \ \ \ \ \} \\
\mbox{} \\
\mbox{}\ \ \ \ \ \ \ \ \textit{/**} \\
\mbox{}\textit{\ \ \ \ \ \ \ \ \ *\ Gets\ the\ Base\ Directory\ from\ the\ config\ file} \\
\mbox{}\textit{\ \ \ \ \ \ \ \ \ *\ }@return\textit{\ The\ path\ of\ the\ Base\ Directory} \\
\mbox{}\textit{\ \ \ \ \ \ \ \ \ */} \\
\mbox{}\ \ \ \ \ \ \ \ \textbf{private}\ \textbf{static}\ String\ \textbf{getBaseDir}()\ \{ \\
\mbox{}\ \ \ \ \ \ \ \ \ \ \ \ \ \ \ \ \textbf{try}\ \{ \\
\mbox{}\ \ \ \ \ \ \ \ \ \ \ \ \ \ \ \ \ \ \ \ \ \ \ \ String\ line\ =\ \texttt{"{}"{}}; \\
\mbox{}\ \ \ \ \ \ \ \ \ \ \ \ \ \ \ \ \ \ \ \ \ \ \ \ \textbf{if}(CONFIG$\_$FILE.\textbf{exists}())\ \{ \\
\mbox{}\ \ \ \ \ \ \ \ \ \ \ \ \ \ \ \ \ \ \ \ \ \ \ \ \ \ \ \ \ \ \ \ Scanner\ s\ =\ \textbf{new}\ \textbf{Scanner}(\textbf{new}\ \textbf{InputStreamReader}( \\
\mbox{}\ \ \ \ \ \ \ \ \ \ \ \ \ \ \ \ \ \ \ \ \ \ \ \ \ \ \ \ \ \ \ \ \ \ \ \ \ \ \ \ \ \ \ \ \ \ \ \ \textbf{new}\ \textbf{FileInputStream}(CONFIG$\_$FILE))); \\
\mbox{}\ \ \ \ \ \ \ \ \ \ \ \ \ \ \ \ \ \ \ \ \ \ \ \ \ \ \ \ \ \ \ \ line\ =\ s.\textbf{nextLine}(); \\
\mbox{}\ \ \ \ \ \ \ \ \ \ \ \ \ \ \ \ \ \ \ \ \ \ \ \ \} \\
\mbox{}\ \ \ \ \ \ \ \  \\
\mbox{}\ \ \ \ \ \ \ \ \ \ \ \ \ \ \ \ \ \ \ \ \ \ \ \ \textbf{return}\ line+\texttt{"{}/"{}}; \\
\mbox{}\ \ \ \ \ \ \ \ \ \ \ \ \ \ \ \ \} \\
\mbox{}\ \ \ \ \ \ \ \ \ \ \ \ \ \ \ \ \textbf{catch}\ (IOException\ ioe)\ \{ \\
\mbox{}\ \ \ \ \ \ \ \ \ \ \ \ \ \ \ \ \ \ \ \ \ \ \ \ ioe.\textbf{printStackTrace}(); \\
\mbox{}\ \ \ \ \ \ \ \ \ \ \ \ \ \ \ \ \ \ \ \ \ \ \ \ \textbf{return}\ \texttt{"{}"{}}; \\
\mbox{}\ \ \ \ \ \ \ \ \ \ \ \ \ \ \ \ \} \\
\mbox{}\ \ \ \ \ \ \ \ \} \\
\mbox{}\} \\

\clearpage
\normalsize
\rmfamily
\subsection{User.java}
\scriptsize
\sffamily
% Generator: GNU source-highlight, by Lorenzo Bettini, http://www.gnu.org/software/src-highlite
\noindent
\mbox{}\textbf{package}\ uk.ac.aber.users.adb9.cs22510.staff$\_$program; \\
\mbox{} \\
\mbox{}\textbf{import}\ java.io.Console; \\
\mbox{}\textbf{import}\ java.io.File; \\
\mbox{}\textbf{import}\ java.io.FileInputStream; \\
\mbox{}\textbf{import}\ java.io.IOException; \\
\mbox{}\textbf{import}\ java.io.InputStream; \\
\mbox{}\textbf{import}\ java.io.InputStreamReader; \\
\mbox{}\textbf{import}\ java.util.Scanner; \\
\mbox{} \\
\mbox{}\textbf{import}\ uk.ac.aber.users.adb9.cs22510.staff$\_$program.io.FileHandller; \\
\mbox{} \\
\mbox{}\textbf{public}\ \textbf{class}\ User\ \{ \\
\mbox{}\ \ \ \ \ \ \ \ \textit{/**\ The\ Singleton\ Instance.\ */} \\
\mbox{}\ \ \ \ \ \ \ \ \textbf{static}\ \textbf{private}\ User\ INSTANCE\ =\ \textbf{null}; \\
\mbox{}\ \ \ \ \ \ \ \  \\
\mbox{}\ \ \ \ \ \ \ \ \textit{/**} \\
\mbox{}\textit{\ \ \ \ \ \ \ \ \ *\ Logs\ in\ a\ User.} \\
\mbox{}\textit{\ \ \ \ \ \ \ \ \ *\ }\textbf{$<$p$>$} \\
\mbox{}\textit{\ \ \ \ \ \ \ \ \ *\ This\ is\ the\ only\ way\ of\ creating\ a\ User,\ and\ ensures\ that\ the\ singleton} \\
\mbox{}\textit{\ \ \ \ \ \ \ \ \ *\ design\ patter\ this\ class\ uses\ is\ kept\ strict.} \\
\mbox{}\textit{\ \ \ \ \ \ \ \ \ *\ }\textbf{$<$p$>$} \\
\mbox{}\textit{\ \ \ \ \ \ \ \ \ *\ As\ expected\ from\ a\ login\ method,\ the\ system\ asks\ for\ a\ User\ ID\ and\ password} \\
\mbox{}\textit{\ \ \ \ \ \ \ \ \ *\ (the\ latter\ of\ which\ is\ concealed\ by\ the\ Console),\ then\ checks\ them\ against} \\
\mbox{}\textit{\ \ \ \ \ \ \ \ \ *\ the\ \{}@linkplain\textit{\ StaffProgram\#AUTH$\_$FILE\ authentication\ file\}\ and\ acts\ } \\
\mbox{}\textit{\ \ \ \ \ \ \ \ \ *\ accordingly.} \\
\mbox{}\textit{\ \ \ \ \ \ \ \ \ *} \\
\mbox{}\textit{\ \ \ \ \ \ \ \ \ *\ }@param\textit{\ logger\ \ \ \ \ \ \ \ The\ link\ back\ to\ the\ Logger.} \\
\mbox{}\textit{\ \ \ \ \ \ \ \ \ *} \\
\mbox{}\textit{\ \ \ \ \ \ \ \ \ *\ }@throws\textit{\ InvalidUserException\ \ \ \ \ \ \ \ If\ the\ User\ ID\ or\ connected\ password\ is\ } \\
\mbox{}\textit{\ \ \ \ \ \ \ \ \ *\ incorrect.} \\
\mbox{}\textit{\ \ \ \ \ \ \ \ \ *} \\
\mbox{}\textit{\ \ \ \ \ \ \ \ \ *\ }@see\textit{\ Console} \\
\mbox{}\textit{\ \ \ \ \ \ \ \ \ */} \\
\mbox{}\ \ \ \ \ \ \ \ \textbf{public}\ \textbf{static}\ User\ \textbf{login}(Logger\ logger)\ \textbf{throws}\ InvalidUserException\ \{ \\
\mbox{}\ \ \ \ \ \ \ \ \ \ \ \ \ \ \ \ Console\ c\ =\ System.\textbf{console}(); \\
\mbox{}\ \ \ \ \ \ \ \ \ \ \ \ \ \ \ \ \textbf{if}(c\ ==\ \textbf{null})\ \{ \\
\mbox{}\ \ \ \ \ \ \ \ \ \ \ \ \ \ \ \ \ \ \ \ \ \ \ \ System.err.\textbf{println}(\texttt{"{}No\ Console."{}}); \\
\mbox{}\ \ \ \ \ \ \ \ \ \ \ \ \ \ \ \ \ \ \ \ \ \ \ \ System.\textbf{exit}(2); \\
\mbox{}\ \ \ \ \ \ \ \ \ \ \ \ \ \ \ \ \} \\
\mbox{}\ \ \ \ \ \ \ \ \ \ \ \ \ \ \ \  \\
\mbox{}\ \ \ \ \ \ \ \ \ \ \ \ \ \ \ \ Scanner\ s\ =\ \textbf{new}\ \textbf{Scanner}(System.in); \\
\mbox{}\ \ \ \ \ \ \ \ \ \ \ \ \ \ \ \ System.out.\textbf{print}(\texttt{"{}Enter\ User\ ID:\ "{}}); \\
\mbox{}\ \ \ \ \ \ \ \ \ \ \ \ \ \ \ \ String\ uid\ =\ s.\textbf{next}(); \\
\mbox{}\ \ \ \ \ \ \ \ \ \ \ \ \ \ \ \  \\
\mbox{}\ \ \ \ \ \ \ \ \ \ \ \ \ \ \ \ System.out.\textbf{printf}(\texttt{"{}Password\ for\ \%s:\ "{}},\ uid); \\
\mbox{}\ \ \ \ \ \ \ \ \ \ \ \ \ \ \ \ char[]\ charPass\ =\ c.\textbf{readPassword}(\texttt{"{}"{}}); \\
\mbox{}\ \ \ \ \ \ \ \ \ \ \ \ \ \ \ \ String\ passwd\ =\ String.\textbf{valueOf}(charPass); \\
\mbox{} \\
\mbox{}\ \ \ \ \ \ \ \ \ \ \ \ \ \ \ \ FileHandller\ f\ =\ \textbf{new}\ \textbf{FileHandller}(StaffProgram.AUTH$\_$FILE); \\
\mbox{}\ \ \ \ \ \ \ \ \ \ \ \ \ \ \ \ File\ auth\ =\ f.\textbf{getFile}(); \\
\mbox{}\ \ \ \ \ \ \ \ \ \ \ \ \ \ \ \ String\ line; \\
\mbox{}\ \ \ \ \ \ \ \ \ \ \ \ \ \ \ \ Scanner\ reader\ =\ \textbf{null}; \\
\mbox{} \\
\mbox{}\ \ \ \ \ \ \ \ \ \ \ \ \ \ \ \ \textbf{try}\ \{ \\
\mbox{}\ \ \ \ \ \ \ \ \ \ \ \ \ \ \ \ \ \ \ \ \ \ \ \ reader\ =\ \textbf{new}\ \textbf{Scanner}(\textbf{new}\ \textbf{InputStreamReader}( \\
\mbox{}\ \ \ \ \ \ \ \ \ \ \ \ \ \ \ \ \ \ \ \ \ \ \ \ \ \ \ \ \ \ \ \ \ \ \ \ \ \ \ \ \textbf{new}\ \textbf{FileInputStream}(auth))); \\
\mbox{}\ \ \ \ \ \ \ \ \ \ \ \ \ \ \ \ \ \ \ \ \ \ \ \ \textbf{while}\ (reader.\textbf{hasNext}())\ \{ \\
\mbox{}\ \ \ \ \ \ \ \ \ \ \ \ \ \ \ \ \ \ \ \ \ \ \ \ \ \ \ \ \ \ \ \ line\ =\ reader.\textbf{nextLine}(); \\
\mbox{}\ \ \ \ \ \ \ \ \ \ \ \ \ \ \ \ \ \ \ \ \ \ \ \ \ \ \ \ \ \ \ \ \textbf{if}\ (line.\textbf{equals}(String.\textbf{format}(\texttt{"{}\%s\ staff\ \%s"{}},\ uid,\ passwd)))\ \{ \\
\mbox{}\ \ \ \ \ \ \ \ \ \ \ \ \ \ \ \ \ \ \ \ \ \ \ \ \ \ \ \ \ \ \ \ \ \ \ \ \ \ \ \ INSTANCE\ =\ \textbf{new}\ \textbf{User}(uid,\ logger); \\
\mbox{}\ \ \ \ \ \ \ \ \ \ \ \ \ \ \ \ \ \ \ \ \ \ \ \ \ \ \ \ \ \ \ \ \} \\
\mbox{}\ \ \ \ \ \ \ \ \ \ \ \ \ \ \ \ \ \ \ \ \ \ \ \ \} \\
\mbox{}\ \ \ \ \ \ \ \ \ \ \ \ \ \ \ \ \ \ \ \ \ \ \ \ reader.\textbf{close}(); \\
\mbox{}\ \ \ \ \ \ \ \ \ \ \ \ \ \ \ \ \}\ \textbf{catch}\ (IOException\ eio)\ \{ \\
\mbox{}\ \ \ \ \ \ \ \ \ \ \ \ \ \ \ \ \ \ \ \ \ \ \ \  \\
\mbox{}\ \ \ \ \ \ \ \ \ \ \ \ \ \ \ \ \} \\
\mbox{}\ \ \ \ \ \ \ \ \ \ \ \ \ \ \ \  \\
\mbox{}\ \ \ \ \ \ \ \ \ \ \ \ \ \ \ \ f.\textbf{close}(); \\
\mbox{} \\
\mbox{}\ \ \ \ \ \ \ \ \ \ \ \ \ \ \ \ \textbf{if}\ (INSTANCE\ ==\ \textbf{null})\ \{ \\
\mbox{}\ \ \ \ \ \ \ \ \ \ \ \ \ \ \ \ \ \ \ \ \ \ \ \ \textbf{throw}\ \textbf{new}\ \textbf{InvalidUserException}(); \\
\mbox{}\ \ \ \ \ \ \ \ \ \ \ \ \ \ \ \ \} \\
\mbox{} \\
\mbox{}\ \ \ \ \ \ \ \ \ \ \ \ \ \ \ \ \textbf{return}\ INSTANCE; \\
\mbox{}\ \ \ \ \ \ \ \ \} \\
\mbox{} \\
\mbox{}\ \ \ \ \ \ \ \ \textit{/**\ The\ User\ ID\ of\ the\ current\ User\ */} \\
\mbox{}\ \ \ \ \ \ \ \ \textbf{private}\ String\ uid; \\
\mbox{} \\
\mbox{}\ \ \ \ \ \ \ \ \textit{/**\ The\ link\ to\ the\ Logger\ */} \\
\mbox{}\ \ \ \ \ \ \ \ \textbf{private}\ Logger\ logger; \\
\mbox{} \\
\mbox{}\ \ \ \ \ \ \ \ \textit{/**\ The\ File\ Handller\ this\ class\ makes\ use\ of\ */} \\
\mbox{}\ \ \ \ \ \ \ \ \textbf{private}\ FileHandller\ fileHandller; \\
\mbox{} \\
\mbox{}\ \ \ \ \ \ \ \ \textit{/**} \\
\mbox{}\textit{\ \ \ \ \ \ \ \ \ *\ Creates\ a\ new\ User,\ with\ the\ given\ User\ ID\ and\ a\ link\ to\ a\ Logger.} \\
\mbox{}\textit{\ \ \ \ \ \ \ \ \ *\ }\textbf{$<$p$>$} \\
\mbox{}\textit{\ \ \ \ \ \ \ \ \ *\ Note\ that\ this\ is\ private\ to\ facilitate\ the\ singleton\ design\ pattern\ this\ class\ uses.} \\
\mbox{}\textit{\ \ \ \ \ \ \ \ \ *} \\
\mbox{}\textit{\ \ \ \ \ \ \ \ \ *\ }@param\textit{\ uid\ \ \ \ \ \ \ \ \ \ \ \ \ \ \ \ \ The\ User\ ID\ of\ this\ User.} \\
\mbox{}\textit{\ \ \ \ \ \ \ \ \ *\ }@param\textit{\ logger\ \ \ \ \ \ \ \ \ The\ link\ to\ the\ Logger.} \\
\mbox{}\textit{\ \ \ \ \ \ \ \ \ */} \\
\mbox{}\ \ \ \ \ \ \ \ \textbf{private}\ \textbf{User}(String\ uid,\ Logger\ logger)\ \{ \\
\mbox{}\ \ \ \ \ \ \ \ \ \ \ \ \ \ \ \ \textbf{this}.uid\ =\ uid; \\
\mbox{}\ \ \ \ \ \ \ \ \ \ \ \ \ \ \ \ \textbf{this}.logger\ =\ logger; \\
\mbox{}\ \ \ \ \ \ \ \ \} \\
\mbox{} \\
\mbox{}\ \ \ \ \ \ \ \ \textit{/**} \\
\mbox{}\textit{\ \ \ \ \ \ \ \ \ *\ Returns\ the\ User\ ID\ of\ the\ current\ User.} \\
\mbox{}\textit{\ \ \ \ \ \ \ \ \ *\ }@return\textit{\ The\ User\ ID\ of\ the\ current\ User.} \\
\mbox{}\textit{\ \ \ \ \ \ \ \ \ */} \\
\mbox{}\ \ \ \ \ \ \ \ \textbf{public}\ String\ \textbf{getUID}()\ \{ \\
\mbox{}\ \ \ \ \ \ \ \ \ \ \ \ \ \ \ \ \textbf{return}\ uid; \\
\mbox{}\ \ \ \ \ \ \ \ \} \\
\mbox{} \\
\mbox{}\ \ \ \ \ \ \ \ \textit{/**} \\
\mbox{}\textit{\ \ \ \ \ \ \ \ \ *\ Lists\ all\ Students\ that\ have\ made\ submissions.} \\
\mbox{}\textit{\ \ \ \ \ \ \ \ \ *\ }\textbf{$<$p$>$} \\
\mbox{}\textit{\ \ \ \ \ \ \ \ \ *\ Works\ by\ looping\ through\ all\ directories\ in\ the\ \{}@linkplain\textit{\ StaffProgram\#STUDENT$\_$DIR\ student\ directory\}} \\
\mbox{}\textit{\ \ \ \ \ \ \ \ \ *\ looking\ for\ those\ with\ more\ than\ 1\ file\ contained\ within\ them\ (bearing\ in\ mind\ a\ directory\ for\ results} \\
\mbox{}\textit{\ \ \ \ \ \ \ \ \ *\ is\ created\ in\ the\ student's\ directory\ when\ it's\ created).\ And\ writing\ their\ username\ (which\ is\ handily} \\
\mbox{}\textit{\ \ \ \ \ \ \ \ \ *\ the\ name\ of\ the\ directory\ too)\ to\ a\ String.} \\
\mbox{}\textit{\ \ \ \ \ \ \ \ \ *\ }\textbf{$<$p$>$} \\
\mbox{}\textit{\ \ \ \ \ \ \ \ \ *\ Assuming\ any\ Students\ have\ made\ a\ submission,\ this\ String\ is\ then\ printed,\ otherwise\ a\ message\ to\ tell} \\
\mbox{}\textit{\ \ \ \ \ \ \ \ \ *\ the\ user\ than\ no\ Students\ have\ made\ submissions\ yet\ is\ displayed.} \\
\mbox{}\textit{\ \ \ \ \ \ \ \ \ *} \\
\mbox{}\textit{\ \ \ \ \ \ \ \ \ *\ }@see\textit{\ FileHandller\#getFile()} \\
\mbox{}\textit{\ \ \ \ \ \ \ \ \ *\ }@see\textit{\ File\#listFiles()} \\
\mbox{}\textit{\ \ \ \ \ \ \ \ \ */} \\
\mbox{}\ \ \ \ \ \ \ \ \textbf{public}\ void\ \textbf{list}()\ \{ \\
\mbox{}\ \ \ \ \ \ \ \ \ \ \ \ \ \ \ \ \textit{//fileHandller\ =\ new\ FileHandller(StaffProgram.STUDENT$\_$DIR);} \\
\mbox{}\ \ \ \ \ \ \ \ \ \ \ \ \ \ \ \ \textit{//TODO\ -\ Cannot\ currently\ lock\ directories.} \\
\mbox{}\ \ \ \ \ \ \ \ \ \ \ \ \ \ \ \ File\ studentDir\ =\ \textbf{new}\ \textbf{File}(StaffProgram.STUDENT$\_$DIR);\ \textit{//fileHandller.getFile();} \\
\mbox{}\ \ \ \ \ \ \ \ \ \ \ \ \ \ \ \ File[]\ students\ =\ studentDir.\textbf{listFiles}(); \\
\mbox{}\ \ \ \ \ \ \ \ \ \ \ \ \ \ \ \ boolean\ submission\ =\ \textbf{false}; \\
\mbox{}\ \ \ \ \ \ \ \ \ \ \ \ \ \ \ \ String\ submissions\ =\ \texttt{"{}The\ following\ students\ have\ made\ submissions:}\texttt{\textbackslash{}n}\texttt{"{}}; \\
\mbox{}\ \ \ \ \ \ \ \ \ \ \ \ \ \ \ \ \textbf{for}(File\ student:students)\ \{ \\
\mbox{}\ \ \ \ \ \ \ \ \ \ \ \ \ \ \ \ \ \ \ \ \ \ \ \ \textbf{if}(student.\textbf{listFiles}().length\ $>$\ 1)\ \{ \\
\mbox{}\ \ \ \ \ \ \ \ \ \ \ \ \ \ \ \ \ \ \ \ \ \ \ \ \ \ \ \ \ \ \ \ submission\ =\ \textbf{true}; \\
\mbox{}\ \ \ \ \ \ \ \ \ \ \ \ \ \ \ \ \ \ \ \ \ \ \ \ \ \ \ \ \ \ \ \ submissions\ +=\ \texttt{"{}}\texttt{\textbackslash{}t}\texttt{"{}}\ +\ student.\textbf{getName}()\ +\ \texttt{"{}}\texttt{\textbackslash{}n}\texttt{"{}}; \\
\mbox{}\ \ \ \ \ \ \ \ \ \ \ \ \ \ \ \ \ \ \ \ \ \ \ \ \} \\
\mbox{}\ \ \ \ \ \ \ \ \ \ \ \ \ \ \ \ \} \\
\mbox{}\ \ \ \ \ \ \ \ \ \ \ \ \ \ \ \ \textbf{if}(submission)\ \{ \\
\mbox{}\ \ \ \ \ \ \ \ \ \ \ \ \ \ \ \ \ \ \ \ \ \ \ \ System.out.\textbf{print}(submissions); \\
\mbox{}\ \ \ \ \ \ \ \ \ \ \ \ \ \ \ \ \}\ \textbf{else}\ \{ \\
\mbox{}\ \ \ \ \ \ \ \ \ \ \ \ \ \ \ \ \ \ \ \ \ \ \ \ System.out.\textbf{println}(\texttt{"{}No\ Students\ have\ made\ submissions"{}}); \\
\mbox{}\ \ \ \ \ \ \ \ \ \ \ \ \ \ \ \ \} \\
\mbox{} \\
\mbox{}\ \ \ \ \ \ \ \ \ \ \ \ \ \ \ \ logger.\textbf{log}(\texttt{"{}Listed\ Submissions"{}},StaffProgram.STUDENT$\_$DIR); \\
\mbox{} \\
\mbox{}\ \ \ \ \ \ \ \ \ \ \ \ \ \ \ \ \textit{//fileHandller.close();} \\
\mbox{}\ \ \ \ \ \ \ \ \} \\
\mbox{} \\
\mbox{}\ \ \ \ \ \ \ \ \textbf{public}\ void\ \textbf{download}(String\ uid,\ String\ destination)\ \{ \\
\mbox{}\ \ \ \ \ \ \ \ \ \ \ \ \ \ \ \ File\ studentDir\ =\ \textbf{new}\ \textbf{File}(StaffProgram.STUDENT$\_$DIR); \\
\mbox{}\ \ \ \ \ \ \ \ \ \ \ \ \ \ \ \ File[]\ studentDirs\ =\ studentDir.\textbf{listFiles}(); \\
\mbox{} \\
\mbox{}\ \ \ \ \ \ \ \ \ \ \ \ \ \ \ \ \textbf{for}(File\ student:studentDirs)\ \{ \\
\mbox{}\ \ \ \ \ \ \ \ \ \ \ \ \ \ \ \ \ \ \ \ \ \ \ \ \textbf{if}(student.\textbf{getName}().\textbf{equals}(uid))\ \{ \\
\mbox{}\ \ \ \ \ \ \ \ \ \ \ \ \ \ \ \ \ \ \ \ \ \ \ \ \ \ \ \ \ \ \ \ File[]\ studentFiles\ =\ student.\textbf{listFiles}(); \\
\mbox{}\ \ \ \ \ \ \ \ \ \ \ \ \ \ \ \ \ \ \ \ \ \ \ \ \ \ \ \ \ \ \ \ \textbf{if}(studentFiles.length\ $>$\ 1)\ \{ \\
\mbox{}\ \ \ \ \ \ \ \ \ \ \ \ \ \ \ \ \ \ \ \ \ \ \ \ \ \ \ \ \ \ \ \ \ \ \ \ \ \ \ \ \textbf{for}(File\ assignFile:studentFiles)\ \{ \\
\mbox{}\ \ \ \ \ \ \ \ \ \ \ \ \ \ \ \ \ \ \ \ \ \ \ \ \ \ \ \ \ \ \ \ \ \ \ \ \ \ \ \ \ \ \ \ \ \ \ \ \textbf{if}(!(assignFile.\textbf{getName}().\textbf{equals}(\texttt{"{}results"{}})))\ \{ \\
\mbox{}\ \ \ \ \ \ \ \ \ \ \ \ \ \ \ \ \ \ \ \ \ \ \ \ \ \ \ \ \ \ \ \ \ \ \ \ \ \ \ \ \ \ \ \ \ \ \ \ \ \ \ \ \ \ \ \ fileHandller\ =\ \textbf{new}\ \textbf{FileHandller}(assignFile.\textbf{getAbsolutePath}()); \\
\mbox{}\ \ \ \ \ \ \ \ \ \ \ \ \ \ \ \ \ \ \ \ \ \ \ \ \ \ \ \ \ \ \ \ \ \ \ \ \ \ \ \ \ \ \ \ \ \ \ \ \ \ \ \ \ \ \ \ fileHandller.\textbf{downloadFile}(destination); \\
\mbox{}\ \ \ \ \ \ \ \ \ \ \ \ \ \ \ \ \ \ \ \ \ \ \ \ \ \ \ \ \ \ \ \ \ \ \ \ \ \ \ \ \ \ \ \ \ \ \ \ \ \ \ \ \ \ \ \ fileHandller.\textbf{close}(); \\
\mbox{}\ \ \ \ \ \ \ \ \ \ \ \ \ \ \ \ \ \ \ \ \ \ \ \ \ \ \ \ \ \ \ \ \ \ \ \ \ \ \ \ \ \ \ \ \ \ \ \ \ \ \ \ \ \ \ \ logger.\textbf{log}(\texttt{"{}Downloaded\ assignment"{}},\ assignFile.\textbf{getAbsolutePath}()); \\
\mbox{}\ \ \ \ \ \ \ \ \ \ \ \ \ \ \ \ \ \ \ \ \ \ \ \ \ \ \ \ \ \ \ \ \ \ \ \ \ \ \ \ \ \ \ \ \ \ \ \ \ \ \ \ \ \ \ \ \textbf{return}; \\
\mbox{}\ \ \ \ \ \ \ \ \ \ \ \ \ \ \ \ \ \ \ \ \ \ \ \ \ \ \ \ \ \ \ \ \ \ \ \ \ \ \ \ \ \ \ \ \ \ \ \ \} \\
\mbox{}\ \ \ \ \ \ \ \ \ \ \ \ \ \ \ \ \ \ \ \ \ \ \ \ \ \ \ \ \ \ \ \ \ \ \ \ \ \ \ \ \} \\
\mbox{}\ \ \ \ \ \ \ \ \ \ \ \ \ \ \ \ \ \ \ \ \ \ \ \ \ \ \ \ \ \ \ \ \} \\
\mbox{}\ \ \ \ \ \ \ \ \ \ \ \ \ \ \ \ \ \ \ \ \ \ \ \ \} \\
\mbox{}\ \ \ \ \ \ \ \ \ \ \ \ \ \ \ \ \} \\
\mbox{}\ \ \ \ \ \ \ \ \ \ \ \ \ \ \ \ System.out.\textbf{println}(\texttt{"{}Could\ not\ find\ assigment\ of\ student\ with\ uid:\ "{}}+uid); \\
\mbox{}\ \ \ \ \ \ \ \ \} \\
\mbox{} \\
\mbox{}\ \ \ \ \ \ \ \ \textbf{public}\ void\ \textbf{mark}(String\ uid)\ \{ \\
\mbox{}\ \ \ \ \ \ \ \ \ \ \ \ \ \ \ \ File\ studentDir\ =\ \textbf{new}\ \textbf{File}(StaffProgram.STUDENT$\_$DIR); \\
\mbox{}\ \ \ \ \ \ \ \ \ \ \ \ \ \ \ \ File[]\ studentDirs\ =\ studentDir.\textbf{listFiles}(); \\
\mbox{}\ \ \ \ \ \ \ \ \ \ \ \ \ \ \ \ \textbf{for}(File\ student:studentDirs)\ \{ \\
\mbox{}\ \ \ \ \ \ \ \ \ \ \ \ \ \ \ \ \ \ \ \ \ \ \ \ \textbf{if}(student.\textbf{getName}().\textbf{equals}(uid))\ \{ \\
\mbox{}\ \ \ \ \ \ \ \ \ \ \ \ \ \ \ \ \ \ \ \ \ \ \ \ \ \ \ \ \ \ \ \ File[]\ studentFiles\ =\ student.\textbf{listFiles}(); \\
\mbox{}\ \ \ \ \ \ \ \ \ \ \ \ \ \ \ \ \ \ \ \ \ \ \ \ \ \ \ \ \ \ \ \ \textbf{if}(studentFiles.length\ $>$\ 1)\ \{ \\
\mbox{}\ \ \ \ \ \ \ \ \ \ \ \ \ \ \ \ \ \ \ \ \ \ \ \ \ \ \ \ \ \ \ \ \ \ \ \ \ \ \ \ File\ markFile\ =\ \textbf{new}\ \textbf{File}(student.\textbf{getAbsolutePath}()+\texttt{"{}/results/result.txt"{}}); \\
\mbox{}\ \ \ \ \ \ \ \ \ \ \ \ \ \ \ \ \ \ \ \ \ \ \ \ \ \ \ \ \ \ \ \ \ \ \ \ \ \ \ \ \textbf{if}(!markFile.\textbf{exists}())\ \{ \\
\mbox{}\ \ \ \ \ \ \ \ \ \ \ \ \ \ \ \ \ \ \ \ \ \ \ \ \ \ \ \ \ \ \ \ \ \ \ \ \ \ \ \ \ \ \ \ \ \ \ \ \textbf{try}\ \{ \\
\mbox{}\ \ \ \ \ \ \ \ \ \ \ \ \ \ \ \ \ \ \ \ \ \ \ \ \ \ \ \ \ \ \ \ \ \ \ \ \ \ \ \ \ \ \ \ \ \ \ \ \ \ \ \ \ \ \ \ System.out.\textbf{println}(markFile.\textbf{getAbsolutePath}()); \\
\mbox{}\ \ \ \ \ \ \ \ \ \ \ \ \ \ \ \ \ \ \ \ \ \ \ \ \ \ \ \ \ \ \ \ \ \ \ \ \ \ \ \ \ \ \ \ \ \ \ \ \ \ \ \ \ \ \ \ markFile.\textbf{createNewFile}(); \\
\mbox{}\ \ \ \ \ \ \ \ \ \ \ \ \ \ \ \ \ \ \ \ \ \ \ \ \ \ \ \ \ \ \ \ \ \ \ \ \ \ \ \ \ \ \ \ \ \ \ \ \ \ \ \ \ \ \ \ fileHandller\ =\ \textbf{new}\ \textbf{FileHandller}(markFile.\textbf{getAbsolutePath}()); \\
\mbox{}\ \ \ \ \ \ \ \ \ \ \ \ \ \ \ \ \ \ \ \ \ \ \ \ \ \ \ \ \ \ \ \ \ \ \ \ \ \ \ \ \ \ \ \ \ \ \ \ \ \ \ \ \ \ \ \ Scanner\ in\ =\ \textbf{new}\ \textbf{Scanner}(System.in); \\
\mbox{}\ \ \ \ \ \ \ \ \ \ \ \ \ \ \ \ \ \ \ \ \ \ \ \ \ \ \ \ \ \ \ \ \ \ \ \ \ \ \ \ \ \ \ \ \ \ \ \ \ \ \ \ \ \ \ \ System.out.\textbf{print}(\texttt{"{}Enter\ the\ mark\ (\%):\ "{}}); \\
\mbox{}\ \ \ \ \ \ \ \ \ \ \ \ \ \ \ \ \ \ \ \ \ \ \ \ \ \ \ \ \ \ \ \ \ \ \ \ \ \ \ \ \ \ \ \ \ \ \ \ \ \ \ \ \ \ \ \ int\ mark\ =\ in.\textbf{nextInt}(); \\
\mbox{}\ \ \ \ \ \ \ \ \ \ \ \ \ \ \ \ \ \ \ \ \ \ \ \ \ \ \ \ \ \ \ \ \ \ \ \ \ \ \ \ \ \ \ \ \ \ \ \ \ \ \ \ \ \ \ \ in.\textbf{nextLine}(); \\
\mbox{}\ \ \ \ \ \ \ \ \ \ \ \ \ \ \ \ \ \ \ \ \ \ \ \ \ \ \ \ \ \ \ \ \ \ \ \ \ \ \ \ \ \ \ \ \ \ \ \ \ \ \ \ \ \ \ \ System.out.\textbf{println}(\texttt{"{}Enter\ any\ comments:\ "{}}); \\
\mbox{}\ \ \ \ \ \ \ \ \ \ \ \ \ \ \ \ \ \ \ \ \ \ \ \ \ \ \ \ \ \ \ \ \ \ \ \ \ \ \ \ \ \ \ \ \ \ \ \ \ \ \ \ \ \ \ \ String\ comment\ =\ in.\textbf{nextLine}(); \\
\mbox{}\ \ \ \ \ \ \ \ \ \ \ \ \ \ \ \ \ \ \ \ \ \ \ \ \ \ \ \ \ \ \ \ \ \ \ \ \ \ \ \ \ \ \ \ \ \ \ \  \\
\mbox{}\ \ \ \ \ \ \ \ \ \ \ \ \ \ \ \ \ \ \ \ \ \ \ \ \ \ \ \ \ \ \ \ \ \ \ \ \ \ \ \ \ \ \ \ \ \ \ \ \ \ \ \ \ \ \ \ fileHandller.\textbf{appendFile}(String.\textbf{format}(\texttt{"{}\%d\%\%}\texttt{\textbackslash{}n}\texttt{\%s}\texttt{\textbackslash{}n}\texttt{"{}},\ mark,\ comment)); \\
\mbox{}\ \ \ \ \ \ \ \ \ \ \ \ \ \ \ \ \ \ \ \ \ \ \ \ \ \ \ \ \ \ \ \ \ \ \ \ \ \ \ \ \ \ \ \ \ \ \ \ \ \ \ \ \ \ \ \ fileHandller.\textbf{close}(); \\
\mbox{} \\
\mbox{}\ \ \ \ \ \ \ \ \ \ \ \ \ \ \ \ \ \ \ \ \ \ \ \ \ \ \ \ \ \ \ \ \ \ \ \ \ \ \ \ \ \ \ \ \ \ \ \ \ \ \ \ \ \ \ \ logger.\textbf{log}(\texttt{"{}Submitted\ mark\ for\ "{}}+uid,\ markFile.\textbf{getAbsolutePath}()); \\
\mbox{}\ \ \ \ \ \ \ \ \ \ \ \ \ \ \ \ \ \ \ \ \ \ \ \ \ \ \ \ \ \ \ \ \ \ \ \ \ \ \ \ \ \ \ \ \ \ \ \ \ \ \ \ \ \ \ \ \textbf{return}; \\
\mbox{}\ \ \ \ \ \ \ \ \ \ \ \ \ \ \ \ \ \ \ \ \ \ \ \ \ \ \ \ \ \ \ \ \ \ \ \ \ \ \ \ \ \ \ \ \ \ \ \ \}\ \textbf{catch}\ (IOException\ e)\ \{ \\
\mbox{}\ \ \ \ \ \ \ \ \ \ \ \ \ \ \ \ \ \ \ \ \ \ \ \ \ \ \ \ \ \ \ \ \ \ \ \ \ \ \ \ \ \ \ \ \ \ \ \ \ \ \ \ \ \ \ \ System.err.\textbf{println}(e); \\
\mbox{}\ \ \ \ \ \ \ \ \ \ \ \ \ \ \ \ \ \ \ \ \ \ \ \ \ \ \ \ \ \ \ \ \ \ \ \ \ \ \ \ \ \ \ \ \ \ \ \ \} \\
\mbox{}\ \ \ \ \ \ \ \ \ \ \ \ \ \ \ \ \ \ \ \ \ \ \ \ \ \ \ \ \ \ \ \ \ \ \ \ \ \ \ \ \} \\
\mbox{}\ \ \ \ \ \ \ \ \ \ \ \ \ \ \ \ \ \ \ \ \ \ \ \ \ \ \ \ \ \ \ \ \} \\
\mbox{}\ \ \ \ \ \ \ \ \ \ \ \ \ \ \ \ \ \ \ \ \ \ \ \ \} \\
\mbox{}\ \ \ \ \ \ \ \ \ \ \ \ \ \ \ \ \} \\
\mbox{} \\
\mbox{}\ \ \ \ \ \ \ \ \ \ \ \ \ \ \ \ System.err.\textbf{println}(\texttt{"{}Cannot\ submit\ mark:\ Student\ does\ not\ exist\ or\ has\ not\ yet\ submitted"{}}); \\
\mbox{}\ \ \ \ \ \ \ \ \} \\
\mbox{}\} \\

\clearpage

		\section{Source code - Student Program}
\normalsize
\rmfamily
\subsection{file\_lock.c}
\scriptsize
\sffamily
% Generator: GNU source-highlight, by Lorenzo Bettini, http://www.gnu.org/software/src-highlite
\noindent
\mbox{}\textit{/**} \\
\mbox{}\textit{\ *\ }@author\textit{\ Alexander\ Brown} \\
\mbox{}\textit{\ */} \\
\mbox{} \\
\mbox{}\textbf{\#include}\ \texttt{$<$stdlib.h$>$} \\
\mbox{}\textbf{\#include}\ \texttt{$<$fcntl.h$>$} \\
\mbox{}\textbf{\#include}\ \texttt{$<$errno.h$>$} \\
\mbox{}\textbf{\#include}\ \texttt{$<$unistd.h$>$} \\
\mbox{}\textbf{\#include}\ \texttt{$<$stdio.h$>$} \\
\mbox{} \\
\mbox{}\textbf{\#include}\ \texttt{"{}file$\_$lock.h"{}} \\
\mbox{} \\
\mbox{}\textit{/**} \\
\mbox{}\textit{\ *\ }@author\textit{\ Neal\ Snooke} \\
\mbox{}\textit{\ */} \\
\mbox{}\textbf{struct}\ flock*\ \textbf{file$\_$lock}(short\ type,\ short\ whence)\ \{ \\
\mbox{}\ \ \ \ \textbf{static}\ \textbf{struct}\ flock\ ret; \\
\mbox{}\ \ \ \ ret.l$\_$type\ =\ type; \\
\mbox{}\ \ \ \ ret.l$\_$start\ =\ 0; \\
\mbox{}\ \ \ \ ret.l$\_$whence\ =\ whence; \\
\mbox{}\ \ \ \ ret.l$\_$len\ =\ 0; \\
\mbox{}\ \ \ \ ret.l$\_$pid\ =\ \textbf{getpid}(); \\
\mbox{}\ \ \ \ \textbf{return}\ \&ret; \\
\mbox{}\} \\
\mbox{} \\
\mbox{}int\ \textbf{lock}(char\ *\ filepath)\ \{ \\
\mbox{}\ \ \ \ \ \ \ \ \textbf{printf}(\texttt{"{}Locking\ file...}\texttt{\textbackslash{}n}\texttt{"{}}); \\
\mbox{}\ \ \ \ \ \ \ \ int\ file$\_$descriptor\ =\ \textbf{open}(filepath,\ O$\_$RDWR); \\
\mbox{}\ \ \ \ \ \ \ \  \\
\mbox{}\ \ \ \ \ \ \ \ \textbf{if}(file$\_$descriptor\ ==\ -1)\ \{ \\
\mbox{}\ \ \ \ \ \ \ \ \ \ \ \ \ \ \ \ \textbf{printf}(\texttt{"{}Could\ not\ open\ \%s:\ File\ Descriptor\ Error}\texttt{\textbackslash{}n}\texttt{"{}},\ filepath); \\
\mbox{}\ \ \ \ \ \ \ \ \ \ \ \ \ \ \ \ \textbf{return}\ -1; \\
\mbox{}\ \ \ \ \ \ \ \ \} \\
\mbox{} \\
\mbox{}\ \ \ \ \ \ \ \ \textbf{struct}\ flock*\ fl\ =\ \textbf{file$\_$lock}(F$\_$WRLCK,\ SEEK$\_$SET); \\
\mbox{}\ \ \ \ \ \ \ \  \\
\mbox{}\ \ \ \ \ \ \ \ \textbf{if}(\textbf{fcntl}(file$\_$descriptor,\ F$\_$SETLK,\ fl)\ ==\ -1)\ \{ \\
\mbox{}\ \ \ \ \ \ \ \ \ \ \ \ \ \ \ \ \textbf{if}(errno\ ==\ EACCES\ $|$$|$\ errno\ ==\ EAGAIN)\ \{ \\
\mbox{}\ \ \ \ \ \ \ \ \ \ \ \ \ \ \ \ \ \ \ \ \ \ \ \ \textbf{printf}(\texttt{"{}Unable\ to\ lock\ \%s:\ File\ is\ already\ locked\ by\ another\ process.}\texttt{\textbackslash{}n}\texttt{"{}},\ filepath); \\
\mbox{}\ \ \ \ \ \ \ \ \ \ \ \ \ \ \ \ \}\ \textbf{else}\ \{ \\
\mbox{}\ \ \ \ \ \ \ \ \ \ \ \ \ \ \ \ \ \ \ \ \ \ \ \ \textbf{printf}(\texttt{"{}Unable\ to\ lock\ \%s:\ Unknown\ error.}\texttt{\textbackslash{}n}\texttt{"{}},\ filepath); \\
\mbox{}\ \ \ \ \ \ \ \ \ \ \ \ \ \ \ \ \} \\
\mbox{}\ \ \ \ \ \ \ \ \ \ \ \ \ \ \ \ \textbf{return}\ -1; \\
\mbox{}\ \ \ \ \ \ \ \ \}\ \textbf{else}\ \{\ \textit{//\ Lock\ has\ been\ obtained.} \\
\mbox{}\ \ \ \ \ \ \ \ \ \ \ \ \ \ \ \ \textbf{return}\ file$\_$descriptor; \\
\mbox{}\ \ \ \ \ \ \ \ \} \\
\mbox{}\} \\
\mbox{} \\
\mbox{}int\ \textbf{unlock}(int\ file$\_$descriptor)\ \{ \\
\mbox{}\ \ \ \ \ \ \ \ \textbf{if}(file$\_$descriptor\ ==\ -1)\ \{ \\
\mbox{}\ \ \ \ \ \ \ \ \ \ \ \ \ \ \ \ \textbf{printf}(\texttt{"{}Cannot\ unlock\ file:\ File\ Descriptor\ Error}\texttt{\textbackslash{}n}\texttt{"{}}); \\
\mbox{}\ \ \ \ \ \ \ \ \ \ \ \ \ \ \ \ \textbf{return}\ 0; \\
\mbox{}\ \ \ \ \ \ \ \ \}\ \textbf{else}\ \{ \\
\mbox{}\ \ \ \ \ \ \ \ \ \ \ \ \ \ \ \ \textbf{fcntl}(file$\_$descriptor,\ F$\_$SETLKW,\ \textbf{file$\_$lock}(F$\_$UNLCK,\ SEEK$\_$SET)); \\
\mbox{}\ \ \ \ \ \ \ \ \ \ \ \ \ \ \ \ \textbf{close}(file$\_$descriptor); \\
\mbox{}\ \ \ \ \ \ \ \ \ \ \ \ \ \ \ \ \textbf{return}\ 1; \\
\mbox{}\ \ \ \ \ \ \ \ \} \\
\mbox{}\} \\

\clearpage
\normalsize
\rmfamily
\subsection{file\_lock.h}
\scriptsize
\sffamily
% Generator: GNU source-highlight, by Lorenzo Bettini, http://www.gnu.org/software/src-highlite
\noindent
\mbox{}\textit{/**} \\
\mbox{}\textit{\ *\ }@brief\textit{\ Creates\ a\ lock\ on\ a\ file.} \\
\mbox{}\textit{\ *} \\
\mbox{}\textit{\ *\ }@author\textit{\ Alexander\ Brown} \\
\mbox{}\textit{\ *\ }@version\textit{\ 1.0} \\
\mbox{}\textit{\ */} \\
\mbox{} \\
\mbox{}\textit{/**} \\
\mbox{}\textit{\ *\ }@brief\textit{\ Locks\ a\ file.} \\
\mbox{}\textit{\ *} \\
\mbox{}\textit{\ *\ }@param\textit{\ filepath\ The\ path\ to\ the\ file.} \\
\mbox{}\textit{\ *\ }@return\textit{\ The\ file\ descriptor\ of\ the\ file\ to\ lock,\ -1\ if\ anything\ fails.} \\
\mbox{}\textit{\ */} \\
\mbox{}int\ \textbf{lock}(char\ *\ filepath); \\
\mbox{} \\
\mbox{}\textit{/**} \\
\mbox{}\textit{\ *\ }@brief\textit{\ Unlocks\ a\ file.} \\
\mbox{}\textit{\ *} \\
\mbox{}\textit{\ *\ }@param\textit{\ file$\_$descriptor} \\
\mbox{}\textit{\ *\ }@return \\
\mbox{}\textit{\ *\ \ \ \ \ \ \ \ \ 1\ -\ If\ the\ file\ unlocked\ sucessfully.} \\
\mbox{}\textit{\ *\ \ \ \ \ \ \ \ \ 0\ -\ If\ not.} \\
\mbox{}\textit{\ */} \\
\mbox{}int\ \textbf{unlock}(int\ file$\_$descriptor); \\

\clearpage
\normalsize
\rmfamily
\subsection{logger.c}
\scriptsize
\sffamily
% Generator: GNU source-highlight, by Lorenzo Bettini, http://www.gnu.org/software/src-highlite
\noindent
\mbox{}\textit{/*} \\
\mbox{}\textit{\ *\ logger.c\ copyright\ (c)\ Alexander\ Brown,\ March\ 2011} \\
\mbox{}\textit{\ */} \\
\mbox{} \\
\mbox{}\textbf{\#include}\ \texttt{$<$stdlib.h$>$} \\
\mbox{}\textbf{\#include}\ \texttt{$<$stdio.h$>$} \\
\mbox{}\textbf{\#include}\ \texttt{$<$string.h$>$} \\
\mbox{}\textbf{\#include}\ \texttt{$<$time.h$>$} \\
\mbox{} \\
\mbox{}\textbf{\#include}\ \texttt{"{}utils.h"{}} \\
\mbox{}\textbf{\#include}\ \texttt{"{}logger.h"{}} \\
\mbox{}\textbf{\#include}\ \texttt{"{}file$\_$lock.h"{}} \\
\mbox{} \\
\mbox{}\textbf{\#ifndef}\ MAX$\_$ENTRY$\_$SIZE \\
\mbox{}\textbf{\#define}\ MAX$\_$ENTRY$\_$SIZE\ 100 \\
\mbox{}\textbf{\#endif} \\
\mbox{} \\
\mbox{}char\ *\ logfile; \\
\mbox{} \\
\mbox{}void\ \textbf{get$\_$logfile}(void)\ \{ \\
\mbox{}\ \ \ \ \ \ \ \ \textbf{if}(logfile\ !=\ NULL)\ \{ \\
\mbox{}\ \ \ \ \ \ \ \ \ \ \ \ \ \ \ \ \textbf{return}; \\
\mbox{}\ \ \ \ \ \ \ \ \} \\
\mbox{}\ \ \ \ \ \ \ \ logfile\ =\ \textbf{malloc}(115\ *\ \textbf{sizeof}(char)); \\
\mbox{}\ \ \ \ \ \ \ \ char\ *\ base$\_$dir\ =\ \textbf{malloc}(85\ *\ \textbf{sizeof}(char)); \\
\mbox{}\ \ \ \ \ \ \ \ \textbf{get$\_$base$\_$dir}(base$\_$dir); \\
\mbox{}\ \ \ \ \ \ \ \ \textbf{strcat}(logfile,\ base$\_$dir); \\
\mbox{}\ \ \ \ \ \ \ \ \textbf{strcat}(logfile,\ \texttt{"{}/logfile.log"{}}); \\
\mbox{}\ \ \ \ \ \ \ \ \textbf{free}(base$\_$dir); \\
\mbox{}\} \\
\mbox{} \\
\mbox{}void\ \textbf{log}(char\ *\ activity,\ char\ *\ uid,\ char\ *\ filepath)\ \{ \\
\mbox{}\ \ \ \ \ \ \ \ \textbf{get$\_$logfile}(); \\
\mbox{}\ \ \ \ \ \ \ \ int\ fd\ =\ \textbf{lock}(logfile); \\
\mbox{}\ \ \ \ \ \ \ \  \\
\mbox{}\ \ \ \ \ \ \ \ FILE\ *\ logfile\ =\ \textbf{fdopen}(fd,\ \texttt{"{}a"{}}); \\
\mbox{}\ \ \ \ \ \ \ \ \textbf{printf}(\texttt{"{}File\ open.}\texttt{\textbackslash{}n}\texttt{"{}}); \\
\mbox{} \\
\mbox{}\ \ \ \ \ \ \ \ char\ *\ entry\ =\ \textbf{malloc}(MAX$\_$ENTRY$\_$SIZE\ *\ \textbf{sizeof}(char)); \\
\mbox{}\ \ \ \ \ \ \ \ \textbf{generate$\_$log$\_$entry}(entry,\ activity,\ uid,\ filepath);\ \ \ \ \ \ \ \  \\
\mbox{}\ \ \ \ \ \ \ \ \textbf{fprintf}(logfile,\ \texttt{"{}\%s"{}},\ entry); \\
\mbox{}\ \ \ \ \ \ \ \ \textbf{fclose}(logfile); \\
\mbox{} \\
\mbox{}\ \ \ \ \ \ \ \ \textbf{unlock}(fd); \\
\mbox{}\ \ \ \ \ \ \ \ \textbf{free}(entry); \\
\mbox{}\} \\
\mbox{} \\
\mbox{}void\ \textbf{generate$\_$log$\_$entry}(char\ *\ entry,\ char\ *\ activity,\ char\ *\ uid,\ char\ *\ filepath)\ \{ \\
\mbox{}\ \ \ \ \ \ \ \ \textit{//yyyy-mm-dd-hh-mm-ss\ stdnt:\ \%uid\ \ \ \ \ \ \ \ \%activity\ \ \ \ \ \ \ \ \%filepath} \\
\mbox{}\ \ \ \ \ \ \ \ \textbf{encode$\_$date$\_$time}(entry); \\
\mbox{}\ \ \ \ \ \ \ \ \textbf{strcat}(entry,\ \texttt{"{}\ stdnt:\ "{}}); \\
\mbox{}\ \ \ \ \ \ \ \ \textbf{strcat}(entry,\ uid); \\
\mbox{}\ \ \ \ \ \ \ \ \textbf{strcat}(entry,\ \texttt{"{}}\texttt{\textbackslash{}t}\texttt{"{}}); \\
\mbox{}\ \ \ \ \ \ \ \ \textbf{strcat}(entry,\ activity); \\
\mbox{}\ \ \ \ \ \ \ \ \textbf{strcat}(entry,\ \texttt{"{}}\texttt{\textbackslash{}t}\texttt{"{}}); \\
\mbox{}\ \ \ \ \ \ \ \ \textbf{strcat}(entry,\ filepath); \\
\mbox{}\ \ \ \ \ \ \ \ \textbf{strcat}(entry,\ \texttt{"{}}\texttt{\textbackslash{}n}\texttt{"{}}); \\
\mbox{}\} \\
\mbox{} \\
\mbox{}void\ \textbf{encode$\_$date$\_$time}(char\ *\ entry)\ \{ \\
\mbox{}\ \ \ \ \ \ \ \ time$\_$t\ rawtime; \\
\mbox{}\ \ \ \ \ \ \ \ \textbf{struct}\ tm\ *\ timeinfo; \\
\mbox{}\ \ \ \ \ \ \ \ char\ time$\_$str[21]; \\
\mbox{}\ \ \ \ \ \ \ \  \\
\mbox{}\ \ \ \ \ \ \ \ \textbf{time}(\&rawtime); \\
\mbox{}\ \ \ \ \ \ \ \ timeinfo\ =\ \textbf{localtime}(\&rawtime); \\
\mbox{}\ \ \ \ \ \ \ \ \textbf{strftime}(time$\_$str,\ 21,\ \texttt{"{}\%Y-\%m-\%d-\%H-\%M-\%S"{}},\ timeinfo); \\
\mbox{}\ \ \ \ \ \ \ \ \textbf{strcat}(entry,\ time$\_$str); \\
\mbox{}\} \\
\mbox{} \\
\mbox{}void\ \textbf{free$\_$logger}()\ \{ \\
\mbox{}\ \ \ \ \ \ \ \ \textbf{free}(logfile); \\
\mbox{}\} \\

\clearpage
\normalsize
\rmfamily
\subsection{logger.h}
\scriptsize
\sffamily
% Generator: GNU source-highlight, by Lorenzo Bettini, http://www.gnu.org/software/src-highlite
\noindent
\mbox{} \\
\mbox{}\textbf{\#ifndef}\ $\_$LOGGER$\_$H \\
\mbox{}\textbf{\#define}\ $\_$LOGGER$\_$H \\
\mbox{} \\
\mbox{}\textit{/**} \\
\mbox{}\textit{\ *\ Logs\ an\ activity.} \\
\mbox{}\textit{\ *} \\
\mbox{}\textit{\ *\ }@param\textit{\ activity\ \ \ \ \ \ \ \ The\ activity\ performed.} \\
\mbox{}\textit{\ *\ }@param\textit{\ uid\ \ \ \ \ \ \ \ \ \ \ \ \ \ \ \ The\ user\ ID\ of\ the\ user.} \\
\mbox{}\textit{\ *\ }@param\textit{\ filepath\ \ \ \ \ \ \ \ The\ path\ to\ the\ file\ that\ the\ activity\ was\ performed\ upon.} \\
\mbox{}\textit{\ */} \\
\mbox{}void\ \textbf{log}(char\ *\ activity,\ char\ *\ uid,\ char\ *\ filepath); \\
\mbox{} \\
\mbox{}void\ \textbf{generate$\_$log$\_$entry}(char\ *\ entry,\ char\ *\ activity,\ char\ *\ uid,\ char\ *\ filepath); \\
\mbox{} \\
\mbox{}void\ \textbf{encode$\_$date$\_$time}(char\ *\ entry); \\
\mbox{} \\
\mbox{}void\ \textbf{free$\_$logger}(void); \\
\mbox{} \\
\mbox{}\textbf{\#endif}\ \textit{/*\ $\_$LOGGER$\_$H\ */} \\

\clearpage
\normalsize
\rmfamily
\subsection{main.c}
\scriptsize
\sffamily
% Generator: GNU source-highlight, by Lorenzo Bettini, http://www.gnu.org/software/src-highlite
\noindent
\mbox{}\textit{/**} \\
\mbox{}\textit{\ *\ }@author\textit{\ Alexander\ Brown} \\
\mbox{}\textit{\ */} \\
\mbox{} \\
\mbox{}\textbf{\#ifndef}\ LOGIN$\_$FAILURE \\
\mbox{}\textbf{\#define}\ LOGIN$\_$FAILURE\ 12 \\
\mbox{}\textbf{\#endif} \\
\mbox{} \\
\mbox{}\textbf{\#include}\ \texttt{$<$stdlib.h$>$} \\
\mbox{}\textbf{\#include}\ \texttt{$<$stdio.h$>$} \\
\mbox{}\textbf{\#include}\ \texttt{"{}user.h"{}} \\
\mbox{}\textbf{\#include}\ \texttt{"{}logger.h"{}} \\
\mbox{} \\
\mbox{}\textbf{\#ifndef}\ MENU$\_$OPTIONS \\
\mbox{}\textbf{\#define}\ MENU$\_$OPTIONS \\
\mbox{}\textbf{\#define}\ QUIT\ \ \ \ \ \ \ \ \texttt{'q'} \\
\mbox{}\textbf{\#define}\ HELP\ \ \ \ \ \ \ \ \texttt{'h'} \\
\mbox{}\textbf{\#define}\ GET\ \ \ \ \ \ \ \ \texttt{'g'} \\
\mbox{}\textbf{\#define}\ SUBMIT\ \ \ \ \ \ \ \ \texttt{'s'} \\
\mbox{}\textbf{\#define}\ MARK\ \ \ \ \ \ \ \ \texttt{'m'} \\
\mbox{}\textbf{\#define}\ NONE\ \ \ \ \ \ \ \ \texttt{'\ '} \\
\mbox{}\textbf{\#endif} \\
\mbox{} \\
\mbox{}\textbf{static}\ char\ uid[7]; \\
\mbox{} \\
\mbox{}int\ \textbf{main}(int\ argc,\ char\ **\ argv)\ \{ \\
\mbox{}\ \ \ \ \ \ \ \ \textbf{get$\_$logfile}(); \\
\mbox{}\ \ \ \ \ \ \ \ \textbf{printf}(\texttt{"{}Enter\ User\ ID:\ "{}}); \\
\mbox{}\ \ \ \ \ \ \ \ \textbf{scanf}(\texttt{"{}\%6s"{}},\ uid); \\
\mbox{}\ \ \ \ \ \ \ \  \\
\mbox{}\ \ \ \ \ \ \ \ \textbf{if}(!(\textbf{login}(uid)\ ==\ 1))\ \{ \\
\mbox{}\ \ \ \ \ \ \ \ \ \ \ \ \ \ \ \ \textbf{printf}(\texttt{"{}Invalid\ Username/Password}\texttt{\textbackslash{}n}\texttt{"{}}); \\
\mbox{}\ \ \ \ \ \ \ \ \ \ \ \ \ \ \ \ \textbf{return}\ 12; \\
\mbox{}\ \ \ \ \ \ \ \ \} \\
\mbox{} \\
\mbox{}\ \ \ \ \ \ \ \ \textbf{if}(argc\ $<$\ 2)\ \{ \\
\mbox{}\ \ \ \ \ \ \ \ \ \ \ \ \ \ \ \ char\ option\ =\ NONE; \\
\mbox{}\ \ \ \ \ \ \ \ \ \ \ \ \ \ \ \ char\ path[100]; \\
\mbox{}\ \ \ \ \ \ \ \ \ \ \ \ \ \ \ \ \textbf{while}(option\ ==\ NONE)\ \{ \\
\mbox{}\ \ \ \ \ \ \ \ \ \ \ \ \ \ \ \ \ \ \ \ \ \ \ \ \textbf{printf}(\texttt{"{}Enter\ option\ code\ (h\ for\ help):\ "{}}); \\
\mbox{}\ \ \ \ \ \ \ \ \ \ \ \ \ \ \ \ \ \ \ \ \ \ \ \ \textbf{fflush}(stdin); \\
\mbox{}\ \ \ \ \ \ \ \ \ \ \ \ \ \ \ \ \ \ \ \ \ \ \ \ \textbf{scanf}(\texttt{"{}\%1s"{}},\ \&option); \\
\mbox{}\ \ \ \ \ \ \ \ \ \ \ \ \ \ \ \ \ \ \ \ \ \ \ \ \textbf{getchar}();\ \textit{//\ getchar()\ apparently\ stops\ this\ looping\ infinitely\ on\ my\ machine.} \\
\mbox{}\ \ \ \ \ \ \ \ \ \ \ \ \ \ \ \ \ \ \ \ \ \ \ \ \textbf{switch}(option)\ \{ \\
\mbox{}\ \ \ \ \ \ \ \ \ \ \ \ \ \ \ \ \ \ \ \ \ \ \ \ \ \ \ \ \ \ \ \ \textbf{case}\ GET\ : \\
\mbox{}\ \ \ \ \ \ \ \ \ \ \ \ \ \ \ \ \ \ \ \ \ \ \ \ \ \ \ \ \ \ \ \ \ \ \ \ \ \ \ \ \textbf{printf}(\texttt{"{}Location\ to\ download\ the\ assignment\ to:\ "{}}); \\
\mbox{}\ \ \ \ \ \ \ \ \ \ \ \ \ \ \ \ \ \ \ \ \ \ \ \ \ \ \ \ \ \ \ \ \ \ \ \ \ \ \ \ \textbf{scanf}(\texttt{"{}\%98s"{}},\ path); \\
\mbox{}\ \ \ \ \ \ \ \ \ \ \ \ \ \ \ \ \ \ \ \ \ \ \ \ \ \ \ \ \ \ \ \ \ \ \ \ \ \ \ \ \textbf{if}(\textbf{download$\_$assignment}(path)\ ==\ 1)\ \{ \\
\mbox{}\ \ \ \ \ \ \ \ \ \ \ \ \ \ \ \ \ \ \ \ \ \ \ \ \ \ \ \ \ \ \ \ \ \ \ \ \ \ \ \ \ \ \ \ \ \ \ \ \textbf{log}(\texttt{"{}Downloaded\ assignment"{}},\ uid,\ \texttt{"{}repository/assignment/"{}}); \\
\mbox{}\ \ \ \ \ \ \ \ \ \ \ \ \ \ \ \ \ \ \ \ \ \ \ \ \ \ \ \ \ \ \ \ \ \ \ \ \ \ \ \ \} \\
\mbox{}\ \ \ \ \ \ \ \ \ \ \ \ \ \ \ \ \ \ \ \ \ \ \ \ \ \ \ \ \ \ \ \ \ \ \ \ \ \ \ \ \textbf{break}; \\
\mbox{}\ \ \ \ \ \ \ \ \ \ \ \ \ \ \ \ \ \ \ \ \ \ \ \ \ \ \ \ \ \ \ \ \textbf{case}\ SUBMIT: \\
\mbox{}\ \ \ \ \ \ \ \ \ \ \ \ \ \ \ \ \ \ \ \ \ \ \ \ \ \ \ \ \ \ \ \ \ \ \ \ \ \ \ \ \textbf{printf}(\texttt{"{}Location\ of\ file\ to\ submit:\ "{}}); \\
\mbox{}\ \ \ \ \ \ \ \ \ \ \ \ \ \ \ \ \ \ \ \ \ \ \ \ \ \ \ \ \ \ \ \ \ \ \ \ \ \ \ \ \textbf{scanf}(\texttt{"{}\%98s"{}},\ path); \\
\mbox{}\ \ \ \ \ \ \ \ \ \ \ \ \ \ \ \ \ \ \ \ \ \ \ \ \ \ \ \ \ \ \ \ \ \ \ \ \ \ \ \ \textbf{if}(\textbf{submit$\_$assignment}(path,\ uid)\ ==\ 1)\ \{ \\
\mbox{}\ \ \ \ \ \ \ \ \ \ \ \ \ \ \ \ \ \ \ \ \ \ \ \ \ \ \ \ \ \ \ \ \ \ \ \ \ \ \ \ \ \ \ \ \ \ \ \ \textbf{strcpy}(path,\ \texttt{"{}repository/students/"{}}); \\
\mbox{}\ \ \ \ \ \ \ \ \ \ \ \ \ \ \ \ \ \ \ \ \ \ \ \ \ \ \ \ \ \ \ \ \ \ \ \ \ \ \ \ \ \ \ \ \ \ \ \ \textbf{strcat}(path,\ uid); \\
\mbox{}\ \ \ \ \ \ \ \ \ \ \ \ \ \ \ \ \ \ \ \ \ \ \ \ \ \ \ \ \ \ \ \ \ \ \ \ \ \ \ \ \ \ \ \ \ \ \ \ \textbf{strcat}(path,\ \texttt{"{}/"{}}); \\
\mbox{}\ \ \ \ \ \ \ \ \ \ \ \ \ \ \ \ \ \ \ \ \ \ \ \ \ \ \ \ \ \ \ \ \ \ \ \ \ \ \ \ \ \ \ \ \ \ \ \ \textbf{log}(\texttt{"{}Uploaded\ assignment"{}},\ uid,\ path);\ \ \ \ \ \ \ \ \ \ \ \ \ \ \ \ \ \ \ \ \ \ \ \ \ \ \ \ \ \ \ \ \ \ \ \ \ \ \ \ \ \ \ \ \ \ \ \  \\
\mbox{}\ \ \ \ \ \ \ \ \ \ \ \ \ \ \ \ \ \ \ \ \ \ \ \ \ \ \ \ \ \ \ \ \ \ \ \ \ \ \ \ \} \\
\mbox{}\ \ \ \ \ \ \ \ \ \ \ \ \ \ \ \ \ \ \ \ \ \ \ \ \ \ \ \ \ \ \ \ \ \ \ \ \ \ \ \ \textbf{break}; \\
\mbox{}\ \ \ \ \ \ \ \ \ \ \ \ \ \ \ \ \ \ \ \ \ \ \ \ \ \ \ \ \ \ \ \ \textbf{case}\ MARK: \\
\mbox{}\ \ \ \ \ \ \ \ \ \ \ \ \ \ \ \ \ \ \ \ \ \ \ \ \ \ \ \ \ \ \ \ \ \ \ \ \ \ \ \ \textbf{view$\_$marks}(uid); \\
\mbox{}\ \ \ \ \ \ \ \ \ \ \ \ \ \ \ \ \ \ \ \ \ \ \ \ \ \ \ \ \ \ \ \ \ \ \ \ \ \ \ \ \textbf{break}; \\
\mbox{}\ \ \ \ \ \ \ \ \ \ \ \ \ \ \ \ \ \ \ \ \ \ \ \ \ \ \ \ \ \ \ \ \textbf{case}\ HELP: \\
\mbox{}\ \ \ \ \ \ \ \ \ \ \ \ \ \ \ \ \ \ \ \ \ \ \ \ \ \ \ \ \ \ \ \ \ \ \ \ \ \ \ \ \textbf{printf}(\texttt{"{}}\texttt{\textbackslash{}t}\texttt{\%c\ -\ Get\ Assignment.}\texttt{\textbackslash{}n\textbackslash{}t}\texttt{\%c\ -\ Submit\ Assignment.}\texttt{\textbackslash{}n\textbackslash{}t}\texttt{\%c\ -\ View\ Marks.}\texttt{\textbackslash{}n\textbackslash{}t}\texttt{\%c\ -\ Quit.}\texttt{\textbackslash{}n}\texttt{"{}},\  \\
\mbox{}\ \ \ \ \ \ \ \ \ \ \ \ \ \ \ \ \ \ \ \ \ \ \ \ \ \ \ \ \ \ \ \ \ \ \ \ \ \ \ \ \ \ \ \ \ \ \ \ GET,\  \\
\mbox{}\ \ \ \ \ \ \ \ \ \ \ \ \ \ \ \ \ \ \ \ \ \ \ \ \ \ \ \ \ \ \ \ \ \ \ \ \ \ \ \ \ \ \ \ \ \ \ \ SUBMIT,\  \\
\mbox{}\ \ \ \ \ \ \ \ \ \ \ \ \ \ \ \ \ \ \ \ \ \ \ \ \ \ \ \ \ \ \ \ \ \ \ \ \ \ \ \ \ \ \ \ \ \ \ \ MARK,\  \\
\mbox{}\ \ \ \ \ \ \ \ \ \ \ \ \ \ \ \ \ \ \ \ \ \ \ \ \ \ \ \ \ \ \ \ \ \ \ \ \ \ \ \ \ \ \ \ \ \ \ \ QUIT); \\
\mbox{}\ \ \ \ \ \ \ \ \ \ \ \ \ \ \ \ \ \ \ \ \ \ \ \ \ \ \ \ \ \ \ \ \ \ \ \ \ \ \ \ option\ =\ NONE;\ \textit{//\ To\ allow\ looping} \\
\mbox{}\ \ \ \ \ \ \ \ \ \ \ \ \ \ \ \ \ \ \ \ \ \ \ \ \ \ \ \ \ \ \ \ \ \ \ \ \ \ \ \ \textbf{break}; \\
\mbox{}\ \ \ \ \ \ \ \ \ \ \ \ \ \ \ \ \ \ \ \ \ \ \ \ \ \ \ \ \ \ \ \ \textbf{case}\ QUIT: \\
\mbox{}\ \ \ \ \ \ \ \ \ \ \ \ \ \ \ \ \ \ \ \ \ \ \ \ \ \ \ \ \ \ \ \ \ \ \ \ \ \ \ \ \textbf{break}; \\
\mbox{}\ \ \ \ \ \ \ \ \ \ \ \ \ \ \ \ \ \ \ \ \ \ \ \ \ \ \ \ \ \ \ \ \textbf{default}: \\
\mbox{}\ \ \ \ \ \ \ \ \ \ \ \ \ \ \ \ \ \ \ \ \ \ \ \ \ \ \ \ \ \ \ \ \ \ \ \ \ \ \ \ \textbf{printf}(\texttt{"{}Unrecognised\ option\ '\%s'}\texttt{\textbackslash{}n}\texttt{"{}},\ \&option); \\
\mbox{}\ \ \ \ \ \ \ \ \ \ \ \ \ \ \ \ \ \ \ \ \ \ \ \ \ \ \ \ \ \ \ \ \ \ \ \ \ \ \ \ option\ =\ NONE; \\
\mbox{}\ \ \ \ \ \ \ \ \ \ \ \ \ \ \ \ \ \ \ \ \ \ \ \ \} \\
\mbox{}\ \ \ \ \ \ \ \ \ \ \ \ \ \ \ \ \} \\
\mbox{}\ \ \ \ \ \ \ \ \} \\
\mbox{}\ \ \ \ \ \ \ \ \textbf{free$\_$logger}(); \\
\mbox{}\ \ \ \ \ \ \ \ \textbf{return}\ EXIT$\_$SUCCESS; \\
\mbox{}\} \\

\clearpage
\normalsize
\rmfamily
\subsection{user.c}
\scriptsize
\sffamily
% Generator: GNU source-highlight, by Lorenzo Bettini, http://www.gnu.org/software/src-highlite
\noindent
\mbox{}\textbf{\#include}\ \texttt{$<$stdlib.h$>$} \\
\mbox{}\textbf{\#include}\ \texttt{$<$stdio.h$>$} \\
\mbox{}\textbf{\#include}\ \texttt{$<$sys/types.h$>$} \\
\mbox{}\textbf{\#include}\ \texttt{$<$dirent.h$>$} \\
\mbox{}\textbf{\#include}\ \texttt{$<$sys/stat.h$>$} \\
\mbox{}\textbf{\#include}\ \texttt{$<$string.h$>$} \\
\mbox{} \\
\mbox{}\textbf{\#include}\ \texttt{$<$termios.h$>$} \\
\mbox{}\textbf{\#include}\ \texttt{$<$unistd.h$>$} \\
\mbox{} \\
\mbox{}\textbf{\#include}\ \texttt{"{}user.h"{}} \\
\mbox{}\textbf{\#include}\ \texttt{"{}utils.h"{}} \\
\mbox{}\textbf{\#include}\ \texttt{"{}file$\_$lock.h"{}} \\
\mbox{} \\
\mbox{}void\ \textbf{set$\_$stdin$\_$echo}(int\ enable)\ \{ \\
\mbox{}\ \ \ \ \textbf{struct}\ termios\ tty; \\
\mbox{}\ \ \ \ \textbf{tcgetattr}(STDIN$\_$FILENO,\ \&tty); \\
\mbox{}\ \ \ \ \textbf{if}\ (!enable) \\
\mbox{}\ \ \ \ \ \ \ \ tty.c$\_$lflag\ \&=\ \textasciitilde{}ECHO; \\
\mbox{}\ \ \ \ \textbf{else} \\
\mbox{}\ \ \ \ \ \ \ \ tty.c$\_$lflag\ $|$=\ ECHO; \\
\mbox{} \\
\mbox{}\ \ \ \ (void)\ \textbf{tcsetattr}(STDIN$\_$FILENO,\ TCSANOW,\ \&tty); \\
\mbox{}\} \\
\mbox{} \\
\mbox{}int\ \textbf{login}(char\ *\ uid)\ \{ \\
\mbox{}\ \ \ \ \ \ \ \ \textbf{set$\_$stdin$\_$echo}(0); \\
\mbox{}\ \ \ \ \ \ \ \ char\ passwd[13]; \\
\mbox{}\ \ \ \ \ \ \ \ \textbf{printf}(\texttt{"{}\%s's\ password:\ "{}},\ uid); \\
\mbox{}\ \ \ \ \ \ \ \ \textbf{scanf}(\texttt{"{}\%12s"{}},\ passwd); \\
\mbox{}\ \ \ \ \ \ \ \ \textbf{set$\_$stdin$\_$echo}(1); \\
\mbox{} \\
\mbox{}\ \ \ \ \ \ \ \ char\ *\ base$\_$dir\ =\ \textbf{malloc}(115\ *\ \textbf{sizeof}(char)); \\
\mbox{}\ \ \ \ \ \ \ \ \textbf{get$\_$base$\_$dir}(base$\_$dir); \\
\mbox{}\ \ \ \ \ \ \ \ \textbf{strcat}(base$\_$dir,\ \texttt{"{}/.auth"{}}); \\
\mbox{} \\
\mbox{}\ \ \ \ \ \ \ \ int\ auth$\_$fd\ =\ \textbf{lock}(base$\_$dir); \\
\mbox{}\ \ \ \ \ \ \ \  \\
\mbox{}\ \ \ \ \ \ \ \ \textbf{if}(auth$\_$fd\ ==\ -1)\ \{ \\
\mbox{}\ \ \ \ \ \ \ \ \ \ \ \ \ \ \ \ \textbf{printf}(\texttt{"{}Error:\ .auth\ file\ not\ found"{}}); \\
\mbox{}\ \ \ \ \ \ \ \ \ \ \ \ \ \ \ \ \textbf{free}(base$\_$dir); \\
\mbox{}\ \ \ \ \ \ \ \ \ \ \ \ \ \ \ \ \textbf{return}\ 0; \\
\mbox{}\ \ \ \ \ \ \ \ \} \\
\mbox{} \\
\mbox{}\ \ \ \ \ \ \ \ FILE\ *\ auth\ =\ \textbf{fdopen}(auth$\_$fd,\ \texttt{"{}r"{}}); \\
\mbox{}\ \ \ \ \ \ \ \ char\ line[100]; \\
\mbox{}\ \ \ \ \ \ \ \ char\ user$\_$line[100]; \\
\mbox{}\ \ \ \ \ \ \ \ int\ autherised\ =\ 0; \\
\mbox{} \\
\mbox{}\ \ \ \ \ \ \ \ \textbf{strcpy}(user$\_$line,\ uid); \\
\mbox{}\ \ \ \ \ \ \ \ \textbf{strcat}(user$\_$line,\ \texttt{"{}\ stdnt\ "{}}); \\
\mbox{}\ \ \ \ \ \ \ \ \textbf{strcat}(user$\_$line,\ passwd); \\
\mbox{}\ \ \ \ \ \ \ \ \textbf{strcat}(user$\_$line,\ \texttt{"{}}\texttt{\textbackslash{}n}\texttt{"{}}); \\
\mbox{}\ \ \ \ \ \ \ \ \textbf{while}(\textbf{fgets}(line,\ 100,\ auth)\ !=\ NULL)\ \{ \\
\mbox{}\ \ \ \ \ \ \ \ \ \ \ \ \ \ \ \ \textbf{if}(\textbf{strcmp}(user$\_$line,\ line)\ ==\ 0)\ \{ \\
\mbox{}\ \ \ \ \ \ \ \ \ \ \ \ \ \ \ \ \ \ \ \ \ \ \ \ autherised\ =\ 1; \\
\mbox{}\ \ \ \ \ \ \ \ \ \ \ \ \ \ \ \ \} \\
\mbox{}\ \ \ \ \ \ \ \ \} \\
\mbox{} \\
\mbox{}\ \ \ \ \ \ \ \ \textbf{fclose}(auth); \\
\mbox{}\ \ \ \ \ \ \ \ \textbf{unlock}(auth$\_$fd); \\
\mbox{}\ \ \ \ \ \ \ \ \textbf{free}(base$\_$dir); \\
\mbox{}\ \ \ \ \ \ \ \ \textbf{return}\ autherised;\ \ \ \ \ \ \ \  \\
\mbox{}\} \\
\mbox{} \\
\mbox{}int\ \textbf{download$\_$assignment}(char\ *\ dest$\_$path)\ \{ \\
\mbox{}\ \ \ \ \ \ \ \ char\ *\ src$\_$path\ =\ \textbf{malloc}(115\ *\ \textbf{sizeof}(char));\ \ \ \ \ \ \ \  \\
\mbox{}\ \ \ \ \ \ \ \ DIR\ *\ assignment$\_$dir; \\
\mbox{}\ \ \ \ \ \ \ \ \textbf{struct}\ dirent\ *\ dir$\_$entry; \\
\mbox{}\ \ \ \ \ \ \ \ \textbf{struct}\ stat\ entry$\_$stat;\  \\
\mbox{}\ \ \ \ \ \ \ \ int\ has$\_$file\ =\ 0; \\
\mbox{} \\
\mbox{}\ \ \ \ \ \ \ \ \textbf{get$\_$base$\_$dir}(src$\_$path); \\
\mbox{}\ \ \ \ \ \ \ \ \textbf{strcat}(src$\_$path,\ \texttt{"{}/assignment/"{}}); \\
\mbox{}\ \ \ \ \ \ \ \ assignment$\_$dir\ =\ \textbf{opendir}(src$\_$path); \\
\mbox{}\ \ \ \ \ \ \ \  \\
\mbox{}\ \ \ \ \ \ \ \ \textbf{while}((dir$\_$entry\ =\ \textbf{readdir}(assignment$\_$dir))\ !=\ NULL)\ \{ \\
\mbox{}\ \ \ \ \ \ \ \ \ \ \ \ \ \ \ \ \textbf{printf}(\texttt{"{}\%s}\texttt{\textbackslash{}n}\texttt{"{}},dir$\_$entry-$>$d$\_$name); \\
\mbox{}\ \ \ \ \ \ \ \ \ \ \ \ \ \ \ \ \textbf{if}(\textbf{stat}(dir$\_$entry-$>$d$\_$name,\ \&entry$\_$stat)\ ==\ -1)\ \{ \\
\mbox{}\ \ \ \ \ \ \ \ \ \ \ \ \ \ \ \ \}\  \\
\mbox{}\ \ \ \ \ \ \ \ \ \ \ \ \ \ \ \  \\
\mbox{}\ \ \ \ \ \ \ \ \ \ \ \ \ \ \ \ \textbf{if}(\textbf{S$\_$ISREG}(entry$\_$stat.st$\_$mode))\ \{ \\
\mbox{}\ \ \ \ \ \ \ \ \ \ \ \ \ \ \ \ \ \ \ \ \ \ \ \ \textbf{strcat}(src$\_$path,\ dir$\_$entry-$>$d$\_$name); \\
\mbox{}\ \ \ \ \ \ \ \ \ \ \ \ \ \ \ \ \ \ \ \ \ \ \ \ has$\_$file\ =\ 1; \\
\mbox{}\ \ \ \ \ \ \ \ \ \ \ \ \ \ \ \ \ \ \ \ \ \ \ \ \textbf{break}; \\
\mbox{}\ \ \ \ \ \ \ \ \ \ \ \ \ \ \ \ \} \\
\mbox{}\ \ \ \ \ \ \ \ \} \\
\mbox{} \\
\mbox{}\ \ \ \ \ \ \ \ \textbf{closedir}(assignment$\_$dir); \\
\mbox{} \\
\mbox{}\ \ \ \ \ \ \ \ \textbf{if}(has$\_$file\ ==\ 0)\ \{ \\
\mbox{}\ \ \ \ \ \ \ \ \ \ \ \ \ \ \ \ \textbf{printf}(\texttt{"{}No\ assignment\ file\ found}\texttt{\textbackslash{}n}\texttt{"{}}); \\
\mbox{}\ \ \ \ \ \ \ \ \ \ \ \ \ \ \ \ \textbf{return}\ 0; \\
\mbox{}\ \ \ \ \ \ \ \ \} \\
\mbox{}\ \ \ \ \ \ \ \  \\
\mbox{}\ \ \ \ \ \ \ \ int\ ret\ =\ \textbf{copy}(src$\_$path,\ dest$\_$path); \\
\mbox{}\ \ \ \ \ \ \ \ \textbf{free}(src$\_$path); \\
\mbox{}\ \ \ \ \ \ \ \ \textbf{return}\ ret; \\
\mbox{}\} \\
\mbox{} \\
\mbox{}int\ \textbf{submit$\_$assignment}(char\ *\ src$\_$path,\ char\ *\ uid)\ \{ \\
\mbox{}\ \ \ \ \ \ \ \ char\ *\ dest$\_$path\ =\ \textbf{malloc}(115\ *\ \textbf{sizeof}(char)); \\
\mbox{}\ \ \ \ \ \ \ \ \textbf{get$\_$base$\_$dir}(dest$\_$path); \\
\mbox{}\ \ \ \ \ \ \ \ \textbf{strcat}(dest$\_$path,\ \texttt{"{}/students/"{}}); \\
\mbox{}\ \ \ \ \ \ \ \ \textbf{strcat}(dest$\_$path,\ uid); \\
\mbox{}\ \ \ \ \ \ \ \ \textbf{strcat}(dest$\_$path,\ \texttt{"{}/"{}}); \\
\mbox{} \\
\mbox{}\ \ \ \ \ \ \ \ char\ *\ tokens\ =\ \textbf{strtok}(src$\_$path,\ \texttt{"{}/"{}}); \\
\mbox{}\ \ \ \ \ \ \ \ char\ *\ filename; \\
\mbox{} \\
\mbox{}\ \ \ \ \ \ \ \ \textbf{while}(tokens\ !=\ NULL)\ \{ \\
\mbox{}\ \ \ \ \ \ \ \ \ \ \ \ \ \ \ \ filename\ =\ tokens; \\
\mbox{}\ \ \ \ \ \ \ \ \ \ \ \ \ \ \ \ tokens\ =\ \textbf{strtok}(NULL,\ \texttt{"{}/"{}}); \\
\mbox{}\ \ \ \ \ \ \ \ \} \\
\mbox{}\ \ \ \ \ \ \ \ \textbf{strcat}(dest$\_$path,\ filename); \\
\mbox{}\ \ \ \ \ \ \ \ int\ ret\ =\ \textbf{copy}(src$\_$path,\ dest$\_$path); \\
\mbox{}\ \ \ \ \ \ \ \ \textbf{free}(dest$\_$path); \\
\mbox{}\ \ \ \ \ \ \ \ \textbf{return}\ ret; \\
\mbox{}\} \\
\mbox{} \\
\mbox{}void\ \textbf{view$\_$marks}(char\ *\ uid)\ \{ \\
\mbox{}\ \ \ \ \ \ \ \ char\ *\ dest$\_$path\ =\ \textbf{malloc}(115\ *\ \textbf{sizeof}(char)); \\
\mbox{}\ \ \ \ \ \ \ \ \textbf{get$\_$base$\_$dir}(dest$\_$path); \\
\mbox{}\ \ \ \ \ \ \ \ \textbf{strcat}(dest$\_$path,\ \texttt{"{}/students/"{}}); \\
\mbox{}\ \ \ \ \ \ \ \ \textbf{strcat}(dest$\_$path,\ uid); \\
\mbox{}\ \ \ \ \ \ \ \ \textbf{strcat}(dest$\_$path,\ \texttt{"{}/results/result.txt"{}}); \\
\mbox{} \\
\mbox{}\ \ \ \ \ \ \ \ int\ dest$\_$fd\ =\ \textbf{lock}(dest$\_$path); \\
\mbox{}\ \ \ \ \ \ \ \ FILE\ *\ dest\ =\ \textbf{fdopen}(dest$\_$fd,\ \texttt{"{}r"{}}); \\
\mbox{}\ \ \ \ \ \ \ \  \\
\mbox{}\ \ \ \ \ \ \ \ \textbf{if}(dest\ ==\ NULL)\ \{ \\
\mbox{}\ \ \ \ \ \ \ \ \ \ \ \ \ \ \ \ \textbf{printf}(\texttt{"{}No\ results\ found\ for\ user\ '\%s'}\texttt{\textbackslash{}n}\texttt{"{}},\ uid); \\
\mbox{}\ \ \ \ \ \ \ \ \ \ \ \ \ \ \ \ \textbf{free}(dest$\_$path); \\
\mbox{}\ \ \ \ \ \ \ \ \ \ \ \ \ \ \ \ \textbf{return}; \\
\mbox{}\ \ \ \ \ \ \ \ \}\ \ \ \ \ \ \ \  \\
\mbox{} \\
\mbox{}\ \ \ \ \ \ \ \ int\ mark; \\
\mbox{}\ \ \ \ \ \ \ \ char\ comments[200]; \\
\mbox{}\ \ \ \ \ \ \ \ \textbf{fscanf}(dest,\ \texttt{"{}\%d\%\%}\texttt{\textbackslash{}n}\texttt{"{}},\ \&mark); \\
\mbox{}\ \ \ \ \ \ \ \ \textbf{fgets}(comments,\ 200,\ dest); \\
\mbox{} \\
\mbox{}\ \ \ \ \ \ \ \ char\ *\ grade; \\
\mbox{} \\
\mbox{}\ \ \ \ \ \ \ \ \textbf{if}\ (mark\ $<$=\ 15) \\
\mbox{}\ \ \ \ \ \ \ \ \ \ \ \ \ \ \ \ grade\ =\ \texttt{"{}F-"{}}; \\
\mbox{}\ \ \ \ \ \ \ \ \textbf{else}\ \textbf{if}\ (mark\ $<$=\ 30) \\
\mbox{}\ \ \ \ \ \ \ \ \ \ \ \ \ \ \ \ grade\ =\ \texttt{"{}F"{}}; \\
\mbox{}\ \ \ \ \ \ \ \ \textbf{else}\ \textbf{if}\ (mark\ $<$=\ 34) \\
\mbox{}\ \ \ \ \ \ \ \ \ \ \ \ \ \ \ \ grade\ =\ \texttt{"{}F+"{}}; \\
\mbox{}\ \ \ \ \ \ \ \ \textbf{else}\ \textbf{if}\ (mark\ $<$=\ 39) \\
\mbox{}\ \ \ \ \ \ \ \ \ \ \ \ \ \ \ \ grade\ =\ \texttt{"{}E"{}}; \\
\mbox{}\ \ \ \ \ \ \ \ \textbf{else}\ \textbf{if}\ (mark\ $<$=\ 43) \\
\mbox{}\ \ \ \ \ \ \ \ \ \ \ \ \ \ \ \ grade\ =\ \texttt{"{}D-"{}}; \\
\mbox{}\ \ \ \ \ \ \ \ \textbf{else}\ \textbf{if}\ (mark\ $<$=\ 46) \\
\mbox{}\ \ \ \ \ \ \ \ \ \ \ \ \ \ \ \ grade\ =\ \texttt{"{}D"{}}; \\
\mbox{}\ \ \ \ \ \ \ \ \textbf{else}\ \textbf{if}\ (mark\ $<$=\ 59) \\
\mbox{}\ \ \ \ \ \ \ \ \ \ \ \ \ \ \ \ grade\ =\ \texttt{"{}D+"{}}; \\
\mbox{}\ \ \ \ \ \ \ \ \textbf{else}\ \textbf{if}\ (mark\ $<$=\ 53) \\
\mbox{}\ \ \ \ \ \ \ \ \ \ \ \ \ \ \ \ grade\ =\ \texttt{"{}C-"{}}; \\
\mbox{}\ \ \ \ \ \ \ \ \textbf{else}\ \textbf{if}\ (mark\ $<$=\ 56) \\
\mbox{}\ \ \ \ \ \ \ \ \ \ \ \ \ \ \ \ grade\ =\ \texttt{"{}C"{}}; \\
\mbox{}\ \ \ \ \ \ \ \ \textbf{else}\ \textbf{if}\ (mark\ $<$=\ 59) \\
\mbox{}\ \ \ \ \ \ \ \ \ \ \ \ \ \ \ \ grade\ =\ \texttt{"{}C+"{}}; \\
\mbox{}\ \ \ \ \ \ \ \ \textbf{else}\ \textbf{if}\ (mark\ $<$=\ 63) \\
\mbox{}\ \ \ \ \ \ \ \ \ \ \ \ \ \ \ \ grade\ =\ \texttt{"{}B-"{}}; \\
\mbox{}\ \ \ \ \ \ \ \ \textbf{else}\ \textbf{if}\ (mark\ $<$=\ 66) \\
\mbox{}\ \ \ \ \ \ \ \ \ \ \ \ \ \ \ \ grade\ =\ \texttt{"{}B"{}}; \\
\mbox{}\ \ \ \ \ \ \ \ \textbf{else}\ \textbf{if}\ (mark\ $<$=\ 69) \\
\mbox{}\ \ \ \ \ \ \ \ \ \ \ \ \ \ \ \ grade\ =\ \texttt{"{}B+"{}}; \\
\mbox{}\ \ \ \ \ \ \ \ \textbf{else}\ \textbf{if}\ (mark\ $<$=\ 79) \\
\mbox{}\ \ \ \ \ \ \ \ \ \ \ \ \ \ \ \ grade\ =\ \texttt{"{}A-"{}}; \\
\mbox{}\ \ \ \ \ \ \ \ \textbf{else}\ \textbf{if}\ (mark\ $<$=\ 89) \\
\mbox{}\ \ \ \ \ \ \ \ \ \ \ \ \ \ \ \ grade\ =\ \texttt{"{}A"{}}; \\
\mbox{}\ \ \ \ \ \ \ \ \textbf{else}\ \textbf{if}\ (mark\ $<$=\ 95) \\
\mbox{}\ \ \ \ \ \ \ \ \ \ \ \ \ \ \ \ grade\ =\ \texttt{"{}A+"{}}; \\
\mbox{}\ \ \ \ \ \ \ \ \textbf{else} \\
\mbox{}\ \ \ \ \ \ \ \ \ \ \ \ \ \ \ \ grade\ =\ \texttt{"{}A++"{}}; \\
\mbox{} \\
\mbox{}\ \ \ \ \ \ \ \ \textbf{printf}(\texttt{"{}Results\ for\ user\ '\%s':}\texttt{\textbackslash{}n\textbackslash{}t}\texttt{Mark:\ \%d\%\%\ (\%s)}\texttt{\textbackslash{}n\textbackslash{}t}\texttt{Comments:\ \%s}\texttt{\textbackslash{}n}\texttt{"{}},\ uid,\ mark,\ grade,\ comments); \\
\mbox{}\ \ \ \ \ \ \ \  \\
\mbox{}\ \ \ \ \ \ \ \ \textbf{fclose}(dest); \\
\mbox{}\ \ \ \ \ \ \ \ \textbf{unlock}(dest$\_$fd); \\
\mbox{}\ \ \ \ \ \ \ \ \textbf{free}(dest$\_$path); \\
\mbox{}\} \\
\mbox{} \\
\mbox{}int\ \textbf{copy}(char\ *\ src$\_$path,\ char\ *\ dest$\_$path)\ \{ \\
\mbox{}\ \ \ \ \ \ \ \ FILE\ *\ dest; \\
\mbox{}\ \ \ \ \ \ \ \ FILE\ *\ src; \\
\mbox{}\ \ \ \ \ \ \ \ int\ src$\_$fd; \\
\mbox{}\ \ \ \ \ \ \ \ int\ reader; \\
\mbox{} \\
\mbox{}\ \ \ \ \ \ \ \ src$\_$fd\ =\ \textbf{lock}(src$\_$path); \\
\mbox{}\ \ \ \ \ \ \ \ \textbf{if}(src$\_$fd\ ==\ -1)\ \{ \\
\mbox{}\ \ \ \ \ \ \ \ \ \ \ \ \ \ \ \ \textbf{return}\ 0; \\
\mbox{}\ \ \ \ \ \ \ \ \} \\
\mbox{}\ \ \ \ \ \ \ \ src\ =\ \textbf{fdopen}(src$\_$fd,\ \texttt{"{}rb"{}}); \\
\mbox{}\ \ \ \ \ \ \ \ dest\ =\ \textbf{fopen}(dest$\_$path,\ \texttt{"{}wb"{}}); \\
\mbox{}\ \ \ \ \ \ \ \  \\
\mbox{}\ \ \ \ \ \ \ \ \textbf{printf}(\texttt{"{}Copying."{}}); \\
\mbox{}\ \ \ \ \ \ \ \ \textbf{while}(1)\ \{ \\
\mbox{}\ \ \ \ \ \ \ \ \ \ \ \ \ \ \ \ reader\ =\ \textbf{fgetc}(src); \\
\mbox{}\ \ \ \ \ \ \ \ \ \ \ \ \ \ \ \ \textbf{if}(reader\ ==\ EOF)\ \textbf{break}; \\
\mbox{}\ \ \ \ \ \ \ \ \ \ \ \ \ \ \ \ \textbf{fputc}((int)\ reader,\ dest); \\
\mbox{}\ \ \ \ \ \ \ \ \} \\
\mbox{}\ \ \ \ \ \ \ \ \textbf{printf}(\texttt{"{}}\texttt{\textbackslash{}n}\texttt{"{}}); \\
\mbox{} \\
\mbox{}\ \ \ \ \ \ \ \ \textbf{fclose}(src); \\
\mbox{}\ \ \ \ \ \ \ \ \textbf{fclose}(dest); \\
\mbox{}\ \ \ \ \ \ \ \ \textbf{unlock}(src$\_$fd); \\
\mbox{}\textit{//\ \ \ \ \ \ \ \ free(src$\_$path);} \\
\mbox{}\ \ \ \ \ \ \ \ \textbf{return}\ 1; \\
\mbox{}\} \\

\clearpage
\normalsize
\rmfamily
\subsection{user.h}
\scriptsize
\sffamily
% Generator: GNU source-highlight, by Lorenzo Bettini, http://www.gnu.org/software/src-highlite
\noindent
\mbox{}\textit{/*\ user.h\ Copyright\ (c)\ Alexander\ Brown,\ March\ 2011\ */} \\
\mbox{} \\
\mbox{}\textit{/**} \\
\mbox{}\textit{\ *\ }@brief\textit{\ \ \ \ \ \ \ \ Defines\ operations\ for\ a\ user.} \\
\mbox{}\textit{\ *\ }@author\textit{\ \ \ \ \ \ \ \ Alexander\ Brown.} \\
\mbox{}\textit{\ *\ }@version\textit{\ \ \ \ \ \ \ \ 1.0} \\
\mbox{}\textit{\ */} \\
\mbox{} \\
\mbox{}\textbf{\#ifndef}\ $\_$USER$\_$H \\
\mbox{}\textbf{\#define}\ $\_$USER$\_$H \\
\mbox{} \\
\mbox{}\textit{/**} \\
\mbox{}\textit{\ *\ Logs\ in\ a\ user.} \\
\mbox{}\textit{\ *\ }@param\textit{\ uid\ \ \ \ \ \ \ \ \ The\ user\ id\ of\ the\ user.} \\
\mbox{}\textit{\ *\ }@return\textit{\ \ \ \ \ \ \ \ \ }\textbf{$<$code$>$}\textit{1}\textbf{$<$/code$>$}\textit{\ -\ If\ the\ operation\ was\ sucessful.} \\
\mbox{}\textit{\ */} \\
\mbox{}int\ \textbf{login}(char\ *\ uid); \\
\mbox{} \\
\mbox{}\textit{/**} \\
\mbox{}\textit{\ *\ Downloads\ the\ assignment.} \\
\mbox{}\textit{\ *\ }@param\textit{\ dest$\_$path\ \ \ \ \ \ \ \ The\ path\ of\ the\ file\ to\ download\ to.} \\
\mbox{}\textit{\ *\ }@return\textit{\ \ \ \ \ \ \ \ \ \ \ \ \ \ \ \ }\textbf{$<$code$>$}\textit{1}\textbf{$<$/code$>$}\textit{\ -\ If\ the\ operation\ was\ sucessful.} \\
\mbox{}\textit{\ */} \\
\mbox{}int\ \textbf{download$\_$assignment}(char\ *\ dest$\_$path); \\
\mbox{} \\
\mbox{} \\
\mbox{}int\ \textbf{submit$\_$assignment}(char\ *\ src$\_$path,\ char\ *\ uid); \\
\mbox{} \\
\mbox{}\textit{/**} \\
\mbox{}\textit{\ *\ Views\ the\ marks\ and\ comments\ for\ this\ assignment.} \\
\mbox{}\textit{\ *\ }@param\textit{\ uid\ \ \ \ \ \ \ \ The\ User\ ID\ of\ the\ user.} \\
\mbox{}\textit{\ */} \\
\mbox{}void\ \textbf{view$\_$marks}(char\ *\ uid); \\
\mbox{} \\
\mbox{}\textit{/**} \\
\mbox{}\textit{\ *\ Copies\ the\ source\ file\ to\ the\ destination\ file.} \\
\mbox{}\textit{\ *\ }@param\textit{\ src$\_$path\ \ \ \ \ \ \ \ The\ path\ to\ the\ source\ file.} \\
\mbox{}\textit{\ *\ }@param\textit{\ dest$\_$path\ \ \ \ \ \ \ \ The\ path\ to\ the\ destination\ file.} \\
\mbox{}\textit{\ *\ }@return\textit{\ \ \ \ \ \ \ \ \ \ \ \ \ \ \ \ \ }\textbf{$<$code$>$}\textit{1}\textbf{$<$/code$>$}\textit{\ -\ If\ the\ operation\ was\ sucessful.} \\
\mbox{}\textit{\ */} \\
\mbox{}int\ \textbf{copy}(char\ *\ src$\_$path,\ char\ *\ dest$\_$path); \\
\mbox{} \\
\mbox{}\textbf{\#endif}\ \textit{/*\ $\_$USER$\_$H\ */} \\

\clearpage
\normalsize
\rmfamily
\subsection{utils.c}
\scriptsize
\sffamily
% Generator: GNU source-highlight, by Lorenzo Bettini, http://www.gnu.org/software/src-highlite
\noindent
\mbox{}\textit{/*} \\
\mbox{}\textit{\ *\ utils.c\ copyright\ (c)\ Alexander\ Brown,\ March\ 2011} \\
\mbox{}\textit{\ */} \\
\mbox{} \\
\mbox{} \\
\mbox{}\textbf{\#include}\ \texttt{$<$stdlib.h$>$} \\
\mbox{}\textbf{\#include}\ \texttt{$<$stdio.h$>$} \\
\mbox{} \\
\mbox{}\textbf{\#include}\ \texttt{"{}utils.h"{}} \\
\mbox{} \\
\mbox{} \\
\mbox{}void\ \textbf{get$\_$base$\_$dir}(char\ *\ base$\_$dir)\ \{ \\
\mbox{}\ \ \ \ \ \ \ \ FILE\ *\ file\ =\ \textbf{fopen}(\texttt{"{}config"{}},\ \texttt{"{}r"{}}); \\
\mbox{}\ \ \ \ \ \ \ \ \textbf{if}(file\ ==\ NULL)\ \{ \\
\mbox{}\ \ \ \ \ \ \ \ \ \ \ \ \ \ \ \ \textbf{printf}(\texttt{"{}config\ file\ not\ found."{}}); \\
\mbox{}\ \ \ \ \ \ \ \ \ \ \ \ \ \ \ \ \textbf{exit}(1); \\
\mbox{}\ \ \ \ \ \ \ \ \} \\
\mbox{}\ \ \ \ \ \ \ \ \textbf{fscanf}(file,\ \texttt{"{}\%s}\texttt{\textbackslash{}n}\texttt{"{}},\ base$\_$dir); \\
\mbox{}\ \ \ \ \ \ \ \ \textbf{fclose}(file); \\
\mbox{}\} \\
\mbox{} \\

\clearpage
\normalsize
\rmfamily
\subsection{utils.h}
\scriptsize
\sffamily
% Generator: GNU source-highlight, by Lorenzo Bettini, http://www.gnu.org/software/src-highlite
\noindent
\mbox{}\textit{/**} \\
\mbox{}\textit{\ *\ }@author\textit{\ Alexander\ Brown.} \\
\mbox{}\textit{\ */} \\
\mbox{} \\
\mbox{}\textbf{\#ifndef}\ $\_$UTILS$\_$H \\
\mbox{}\textbf{\#define}\ $\_$UTILS$\_$G \\
\mbox{} \\
\mbox{}void\ \textbf{get$\_$base$\_$dir}(char\ *\ base$\_$dir); \\
\mbox{} \\
\mbox{}\textbf{\#endif}\ \textit{/*\ $\_$UTILS$\_$H\ */} \\

\clearpage

\end{document}
